#1. Regarding a 'Human Stomach', consider the following statements: \n 1. Human stomach produces sulphuric acid that helps in the digestion of food without harming the stomach. \n 2. During indigestion, the stomach produces too much acid and this causes pain and irritation. Which of the statements given above is/are correct? 
A. 1 only 
B. 2 only 
C. Both 1 and 2 
D. Neither 1 nor 2 
Answer: B
Explanation: Our stomach produces hydrochloric acid. It helps in digestion of food without harming the stomach. During indigestion, the stomach produces too much acid and this causes pain and irritation. To get rid of this pain, people use bases called antacids. These antacids neutralise the excess acid. Magnesium hydroxide (Milk of magnesia), a mild base, is often used for this purpose. 



#2. Consider the following statements: \n 1. The human body works within the pH range of 7.0 to 7.8. \n 2. When pH of rain water is less than 5.6, it is called acid rain. \n 3. Tooth decay starts when the pH value of the mouth increases. \n Which of the statements given above is/are correct? 
A. 1 and 2 only 
B. 2 and 3 only 
C. 3 only 
D. 1, 2 and 3 only 
Answer: A
Explanation: Our body works within the pH range of 7.0 to 7.8. Living organisms can survive only in a narrow range of pH change. When pH of rain water is less than 5.6, it is called acid rain. When acid rain flows into the rivers, it lowers the pH of the river water. The survival of aquatic life in such rivers becomes difficult. Tooth decay starts when pH of the mouth is lower than 5.5. Tooth enamel, made up of calcium phosphate is the hardest substance in the body. It does not dissolve in water, but gets corroded when the pH level in the mouth goes below 5.5. Using toothpastes, which are generally basic, for cleaning the teeth can neutralise the excess acid and thus can prevent tooth decay . 



#3. Match List-I with List-II and using the code given below, select the correct answer \n List-I (matter) List-II (pH value)  \n A. Human blood 1). 5.5 to 7.5  \n B. Milk 2). 7.3 to 7.5  \n C. Human saliva 3). 6.4  \n D. Humarine 4). 6.5 to 7.5
A. A-2 B-3 C-1 D-4 
B. A-2 B-3 C-4 D-1 
C. A-3 B-2 C-1 D-4 
D. A-1 B-3 C-2 D-4 
Answer: A
Explanation: The correct combination of the above list is as follows: (Substance) - (pH value) Human blood - 7.3 to 7.5 Milk - 6.4 Human saliva - 5.5 to 7.5 Human urine - 6.5 to 7.5 In addition, Lemon juice - 2.2 (approx) Gastric juice - 1.2 (approx.) Blood- 7.4 Pure water- 7 Sodium Hydroxide- 14 (Approx.) Milk of magnesia- 10 



#4. Match List-I with List-II and using the code given below, select the correct answer:  \n List I (Natural source) List II (Acid)  \n A. Bee sting    1). Tartaric acid  \n B. Tamarind     2). Methanoic acid  \n C. Tomato       3). Acetic acid  \n D. Vinegar      4). Oxalic acid 
A. A-2 B-1 C-4 D-3 
B. A-2 B-4 C-1 D-3 
C. A-1 B-2 C-3 D-4 
D. A-2 B-1 C-3 D-4 
Answer: A
Explanation: Correct combination of the above list with some other important natural sources containing acids is as follow: Natural source Acid Vinegar Acetic acid Orange Citric acid Tamarind Tartaric acid Tomato Oxalic acid Sour milk (Curd)Lactic acid Lemon Citric acid Ant sting Methanoic acid Nettle sting Methanoic acid 



#5. Which of the following is used in soda-acid fire extinguisher? 
A. Washing soda 
B. Bleaching powder 
C. Baking soda 
D. Sodium chloride 
Answer: C
Explanation: Baking soda is also known as sodium hydrogen carbonate (NaHCO3), sodium bicarbonate, sweet soda and food soda. It is also an ingredient of antacids. Due to its alkaline nature, it relieves the depression of excessive acid in the stomach.It is also used in soda-acid fire extinguishers. 



#6. Consider the following pairs: \n Match-I (Homogeneous mixture) Match-II (Alloy)  \n 1. Mercury and any other metal Amalgam  \n 2. Copper and Zinc Brass  \n 3. Copper and Tin Bronze  \n 4. Lead and Tin Solder 
Which of the following pairs given above is/are correctly matched? 
A. 1 and 4 only 
B. 1 and 2 only 
C. 2 and 3 only 
D. 1, 2, 3 and 4 
Answer: D
Explanation: An alloy is a homogeneous mixture of two or more metals, or a metal and a non-metal. It is prepared by first melting the primary metal, and then, dissolving the other elements in it in definite proportions. It is then cooled to room temperature. If one of the metals is mercury, then the alloy is known as an amalgam. The electrical conductivity and melting point of an alloy is less than that of pure metals. For example, brass, an alloy of copper and zinc (Cu and Zn), and bronze, an alloy of copper and tin (Cu and Sn), are not good conductors of electricity whereas copper is used for making electrical circuits. Solder, an alloy of lead and tin (Pb and Sn), has a low melting point and is used for welding electrical wires together. 


#7. With reference to Non-metals, consider the following statements:  \n 1. Unlike metals, they are neither malleable nor ductile.  \n 2. They are bad conductors of heat and electricity, except for graphite, which conducts electricity.  \n 3. They react with hydrogen of dilute acids to form hydrides.  \n Which of the statements given above is/are correct? 
A. 1 and 2 only 
B. 2 only 
C. 3 only 
D. 1, 2 and 3 
Answer: D
Explanation: 1. Non-metals have properties opposite to that of metals. They are neither malleable nor ductile. 2. They are bad conductors of heat and electricity, except for graphite, which conducts electricity. 3. Non-metals form negatively charged ions by gaining electrons when reacting with metals. 4. Non-metals form oxides which are either acidic or neutral. 5. Non-metals do not displace hydrogen from dilute acids. They react with hydrogen to form hydrides. 


#8. Regarding the physical properties of metals, consider the following pairs: \n 1. Malleability: metals can be beaten into thin sheets. \n 2. Ductility: the ability of metals to be drawn into thin wires. \n 3. Sonorous: metals that produce a sound on striking a hard surface. Which of the pairs given above is/are correctly matched? 
A. 1 only 
B. 2 and 3 only 
C. 3 only 
D. 1, 2 and 3
Answer: D
Explanation: Some metals can be beaten into thin sheets. This property is called malleability. Gold and silver are the most malleable metals. The ability of metals to be drawn into thin wires is called ductility. Gold is the most ductile metal. It is because of their malleability and ductility that metals can be given different shapes according to our needs.The metals that produce a sound on striking a hard surface are said to be sonorous. For instance, school bells are made of metals. 


#9. Regarding the non-metals, consider the following statements:  \n 1. Diamond is the hardest natural substance.  \n 2. Graphite is the poor conductor of electricity. \n Which of the statements given above is/are correct? 
A. 1 only 
B. 2 only 
C. Both 1 and 2 
D. Neither 1 nor 2 
Answer: A
Explanation: Carbon is a non-metal that can exist in different forms. Each form is called an allotrope. Diamond, an allotrope of carbon, is the hardest natural substance known and has a very high melting and boiling point. Graphite, another allotrope of carbon, is a good conductor of electricity. 


#10. Regarding the anodising, which of the following statements is correct? 
A. It is a process of forming a oxide layer on the non-metals. 
B. It is a process of removing the rust layer of metals. 
C. It is a process of removing the oxide layer from aluminium. 
D. It is a process of forming a thick oxide layer of aluminium. 
Answer: D
Explanation: Anodising is a process of forming a thick oxide layer of aluminium. Aluminium develops a thin oxide layer when exposed to air. This aluminium oxide coat makes it resistant to further corrosion. The resistance can be improved further by making the oxide layer thicker. During anodising, a clean aluminium article is made the anode and is electrolysed with dilute sulphuric acid. The oxygen gas evolved at the anode reacts with aluminium to make a thicker protective oxide layer. This oxide layer can be dyed easily to give aluminium articles an attractive finish. 




#1. Regarding the covalent bonds, which of the following statements is/are incorrect? \n 1. Bonds which are formed by the sharing of an electron pair between two atoms are known as Covalent Bonds. \n 2. Covalently bonded molecules are seen to have weak bonds within the molecule, but intermolecular forces are large. \n 3. Covalent compounds are poor conductors of electricity. Select the correct answer using the code given below: 
A. 1 and 2 only 
B. 2 only 
C. 1 and 3 
D. 1, 2 and 3 
Answer: B
Explanation: The bonds which are formed by the sharing of an electron pair between two atoms are known as covalent bonds. Since the electrons are shared between atoms and no charged particles are formed, such covalent compounds are generally poor conductors of electricity as they do not have a free electron to move. Covalently bonded molecules are seen to have strong bonds within the molecule, but intermolecular forces are small. This gives rise to the low melting and boiling points of these compounds. 


#2. Which of the following structures are allotropes of carbon? \n 1. Diamond \n 2. Graphite \n 3. Fullerene Select the correct answer using the code given below: 
A. 1 and 2 only 
B. 2 and 3 only 
C. 1 and 3 only 
D. 1, 2 and 3 
Answer: D
Explanation: Diamond and Graphite are formed by carbon atoms, the difference lies in the manner in which the carbon atoms are bonded to one another. In Diamond, each carbon atom is bonded to four other carbon atoms forming a rigid three-dimensional structure. In graphite, each carbon atom is bonded to three other carbon atoms in the same plane giving a hexagonal array. Fullerenes forms another class of carbon allotropes. The first one to be identified was C-60 which has carbon atoms arranged in the shape of a football. 

#3. Regarding ethanol, which of the following statements is incorrect? 
A. It is used in making tincture iodine and cough syrups. 
B. It tends to positively affect the central nervous system and makes it even stronger. 
C. Consumption of ethanol tends to slow metabolic processes. 
D. Ethanol is a liquid at room temperature. 
Answer: B
Explanation: Ethanol is a good solvent, it is also used in medicines such as tincture iodine, cough syrups, and many tonics. When large quantities of ethanol are consumed, it tends to depress the central nervous system. This results in lack of coordination, mental confusion, drowsiness, lowering of the normal inhibitions, and finally stupor. Consumption of ethanol tends to slow metabolic process. Alcohol contains empty calories and has no nutritional value. It often contributes to malnutrition because the high levels of calories in most alcoholic drinks can account for a large percentage of your daily energy requirementsEthanol is liquid at room temperature. Ethanol is commonly called alcohol and is the active ingredient of all alcoholic drinks. 

#4. With reference to alcohol, consider the following statements: \n 1. Sugarcane juice can be used to prepare molasses which is fermented to give alcohol (ethanol). \n 2. Alcohol can be used as an additive in petrol since it is a cleaner fuel. \n Which of the statements given above is/are correct? 
A. 1 only 
B. 2 only 
C. Both 1 and 2 
D. Neither 1 nor 2 
Answer: C
Explanation: Sugarcane plants are one of the most efficient convertors of sunlight into chemical energy. Sugarcane juice is used to prepare molasses which is fermented to give alcohol (ethanol). Some countries now use alcohol as an additive in petrol since it is a cleaner fuel which gives rise to only carbon dioxide and water on burning in sufficient air (oxygen). As the Alcohol molecule contains oxygen, it allows the engine to more completely combust the fuel, resulting in fewer emissions and thereby reducing the occurrence of environmental pollution. Since alcohol is produced from plants that harness the power of the sun, it is also considered as renewable fuel. 

#5. Which of the following elements are responsible for the hardness of water? 
A. Hydrogen and Oxygen 
B. Calcium and Magnesium 
C. Carboxylic acid 
D. Sodium hydroxide 
Answer: B
Explanation: Bathing foam forms an insoluble substance (scum), this is caused by the reaction of soap with the calcium and magnesium salts, which cause the hardness of water. This problem is overcome by using another class of compounds called detergents as cleansing agents. Detergents are generally ammonium or sulphonate salts of long chain carboxylic acids. The charged ends of these compounds do not form insoluble precipitates with the calcium and magnesium ions in hard water. Thus, they remain effective in hard water. Detergents are usually used to make shampoos and products for cleaning clothes. 

#6. Match List-I with List-II and select the correct answer: \n List-I (Scientist) List-II (Work/theory)  \n A. Johann Wolfgang Döbereiner 1. Law of Octaves  \n B. John Newlands 2. Coinehe term ‘Triads’  \n C. Dmitri Ivanovich Mendeléev 3. Modern Periodic Table  \n D. Henry Moseley 4. Early development of the periodic table of elements
A. A-4 B-1 C-3 D-2 
B. A-2 B-1 C-4 D-3 
C. A-3 B-4 C-1 D-2 
D. A-2 B-3 C-4 D-1 
Answer: B
Explanation: In the year 1817, Johann Wolfgang Döbereiner, a German chemist, tried to arrange the elements with similar properties into groups. He identified some groups having three elements each. So he called these groups 'triads'. In 1866, John Newlands, an English scientist arranged the then known elements in the order of increasing atomic masses. He compared the table to the octaves found in music. Therefore, he called it the ‘Law of Octaves’. The main credit for classifying elements goes to Dmitri Ivanovich Mendeléev, a Russian chemist. He was the most important contributor to the early development of a Periodic Table of elements wherein the elements were arranged on the basis of their fundamental property, the atomic mass, and also on the similarity of chemical properties. In 1913, Henry Moseley showed that the atomic number of an element is a more fundamental property than its atomic mass as described below. Accordingly, Mendeléev’s Periodic Law was modified and atomic number was adopted as the basis of Modern Periodic Table. 

#7. With reference to Mendeléev’s Periodic Table, consider the following statements: \n 1. Mendeléev arranged the elements in the order of their increasing atomic masses. \n 2. Mendeléev left some gaps in Periodic Table \n Which of the statements given above is/are correct? 
A. 1 only 
B. 2 only 
C. Both 1 and 2 
D. Neither 1 nor 2 
Answer: C
Explanation: Mendeleev examined the relationship between the atomic masses of the elements and their physical and chemical properties. He observed that most of the elements got a place in a Periodic Table and were arranged in the order of their increasing atomic masses. It was also observed that there occurs a periodic recurrence of elements with similar physical and chemical properties. Mendeléev left some gaps in his Periodic Table. Instead of looking upon these gaps as defects, Mendeléev boldly predicted the existence of some elements that had not been discovered at that time. 

#8. Regarding the limitations of Mendeléev’s classification, consider the following statements: \n 1. The Position of Hydrogen was ambiguous. \n 2. Isotopes of all elements posed a challenge to Mendeleev’s Periodic Law. \n 3. The atomic masses do not increase in a regular manner in going from one element to the next.\n  Which of the statements given above is/are correct? 
A. 1 and 2 only 
B. 2 only 
C. 1 and 3 only 
D. 1, 2 and 3 
Answer: D
Explanation: Certainly, no fixed position can be given to hydrogen in the Periodic Table. This was the first limitation of Mendeléev’s Periodic Table. He could not assign a correct position to hydrogen in his Table. Isotopes were discovered long after Mendeléev had proposed his periodic classification of elements. Isotopes of an element have similar chemical properties, but different atomic masses. Thus, isotopes of all elements posed a challenge to Mendeleev’s Periodic Law. Another problem was that the atomic masses do not increase in a regular manner in going from one element to the next. So it was not possible to predict how many elements could be discovered between two elements — especially when we consider the heavier elements. 

#9. Regarding the modern periodic table, consider the following statements: \n 1. Properties of elements are a periodic function of their atomic number. \n 2. There are 7 groups and 18 periods. \n Which of the statements given above is/are correct? 
A. 1 only 
B. 2 only 
C. Both 1 and 2 
D. Neither 1 nor 2 
Answer: A
Explanation:  The Modern Periodic Law states: 'Properties of elements are a periodic function of their atomic number'.  The atomic number gives us the number of protons in the nucleus of an atom and this number increases by one in going from one element to the next.  Elements, when arranged in order of increasing atomic number Z, lead us to the classification known as the Modern Periodic Table.  Prediction of properties of elements could be made with more precision when elements were arranged on the basis of increasing atomic number.  The Modern Periodic Table has 18 vertical columns known as ‘groups’ and 7 horizontal rows known as ‘periods’. 

#10. Regarding the modern periodic table, which of the following statements is incorrect? 
A. The elements present in any one group have the same number of valence electrons. 
B. Each valence shell indicates that the outer shell is filled with electrons. 
C. The atomic radius decreases in moving from left to right along a period, whereas the atomic size increases down the group. 
D. Metallic character increases across a period and decreases down a group. 
Answer: D
Explanation: As the effective nuclear charge acting on the valence shell electrons increases across a period, the tendency to lose electrons will decrease. Down the group, the effective nuclear charge experienced by valence electrons is decreasing because the outermost electrons are farther away from the nucleus. Therefore, these can be lost easily. Hence metallic character decreases across a period and increases down a group. 

#1. Match List-I with List-II and select the correct answer: \n List-I (Plant hormones) LIST-II (Effects) \n A. Auxin 1. Promote cell division \n B. Gibberellins 2. Inhibits growth \n C. Cytokinins 3. It helps the cells to grow longer \n D. Abscisic acid 4. Help in the growth of the stem 
A. 1 2 3 4 
B. 3 4 1 2 
C. 2 3 1 4 
D. 1 3 2 4 
Answer: B
Explanation: Different plant hormones help to coordinate growth, development and responses to the environment. They are synthesised at places away from where they act and simply diffuse to the area of action. 1. When growing plants detect light, a hormone called Auxin, synthesised at the shoot tip, helps the cells to grow longer. 2. Another example of plant hormones are Gibberellins which, help in the growth of the stem. 3. Cytokinins promote cell division, and it is natural that they are present in greater concentration in areas of rapid cell division, such as in fruits and seeds. These are examples of plant hormones that help in promoting growth. 4. But plants also need signals to stop growing. Abscisic acid is one example of a hormone which inhibits growth. Its effects include wilting of leaves. 

#2. Consider the following hormones: \n 1. Adrenaline \n 2. Testosterone \n 3. Estrogen \n Which of the above hormones, prepare the human body for fighting or running instantaneously? 
A. 1 only 
B. 2 only 
C. 2 and 3 only 
D. 1, 2 and 3 
Answer: A
Explanation:  Adrenaline is secreted directly into the blood and carried to different parts of the body and is known as "fight or flight" Hormone.  The target organs or the specific tissues on which it acts include the heart. As a result, the heart beats faster, resulting in supply of more oxygen to our muscles.  The blood to the digestive system and skin is reduced due to contraction of muscles around small arteries in these organs. This diverts the blood to our skeletal muscles.  The breathing rate also increases because of the contractions of the diaphragm and the rib muscles.  All these responses together enable the animal body to be ready to deal with the situations like fighting and running.  Testosterone and estrogen are the reproductive hormones such animal hormones are part of the endocrine system which constitutes a second way of control and coordination in our body. 

#3. Consider the following pairs: \n 1. Thyroid Gland: Thyroxine hormone \n 2. Pituitary Gland: Growth hormone \n 3. Pancreas: Insulin \n Which of the pairs given above is/are correctly matched? 
A. 1 and 2 only 
B. 2 only 
C. 1 and 3 only 
D. 1, 2 and 3 
Answer: D
Explanation: 1. Thyroxine regulates carbohydrate, protein and fat metabolism in the body so as to provide the best balance for growth. Iodine is necessary for the thyroid gland to make thyroxine hormone. In case iodine is deficient in our diet, there is a possibility that we might suffer from goitre. 2. Growth hormone is one of the hormones secreted by the pituitary. As its name indicates, growth hormone regulates growth and development of the body. If there is a deficiency of this hormone in childhood, it leads to dwarfism. 3. Insulin is the hormone which is produced by the pancreas and helps in regulating blood sugar levels. If it is not secreted in proper amounts, the sugar level in the blood rises causing many harmful effects.

#4. Consider the following statements: \n 1. Voluntary actions are controlled by Hindbrain. \n 2. Hearing, Smell is controlled by Fore-Brain. \n Which of the above statements given above is/are Correct? 
A. 1 only 
B. 2 only 
C. Both 1 and 2 
D. Neither 1 nor 2 
Answer: B
Explanation: Medulla in Hind Brain controls involuntary function such as blood pressure, salivation, vomiting. Forebrain is the main thinking part of brain .It has regions which receive sensory impulses from various receptors. Separate areas of the forebrain are specialised for hearing, smell and sight and so on. 

#5. Consider the following pairs: \n 1. Benzoic acid formed in our muscles during a physical activity leads to cramps. \n 2. The rate of breathing in aquatic organisms is much faster than that in terrestrial organisms. \n Which of the statements given above is/are correct? 
A. 1 only 
B. 2 only 
C. Both 1 and 2 
D. Neither 1 nor 2 
Answer: B
Explanation: In anaerobic respiration, there is a lack of oxygen in our muscles. Pyruvate instead of breaking down to form carbondioxide and water, it disintegrates into lactic acid and its accumulation in our muscles leads to cramps. Aquatic organisms breathe much faster than terrestrial organisms as the amount of oxygen dissolved in water is less as compared to the amount of oxygen in air. To compensate this, breathing rate of aquatic organisms is faster. 

#6. Consider the following statements: \n 1. Plants store carbohydrates in the form of glycogen. \n 2. Human beings store carbohydrates in the form of starch. \n Which of the statements given above is/are incorrect? 
A. 1 only 
B. 2 only 
C. Both 1 and 2 
D. Neither 1 nor 2 
Answer: D
Explanation: Carbon and energy requirements of an autotrophic organism are fulfilled by photosynthesis. It is the process by which autotrophs take in substances from the outside and convert them into stored forms of energy. This material is taken in the form of carbon dioxide and water which is then converted into carbohydrates, in the presence of sunlight and chlorophyll. The carbohydrates which are not used immediately are stored in the form of starch, which serves as the internal energy reserve to be used as and when required by the plant. Hence, Statement 1 is not Correct. The energy derived from the food human beings eat, is stored in their body in the form of glycogen.

#7. Consider the following events about Photosynthesis: \n 1. Absorption of light energy by chlorophyll. \n 2. Reduction of carbon dioxide to carbohydrates. \n Which of the statements given above is/are correct? 
A. 1 only 
B. 2 only 
C. Both 1 and 2 
D. Neither 1 nor 2 
Answer: C
Explanation: Photosynthesis is a process by which plants converts inorganic materials such as carbon dioxide and oxygen into carbohydrates using sunlight. This process occurs in the chloroplast of plants. Chlorophyll traps the sunlight to reduce carbon dioxide and water into oxygen and Glucose which allows the plant to grow. Hence, Statement 1 and 2 are Correct. Water splits into oxygen, hydrogen ions, and electrons to replace the lost electrons in light-dependent reaction. As hydrogen ions pass through ATP synthase, ATP is formed. 6CO2 + 6H2O (chlorophyll + sunlight) → C6H12O6 + 6O2 (Glucose) Carbon gets into the leaf through minute pores in leaves called stomata and water goes into the plant through its roots. Oxygen gets out of the cell through stomata.

#8. Consider the following statements about Digestion: \n 1. The secretion is done by the small intestine. \n 2. Hydrochloric acid inhibits the functioning of Pepsin. \n 3. The Mucus protects the inner lining of the stomach from the action of the acid. \n Select the correct answer using the code given below: 
A. 1 only 
B. 2 and 3 only 
C. 3 only 
D. 1, 2 and 3 
Answer: C
Explanation: The digestion functions are taken care of by the gastric glands present in the wall of the stomach. These release Hydrochloric acid, a protein digesting enzyme called pepsin, and mucus. The Hydrochloric acid creates an acidic medium which facilitates the action of the enzyme ‘pepsin’. The mucus protects the inner lining of the stomach from the action of the acid, that causes "acidity" in adults. 

#9. With reference to the respiration in human beings, consider the following statements: \n 1. Cartilage blocks the air passage. \n 2. Haemoglobin has a very high affinity for oxygen. \n 3. Fine hairs in the nostril traps harmful microbes in respiration. \n Which of the statements given above is/are correct? 
A. 1 and 2 only 
B. 2 only 
C. 2 and 3 only
D. 1, 2 and 3 
Answer: C
Explanation: In human beings, air is taken into the body through the nostrils. The air passing through the nostrils is filtered by fine hairs and harmful microbes are trapped. The passage is also lined with mucus which helps in this process. Rings of cartilage are present in throat which ensure that the air passage does not collapse. Once oxygen enters the blood from the lungs, it is taken up by haemoglobin (Hb) in the red blood cells and form oxyhaemoglobin thereby oxygen get circulated in the entire body. 

#10. Regarding blood pressure, consider the following statements: \n 1. The Blood pressure is higher in Veins than in Arteries. \n 2. The normal systolic pressure is about 120 mm of Hg and diastolic pressure is 80 mm of Hg. \n 3. Blood pressure is measured with an instrument called sphygmomanometer. \n Which of the statements given above is/are correct? 
A. 1 and 2 only 
B. 2 and 3 only 
C. 3 only 
D. 1, 2 and 3 
Answer: B
Explanation: The force that blood exerts against the wall of a vessel is called blood pressure. This pressure is much greater in arteries than in veins as arteries carry oxygenated blood from heart to other parts of body that emerge at high pressure and remains in the same. The pressure of blood inside the artery during ventricular systole (contraction) is called systolic pressure and pressure in artery during ventricular diastole (relaxation) is called diastolic pressure. The normal systolic pressure is about 120 mm of Hg and diastolic pressure is 80 mm of Hg. Blood pressure is measured with an instrument called sphygmomanometer. 

#1. Regarding 'reproduction', consider the following statements: \n 1. The basic event in reproduction is the creation of a DNA copy. \n 2. Two copies of DNA in a reproducing cell are completely identical to each other. \n 3. Reproduction is linked to the stability of population of species. \n Which of the statements given above is/are correct? 
A. 1 and 2 only 
B. 1 and 3 only 
C. 3 only 
D. 1, 2 and 3 
Answer: B
Explanation: The DNA in a cell nucleus is an information source for making proteins. If the information is changed, different proteins will be made. Different proteins will eventually lead to altered body designs. Therefore, a basic event in reproduction is the creation of a DNA copy. Cells use chemical reactions to build copies of their DNA. This creates two copies of the DNA in a reproducing cell, that will need to be separated from each other. No bio-chemical reaction is absolutely reliable. Therefore, it is only to be expected that the process of copying the DNA will have some variations each time. As a result, the DNA copies generated will be similar, but may not be identical to the original. A population of organisms fill well-defined places, or niches, in the ecosystem, by using their ability to reproduce. The consistency of DNA copying during reproduction is important for the maintenance of body design features that allows the organism to use that particular niche. Reproduction is therefore linked to the stability of population of species. 

#2. With reference to the male and female gametes, consider the following statements: \n 1. A female gamete is smaller than a male gamete and likely to be motile. \n 2. Male gametes are large and contains the food-stores. \n Which of the statements given above is/are correct? 
A. 1 only 
B. 2 only 
C. Both 1 and 2 
D. Neither 1 nor 2 
Answer: D
Explanation: Conventionally, the motile germ cell is called the male gamete and the germ-cell containing the stored food is called the female gamete. One germ-cell is large and contains the food-stores while the other is smaller and likely to be motile. 

#3. With reference to the reproductive parts of flower, consider the following statements: \n 1. Stamen produces pollen grains. \n 2. The Swollen part of Carpel is known as ovary \n 3. Cross Pollination involves transfer of pollen grains to the same flower \n Which of the statements given above is/are correct? 
A. 1 only 
B. 2 and 3 only 
C. 1 and 3 only 
D. 1 and 2 only 
Answer: D
Explanation: Stamen is the male reproductive part and it produces pollen grains that are yellowish in colour. Carpel is present in the centre of a flower and is the female reproductive part. It is made up of three parts. The swollen bottom part is the ovary, middle elongated part is the style and the terminal part which may be sticky is the stigma. The ovary contains ovules and each ovule has an egg cell. The pollen needs to be transferred from the stamen to the stigma. If this transfer of pollen occurs in the same flower, it is referred to as self-pollination. On the other hand, if the pollen is transferred from one flower to another, it is known as cross pollination. This transfer of pollen from one flower to another is achieved by agents like wind, water or animals.

#4. Regarding the male reproductive system, consider the following statements: \n1. Sperm formation requires a higher temperature than the normal body temperature. \n2. Testosterone triggers Changes at Puberty. \n Which of the statements given above is/are correct? 
A. 1 only 
B. 2 only 
C. Both 1 and 2 
D. Neither 1 nor 2 
Answer: B
Explanation: The formation of germ-cells or sperms takes place in the testes. These are located outside the abdominal cavity in scrotum because sperm formation requires a lower temperature than the normal body temperature. The role of the testes in the secretion of the hormone, testosterone, is to regulate the formation of sperms, and brings about changes in appearance, seen in boys at the time of puberty such as broadening of chest. In adolescent boys, sometimes, the muscles of the growing Voice Box go out of control and the voice becomes hoarse. 

#5. Regarding the female reproductive system, consider the following statements: \n1. Fertilization takes place in the female's Uterus. \n2. The embryo gets nutrition from the mother’s blood with the help of a special tissue called placenta. \n Which of the statements given above is/are incorrect? 
A. 1 only 
B. 2 only 
C. Both 1 and 2 
D. Neither 1 nor 2 
Answer: C
Explanation: The ovary gets produced in the ovaries in the female. The fertilization of gametes takes place in the fallopian tube. The fertilised egg, the zygote, gets implanted in the lining of the uterus, and starts dividing. The embryo gets nutrition from the mother’s blood with the help of a special tissue called placenta. This is a disc which is embedded in the uterine wall. It contains villi on the embryo’s side of the tissue. On the mother’s side are blood spaces, which surround the villi. 

#6. Consider the following: \n 1. Menstruation takes place when egg is not fertilized. \n 2. HIV gets transmit by eating with an infected person. \n Which of the statements given above is/are incorrect? 
A. 1 only 
B. 2 only 
C. Both 1 and 2 
D. Neither 1 nor 2 
Answer: A
Explanation: Menstruation is a periodic cycle that recurs every month and lasts for about two to eight days. When egg does not gets fertilized, the thin lining of uterus which nourishes the embryo in event of fertilization, breakdowns and comes out of vagina as blood and mucous HIV is a Sexually Transmitted Disease which by mere contact (activities such as playing or eating) with infected person does not get transferred. 

#7. Consider the following statements: \n 1. The number of successful variations in second generation is maximised by sexual reproduction than by asexual mode of reproduction. \n 2. Dominant and Recessive traits get expressed in sexual reproduction. \n Which of the statements given above is/are correct? 
A. 1 only 
B. 2 only 
C. Both 1 and 2 
D. Neither 1 nor 2 
Answer: C
Explanation: In asexual reproduction, if one bacterium divides, and then the resultant two bacteria divide again, the four individual bacteria generated would be very similar. There would be only very minor differences between them, generated due to small inaccuracies in DNA copying. However, if sexual reproduction is involved, greater diversity will be generated, and the second generation will have differences that they inherit from the first generation, as well as newly created differences. Variations generated during birth can be hereditary. Due to these differences, survival of the organism can increase. Two copies of the trait are inherited in each sexually reproducing organism. These two may be identical, or may be different, depending on the parentage. For example, both TT and Tt are tall plants, while only tt is a short plant. In other words, a single copy of ‘T’ is enough to make the plant tall, while both copies have to be ‘t’ for the plant to be short. Traits like ‘T’ are called dominant traits, while those that behave like ‘t’ are called recessive traits. The trait that gets expressed in phenotype is called the dominant trait and the other which fails to express is called the recessive trait. 

#8. Regarding sex determination, consider the following statements: \n 1. Sex is the determined by temperature in some species. \n 2. Sex is not genetically determined in snails. \n 3. The sex of human children is determined by which chromosome they inherit from their father. Which of the statements given above is/are correct? 
A. 1 and 2 only 
B. 2 only 
C. 1 and 3 only 
D. 1, 2 and 3 
Answer: D
Explanation: Some species rely entirely on environmental cues. Thus, in some animals, the temperature at which fertilised eggs are kept determines whether the animals developing in the eggs will be male or female. In other animals, such as snails, individuals can change sex, indicating that sex is not genetically determined. . In human beings, the sex of the individual is largely genetically determined. Most human chromosomes have a maternal and a paternal copy, and we have 22 such pairs. But one pair, called the sex chromosomes, is different in men and women. So women are XX, while men are XY. Thus, the sex of the children will be determined by what they inherit from their father. A child who inherits an X chromosome from her father will be a girl, and one who inherits a Y chromosome from him will be a boy. 

#9. Which of the following places, witnesses the emergence of a modern human species? 
A. Europe 
B. Africa 
C. India 
D. North America 
Answer: B
Explanation: The earliest members of human species, Homo sapiens, arose in Africa, moved across continents and developed into distinct races. 

#10. Speciation is a process that leads to: 
A. formation of new species from older ones 
B. formation of clones 
C. formation of taller organisms 
D. formation of dwarf organisms 
Answer: A
Explanation: Speciation is defined as the formation of new species from an existing species, either by evolution or by genetic modification. A species is a group of organisms with similar characteristics and can interbreed to give a fertile offspring. Speciation is an evolutionary process of the formation of new and distinct species. The new species get reproductively isolated from the previous species i.e., the new species can’t reproduce with the old species. 


#11. With reference to the fossils, consider the following statements: \n 1. Preserved traces of living organisms, in form of their impressions on mud are called fossils. \n 2. Fossil dating is done by detecting the ratios of different isotopes of the same element in the fossil material. \n Which of the statements given above is/are correct? 
A. 1 only 
B. 2 only 
C. Both 1 and 2 
D. Neither 1 nor 2 
Answer: C
Explanation: If a dead insect gets caught in hot mud, for example, it will not decompose quickly, and the mud will eventually harden and retain the impression of the body parts of the insect. All such preserved traces of living organisms are called fossils. There are two methods of fossil dating. One is relative. If we dig into the earth and start finding fossils, it is reasonable to suppose that the fossils we find closer to the surface are more recent than the fossils we find in deeper layers. The second way of dating fossils is by detecting the ratios of different isotopes of the same element in the fossil material. 


#1. If an opaque object on the path of light becomes very small, light has a tendency to bend around it and not walk in a straight line. Which of the following effects given below, defines the statement? 
A. Diffraction of light 
B. Refraction of light 
C. Light interference 
D. Reflection of light 
Answer: A
Explanation: If an opaque object on the path of light becomes very small, light has a tendency to bend around it and not walk in a straight line - this effect is known as the diffraction of light. 


#2. Regarding the properties of the image formed by a plane mirror, consider the following statements: \n 1. Virtual and Erect images are formed by Plane Mirror. \n 2. The size of the image is smaller to that of the object. \n Which of the statements given above is/are correct? 
A. 1 only 
B. 2 only 
C. Both 1 and 2 
D. Neither 1 nor 2 
Answer: A
Explanation: The properties of the image formed by a plane mirror are: 1. Image formed by a plane mirror is always virtual and erect. 2. The size of the image is equal to that of the object. 3. The image formed is as far behind the mirror as the object is in front of it. 4. The image formed is laterally inverted. 


#3. Regarding the lens, consider the following statements: \n 1. A transparent material bound by two surfaces, of which one or both surfaces are spherical, forms a lens. \n 2. A lens, either a convex lens or a concave lens, has two spherical surfaces. \n Which of the statements given above is/are correct? 
A. 1 only 
B. 2 only 
C. Both 1 and 2 
D. Neither 1 nor 2 
Answer: C
Explanation: A transparent material bound by two surfaces, of which one or both surfaces are spherical, forms a lens. This means that a lens is bound by at least one spherical surface. In such lenses, the other surface would be plane. A lens may have two spherical surfaces, bulging outwards. Such a lens is called a double convex lens. It is simply called a convex lens. It is thicker at the middle as compared to the edges. Similarly, a double concave lens is bounded by two spherical surfaces, curved inwards. It is thicker at the edges than at the middle. A double concave lens is simply called a concave lens.A lens, either a convex lens or a concave lens, has two spherical surfaces. Each of these surfaces forms a part of a sphere. 


#4. Which of the following use concave mirror? \n 1. Torches \n 2. Vehicles headlights \n 3. Dentists to see larger images of the patient's teeth. \n 4. Shaving mirror \n Select the correct answer using the code given below: 
A. 1 and 2 only 
B. 2, 3 and 4 only 
C. 1, 3 and 4 only 
D. All of the above 
Answer: D
Explanation: Concave mirrors are commonly used in torches, searchlights and vehicles headlights to get powerful parallel beams of light as these mirrors possess the ability to focus parallel rays of light to a point, produce a large size image.They are often used as shaving mirrors to see a larger image of the face The dentists use concave mirrors to see large images of the teeth of patients.Large concave mirrors are used to concentrate sunlight to produce heat in solar furnaces. 


#5. Light travels the fastest in which of the following mediums? 
A. In air 
B. In vacuum 
C. In water 
D. In diamond 
Answer: B
Explanation: Light propagates with different speeds in different media. Light travels the fastest in vacuum with the highest speed of 3×108 m s-1. Vacuum is free space and there are no obstacle to slow down light as the refractive index of vacuum is very low. In air, the speed of light is only marginally less, compared to that in vacuum. It reduces considerably in glass or water. 

#6. Regarding the human eye, consider the following statements: \n 1. Light enters the eye through the cornea. \n 2. The pupil controls the amount of light entering the eye. \n 3. An inverted real image of the object is formed on the retina.\n  Which of the statements given above is/are correct? 
A. 1 only 
B. 2 and 3 only 
C. 1 and 3 only 
D. 1, 2 and 3 
Answer: D
Explanation: Light enters the eye, through a thin membrane called cornea. Cornea forms a transparent bulge on the front surface of the eyeball. .Pupil is a pigmented layer of tissues that makes up the colored portion of the eye. Its primary function is to control the amount of light entering in the eye. The eye lens forms an inverted real image of the object on the retina. The retina is a delicate membrane having enormous number of light-sensitive cells. 


#7. Regarding the myopia, consider the following statements: \n 1. In a myopic eye, the image of a distant object gets formed behind the retina. \n 2. Myopia is caused by shortening of eyeball. \n Which of the statements given above is/are correct? 
A. 1 only 
B. 2 only 
C. Both 1 and 2 
D. Neither 1 nor 2 
Answer: D
Explanation: Myopia is also known as near sightedness. A person with myopia can see nearby objects clearly but cannot see distant objects distinctly. Myopia is caused by two reasons: 1. Excessive curvature of eye lens. 2. Elongation of eyeball. A person with this defect has the far point nearer than infinity. Such a person may see clearly upto a distance of a few metres.In a myopic eye, the image of a distant object is formed in front of the retina and not at the retina itself. 



#8. Regarding the hypermetropia, consider the following statements: \n 1. A person with hypermetropia can see distant objects clearly. \n 2. Image is formed behind the retina. \n 3. Convex lens is used to correct this defect. \n Which of the statements given above is/are correct? 
A. 1 only
B. 2 and 3 only
C. 3 only
D. All of the above
Answer: D
Explanation: Hypermetropia is also known as farsightedness. A person with hypermetropia can see distant objects clearly but cannot see nearby objects distinctly. The near point, for the person, is farther away from the normal near point (25 cm). Such a person has to keep a reading material much beyond 25 cm from the eye for comfortable reading. This is because the light rays from a closeby object are focussed at a point behind the retina. This defect can be corrected by using a convex lens of appropriate power. Eye-glasses with converging lenses provide the additional focussing power required for forming the image on the retina. 


#9. Which of the following parts of eyes is transplanted during eye donation? 
A. Retina 
B. Ciliary muscles 
C. Pupil 
D. Cornea 
Answer: D
Explanation: Corneal blindness can be cured through corneal transplantation of donated eyes. Eye donors can belong to any age group or sex. People who use spectacles, or those operated for cataract, can still donate the eyes. People who are diabetic, have hypertension, asthma patients and those without communicable diseases can also donate eyes. 




#10. Which of the following is the correct cause of the twinkling of stars? 
A. Scattering of light 
B. Refraction of light 
C. Total internal reflection of light 
D. Diffraction of light 
Answer: B
Explanation: The twinkling of a star is due to atmospheric refraction. The starlight, on entering the earth’s atmosphere, undergoes refraction continuously, before it reaches the earth. The atmospheric refraction occurs in a medium of gradually changing refractive index. Since the atmosphere bends starlight towards the normal, the apparent position of the star is slightly different from its actual position. The star appears slightly higher (above) than its actual position when viewed near the horizon. Further, this apparent position of the star is not stationary, but keeps on changing slightly. Since the stars are very distant, they approximate point-sized sources of light. As the path of rays of light coming from the star goes on varying slightly, the apparent position of the star fluctuates and the amount of starlight entering the eye flickers - the star sometimes appears brighter, and at some other time, fainter, which gives the twinkling effect. 



#11. Which of the following phenomena are caused by the scattering of light?  \n 1. The colour of clear sky is blue.  \n 2. Twinkling of the stars.  \n 3. The colour of the Sun at Sunrise and Sunset is red. 
A. 1 and 2 only 
B. 1, 2 and 3 
C. 2 and 3 only 
D. 1 and 3 only 
Answer: D
Explanation: The sky appears blue during a clear cloudless day because the molecules in the air scatter blue light from the sun more than they scatter red light. Twinkling of stars is caused by the atmospheric refraction of light. Light from the sun near the horizon passes through thicker layers of air and travel larger distance in the earth’s atmosphere before reaching our eyes. Near the horizon, most of the blue light and other lights with shorter wavelengths are scattered away by the particles. Therefore, the light that reaches our eyes is of longer wavelength. This gives rise to the reddish appearance of the Sun.  


#1. Consider the following pairs: \n 1. SI unit of electric charge - Ampere (A) \n 2. SI unit of electric current - Coulomb (C) \n 3. Instrument for measuring electric current in a circuit - Ammeter \n Which of the pairs given above is/are correctly matched? 
A. 1 only 
B. 2 and 3 only 
C. 1 and 3 only 
D. 3 only 
Answer: D
Explanation: The SI unit of electric charge is coulomb (C), which is equivalent to the charge contained in nearly 6 × 1018 electrons. The electric current is expressed by a unit called ampere (A). One ampere is constituted by the flow of one coulomb of charge in a second, that is, 1 A = 1 C/1 s. An instrument called ammeter measures electric current in a circuit. To measure the current in a circuit, it is connected in series in a circuit. 


#2. Which of the following laws states that- the electric current flowing through a metallic wire is directly proportional to the  potential difference V, across its ends, provided its temperature remains the same? 
A. Faraday's law 
B. Charles’s law 
C. Ohm’s law 
D. Fleming's law 
Answer: C
Explanation: In 1827, a German physicist Georg Simon Ohm (1787- 1854) found out the relationship between the current I, flowing in a metallic wire and the potential difference across its terminals. He stated that the electric current flowing through a metallic wire is directly proportional to the potential difference V, across its ends provided its temperature remains the same. This is called Ohm’s law. 



#3. Resistance of a conductor depends upon which of the following factors: \n 1. On its length \n 2. On its area of cross-section \n 3. On the nature of its material \n Which of the statements given above is/are correct? 
A. 1 only  
B. 1 and 3 only 
C. 2 and 3 only 
D. 1, 2 and 3 
Answer: D
Explanation: On applying Ohm’s law, we observe that the resistance of a conductor depends (i) on its length, (ii) on its area of crosssection, and (iii) on the nature of its material. Precise measurements have shown that resistance of a uniform metallic conductor is directly proportional to its length (l) and inversely proportional to the area of cross-section (A). 



#4. Which of the following inorganic gases are filled in bulbs? \n 1. Nitrogen \n 2. Helium \n 3. Argon \n 4. Hydrogen 
A. 1 and 3 only 
B. 1 and 4 only 
C. 3 and 4 only 
D. 2 and 3 only  
Answer: A
Explanation: The bulbs are usually filled with chemically inactive gas - nitrogen/ argon, to prolong the life of filament. The closed glass chamber of a bulb contains an inactive gas Argon or Nitrogen. The glass chamber cannot be filled with air as the presence of oxygen will cause the filament to burn and a vacuum will evaporate the filament. Inert gases like argon do not react and will maintain a particular pressure preventing filament from burning. 

#5. Which one of the following describes the direction of an electric current? 
A. Opposite to the direction of flow of electrons. 
B. In the direction of flow of electrons. 
C. In the direction of flow of protons. 
D. Opposite to the direction of flow of protons. 
Answer: A
Explanation: A stream of electrons moving through a conductor constitutes an electric current. Conventionally, the direction of current is taken opposite to the direction of flow of electrons. 

#6. Which of the following scientists had first observed the magnetic effect of electric current?  
A. Henry 
B. Oersted 
C. Faraday 
D. Volt 
Answer: B
Explanation: Hans Christian Oersted, one of the leading scientists of the 19th century, played a crucial role in understanding electromagnetism. In 1820 he accidentally discovered that a compass needle got deflected when an electric current passed through a metallic wire placed nearby. Through this observation Oersted showed that electricity and magnetism are related phenomena. His research later created technologies such as radio, television and fiber optics. The unit of magnetic field strength has been named as Oersted in his honor. 

#7. If the current through the wire increases, the magnitude of the magnetic field produced at a given point: 
A. Increases 
B. Decreases 
C. Firstly increases and then decreases 
D. Can’t comment anything 
Answer: A 
37, SOLUTIONS
Explanation: The current flowing through a conductor is directly proportional to magnetic field. As the electric current through the wire increases, the magnitude of the magnetic field produced at a given point increases. 

#8. Consider the following statements: \n 1. The Direct Current does not change its direction with time \n 2. A current, which changes direction after equal intervals of time, is called an alternating current. \n 3. Alternating current is used in transmissions over long distances. \n Which of the statements given above is/are correct? 
A. 1 only 
B. 1 and 2 only 
C. 2 and 3 only 
D. All of the above 
Answer: D
Explanation: A current, which changes direction after equal intervals of time, is called an alternating current while direct current (DC) does not change its direction with time. The difference between the direct and alternating currents is that the direct current always flows in one direction, whereas the  alternating current reverses its direction periodically. Most power stations constructed these days produce alternating current (AC). In India, the AC changes direction after every 1/100 second, that is, the frequency of AC is 50 Hz. An important advantage of AC over DC is that electric power can be transmitted over long distances without much loss of energy 

#9. Which of the following statements is the phenomenon of electromagnetic induction? 
A. The process of charging a body 
B. The process of generating magnetic field due to a current passing through a coil 
C. Producing induced current in a coil due to relative motion between a magnet and the coil 
D. The process of rotating a coil of an electric motor. 
Answer: B
Explanation: The phenomenon of electromagnetic induction is the production of induced current in a coil placed in a region where the magnetic field changes with time. The magnetic field may change due to a relative motion between the coil and a magnet placed near to the coil. If the coil is placed near to a current-carrying conductor, the magnetic field may change either due to a change in the current through the conductor or due to the relative motion between the coil and conductor.  


#10. Which of the following organs in the human body produce the magnetic field? \n 1. Brain \n 2. Kidney \n 3. Heart \n 4. Liver 
A. 1 and 2 only 
B. 2, 3 and 4 only 
C. 1 and 3 only 
D. All of the above 
Answer: C
Explanation: Two main organs in the human body where the magnetic field gets produced are the heart and the brain. The magnetic field inside the body forms the basis of obtaining the images of different body parts. This is done using a technique called Magnetic Resonance Imaging (MRI). Analysis of these images helps in medical diagnosis. Magnetism has, thus also got an important application in medicine   


#1. Different sources are used in India to meet the energy requirement. Depending on the consumption of energy from different sources, select the correct answer from the code given below in descending order: 
A. Coal > Water > Petroleum and Natural Gas > Nuclear energy 
B. Coal > Petroleum and Natural Gas > Water > Nuclear energy 
C. Coal > Petroleum and Natural Gas > Nuclear energy > Water 
D. Coal > Nuclear energy > Water > Petroleum and Natural Gas 
Answer: A
Explanation: The following sources of energy are used to meet the energy requirements in India, in descending order: Coal > Water > Petroleum and Natural Gas > Nuclear energy. 


#2. Consider the following statements: \n 1. Hydro power plants convert the potential energy of falling water into electricity. \n 2. Construction of big dams generates greenhouse gas. \n Which of the statements given above is/are correct? 
A. 1 only 
B. 2 only  
C. Both 1 and 2 
D. Neither 1 nor 2 
Answer: C
Explanation: Hydro power plants convert the potential energy of falling water into electricity. In order to produce hydel electricity, high-rise dams are constructed on the river to obstruct the flow of water and thereby collect water in larger reservoirs. The water level rises and in this process the kinetic energy of flowing water gets transformed into potential energy. But, constructions of big dams have certain problems associated with it. Large ecosystems get destroyed when submerged under the water in dams. The vegetation which is submerged, rots under anaerobic conditions and gives rise to large amounts of methane which is a greenhouse gas. 


#3. Consider the following statements: \n 1. The fuels which are produced by plants and animals are called biomass. \n 2. The main Constituent of Biogas is Methane. \n Which of the statements given above is/are correct? 
A. 1 only  
B. 2 only 
C. Both 1 and 2 
D. Neither 1 nor 2 
Answer: C 
Explanation Cow dung cakes serve as a steady source of fuel. Since, these fuels are derived from plants 
and animalence they constitute - Biomass. These fuels, however, do not produce much heat on burning but a lot of smoke is given out when they are burnt. Cow-dung, various plant materials like crops residue, vegetable waste and sewage are decomposed in the absence of oxygen to give Biogas. Since the starting material is mainly cow-dung, it is popularly known as ‘gobar-gas’. Bio-gas is produced in a dome-shaped plant. Bio-gas is an excellent fuel as it contains up to 75% methane. It burns without smoke, leaves no residue like ash in wood, charcoal and coal burning. Its heating capacity is high. 


#4. The slurry left behind in biogas plant, contains which of the following elements? 
A. Potash and phosphorus 
B. Nitrogen and phosphorus 
C. Potash only 
D. Nitrogen only  
Answer: B 
Explanation The slurry left behind in bio-gas plant is removed periodically and used as excellent manure, rich nitrogen and phosphorous. 


#5. Which of the following countries is called the country of 'winds'? 
A. Norway 
B. Canada 
C. Denmark 
D. Japan 
Answer: C
Explanation: Denmark is called the country of ‘winds’. More than 25% of their electricity needs are generated through a vast network of windmills. In terms of total output, Germany is the leader, while India is ranked fifth in harnessing wind energy for the production of electricity. 


#6. Which of the following elements is/are used in making solar cells?  \n 1. Silicon  \n 2. Astatine   \n 3. Sirium  \n 4. Vanadium 
A. 1 only 
B. 3 and 4 only 
C. 1, 2 and 3 only 
D. All of the above 
Answer: A
Explanation: Solar cells convert solar energy into electricity. Silicon is used for making solar cells, which is abundant in nature, but availability of the special grade silicon for making solar cells is limited. The entire process of manufacture is still very expensive, silver used for interconnection of the cells in the panel further adds to the cost. The principal advantages associated with solar cells are that they have no moving parts, require little maintenance and work quite satisfactorily without the use of any focusing device. Another advantage is that they can be set up in remote and inaccessible hamlets or very sparsely inhabited areas in which laying of a power transmission line may be expensive and not commercially viable.  


#7. Sea energy can be converted into electricity in oceanthermal-energy-conversion plants when 
A. The temperature difference between the water at the surface and water at depths up to 2 km is 293 K (20°C) or more. 
B. There are narrow valleys on the coast where the dams can be built. 
C. There are shallow and wide coastal shores, where dams are to be built by the river. 
D. All of the above. 
Answer: A
Explanation: The water at the surface of the sea or ocean is heated by the Sun while the water in deeper sections is relatively cold. This difference in temperature is exploited to obtain energy in ocean-thermal-energy conversion plants. These plants can operate if the temperature difference between the water at the surface and water at depths up to 2 km is 293 K (20°C) or more. The warm surface-water is used to boil a volatile liquid like ammonia. The vapours of the liquid are then used to run the turbine of generator. The cold water from the depth of the ocean is pumped up and condense vapour again to liquid.  


#8. Consider the following statements: \n 1. In the nuclear fission, the nucleus of a heavy atom can be split apart into lighter nuclei. \n 2. In a nuclear reactor designed for electric power generation, nuclear ‘fuel’ releases energy at a controlled rate. \n Which of the statements given above is/are correct? 
A. 1 only 
B. 2 only 
C. Both 1 and 2 
D. Neither 1 nor 2 
Answer: C
Explanation: In Nuclear fission, the nucleus of a heavy atom (such as uranium, plutonium or thorium), when bombarded with low-energy neutrons, can be split apart into lighter nuclei. When this is done, a tremendous amount of energy is released if the mass of the original nucleus is just a little more than the sum of the masses of the individual products. The fission of an atom of uranium, for example, produces 10 million times the energy produced by the combustion of an atom of carbon from coal. In a nuclear reactor designed for electric power generation, such nuclear 'fuel' can be part of a self-sustaining fission chain reaction  that releases energy at a controlled rate. The released energy can be used to produce steam and further generate electricity. 


#9. Match List-I with List-II and select the correct answer using the code given below: \n List-I (Nuclear power reactors) List-II (States)  \n A. Kalpakkam    1. Uttar Pradesh  \n B. Narora       2. Gujarat  \n C. Kakrapar     3. Tamil Nadu  \n D. Tarapur      4. Maharashtra 
A. A-3 B-1 C-2 D-4 
B. A-3 B-1 C-4 D-2 
C. A-3 B-4 C-2 D-1 
D. A-3 B-4 C-1 D-2 
Answer: A
Explanation: Nuclear power reactors located at Tarapur (Maharashtra), Rana Pratap Sagar (Rajasthan), Kalpakkam (Tamil Nadu), Narora (UP), Kakrapar (Gujarat) and Kaiga (Karnataka) have the installed capacity of less than 3% of the total electricity generation capacity of our country.  


#10. With reference to nuclear fusion, consider the following statements: \n 1. During the nuclear fusion reaction two lighter nuclei are joined to make a heavier nucleus. \n 2. Nuclear fusion reaction takes place in Sun and Star. It takes considerable energy to force the nuclei to fuse. \n Which of the statements given above is/are correct? 
A. 1 only 
B. 1 and 2 only 
C. 1, 2 and 3 
D. 2 and 3 only 
Answer: C
Explanation: Fusion means joining lighter nuclei to make a heavier nucleus, most commonly hydrogen isotopesare added together to create Helium. It releases a tremendous amount of energy, according to the Einstein equation, as the mass of the product is little less than the sum of the masses of the original individual nuclei. Such nuclear fusion reactions are the source of energy in the Sun and other stars. It takes considerable energy to force the nuclei to  fuse. The conditions needed for this process are extreme - millions of degrees of temperature and millions of pascals of pressure.  



#1. Which of the following statements best describes the term ecosystem? 
A. A community of people interacting with each other (organism). 
B. The part of the Earth that is inhabited by living organisms. 
C. The interacting organisms in an area together with the non-living constituents of the environment. 
D. Flora and fauna of a geographical area. 
Answer: C
Explanation: All the Biotic Components in an area together with the abiotic constituents of the environment form an ecosystem. Biotic components comprising living organisms and abiotic components comprising physical factors like temperature, rainfall, wind, soil and minerals2 



#2. Which of the following ecosystems is/are examples of natural ecosystem? \n 1. Ponds \n 2. Lakes \n 3. Crop-fields \n 4. Gardens Select the correct answer using the code given below:  
A. 1, 2 and 4 only 
B. 2, 3 and 4 only 
C. 1 and 2 only 
D. 2 only 
Answer: C
Explanation: Forests, ponds and lakes are natural ecosystems while gardens and crop-fields are human made (artificial) ecosystems. 



#3. Regarding the producers, consider the following statements: \n 1. They make organic compounds from inorganic substances in the presence of sunlight and chlorophyll. \n 2. Fungi are Heterotrophs. \n Which of the statements given below is/are correct? 
A. 1 only 
B. 2 only 
C. Both 1 and 2 
D. Neither 1 nor 2 
Answer: C 
Explanation: Organisms which can make organic compounds like sugar and starch from inorganic substances using the radiant energy of the Sun in the presence of chlorophyll are called the producers or autotrophs. All green plants and certain blue green algae which can produce food by photosynthesis come under this category. All fungi are heterotrophs, they use enzymes to break down the materials on which they are growing, fungi are largely responsible for organic decomposition, they are also capable of fermentation. 



#4. Regarding the categories of consumers in ecosystem, which of the following types of organisms are called decomposers? \n 1. Virus \n 2. Bacteria \n 3. Fungus 
A. 1 only 
B. 2 and 3 only 
C. 3 only 
D. All of the above 
Answer: B
Explanation: The microorganisms, comprising Bacteria and Fungi, break-down the dead remains and waste products of organisms. These microorganisms are the decomposers as they break-down the complex organic substances into simple inorganic substances that go into the soil and are used up once more by the plants. Bacteria unlike viruses have their own enzymes and all the molecules to survive on their own as long as food is available. The Bacteria feed on the animal or dead animal and grow and therefore, decompose the bodies. Contrary to this, viruses are nonliving when outside the host. For them to replicate and produce more virus particles, they take the help of machinery of other living beings. When other animals are dead, they cannot support virus particles to be able to sustain their life. If the animal is dead and if it contains virus they will also die or remain there till they find a living and suitable host. 



#5. Regarding a food chain in an ecosystem, consider the following statements: \n 1. Food chain involves flow of energy from one component of the system to another. \n 2. The Food chain demonstrates the numbers of every organism that are eaten by others in line. \n Which of the statements given above is/are correct? 
A. 1 only 
B. 2 only  
C. Both 1 and 2 
D. Neither 1 nor 2 
Answer: A
Explanation: Food chain is the series of organisms feeding on one another. The food we eat acts as a fuel to provide us energy to do work. Thus the interactions among various components of the environment involves flow of energy from one component of the system to another. The length and complexity of food chains vary greatly. Each organism is generally eaten by two or more other kinds of organisms which in turn are eaten by several other organisms. So instead of a straight line food chain, the relationship can be shown as a series of branching lines called a food web. 



#6. Consider the following statements: \n 1. Producers convert solar energy into chemical energy. \n 2. The flow of energy is unidirectional in food chain. \n 3. The green plants in a terrestrial ecosystem capture about 100% of the energy of sunlight that falls on their leaves and convert it into food energy. \n Which of the statements given below is/are correct? 
A. 1 only 
B. 2 and 3 only  
C. 1 and 2 only 
D. All of the above 
Answer: C
Explanation: The autotrophs capture the energy present in sunlight and convert it into chemical energy which remains stored in the form of Carbohydrates. The flow of energy is unidirectional. The energy that is captured by the autotrophs does not revert back to the solar input and the energy which passes to the herbivores does not come back to autotrophs. As it moves progressively through the various trophic levels it is no longer available to the previous level. The green plants in a terrestrial ecosystem capture about 1% of the energy of sunlight that falls on their leaves, rest gets dissipated. 



#7. What is the average value for the amount of organic matter that is present at each step and reaches the next level of consumers? 
A. 10% 
B. 20% 
C. 30% 
D. 50% 
Answer: A
Explanation: When green plants are eaten by primary consumers, a great deal of energy is lost as heat to the environment, some amount goes into digestion and in doing work and the rest goes towards growth and reproduction. An average of 10% of the food eaten is turned into its own body and made available for the next level of consumers.Therefore, 10% can be taken as the average value for the amount of organic matter that is present at each step and reaches the next level of consumers. 



#8. Which of the following statements defines Biological Magnification? 
A. Attack of an exotic species of plants in any ecosystem. 
B. The energy flow from one level to another level in the food chain. 
C. Concentration of harmful biochemicals from one level to another level in the food chain. 
D. The lack in elements for increasing the productivity of any ecosystem. 
Answer: C
Explanation: The chemicals which are used to protect the crops from pests are either washed down into the soil or into the water bodies. From the soil, these are absorbed by the plants along with water and minerals, and from the water bodies these are taken up by aquatic plants and animals. Thus they enter the food chain.  As these chemicals are not degradable, these get accumulated progressively at each trophic level. As Human beings occupy the top level in any food chain, the maximum concentration of these chemicals get accumulated in our bodies. This phenomenon is known as Biological Magnification. 



#9. Regarding the ozone layer, consider the following statements: \n 1. Ozone (O3) is a molecule formed by three atoms of oxygen. \n 2. Ozone shields the surface of the earth from ultraviolet radiation from the Sun. \n 3. Human synthesized chemicals such as chlorofluorocarbons (CFCs) have started declining the amount of ozone. Which of the statements given above is/are correct? 
A. 1 only 
B. 2 and 3 only 
C. 1, 2 and 3 
D. 1 and 2 only 
Answer: C
Explanation: Ozone (O3) is a molecule formed by three atoms of oxygen. Ozone, is a deadly poison. However, at the higher levels of the atmosphere, ozone performs an essential function. It shields the surface of the earth from ultraviolet (UV) radiation from the Sun.  This radiation is highly damaging to organisms, for example, it is known to cause skin cancer in human beings. The amount of ozone in the atmosphere began to drop sharply in the 1980s. This decrease has been linked to synthetic chemicals like chlorofluorocarbons (CFCs) which are used as refrigerants and in fire extinguishers. In 1987, the United Nations Environment Programme (UNEP) succeeded in forging an agreement to freeze CFC production at 1986 levels. 


#10. Hydrogen bomb is based on which of the following reactions? 
A. Controlled fusion reaction 
B. Thermonuclear fusion reaction 
C. Controlled fission reaction 
D. Thermonuclear fission reaction 
Answer: B
Explanation: Hydrogen bomb is based on the 'nuclear fusion' technology. In order to release more energy, atoms have been fused together in the making of H-Bomb.Hydrogen Bombs use fusion, the same way that powers the Sun or any other star. Isotopes of hydrogen are forced together to release a much bigger blast — hundred times powerful than the nuclear weapon that have been used in warfare.  


#1. The Ganga Action Plan was first introduced in which of the following years? 
A. 1982 
B. 1985 
C. 1988 
D. 2000 
Answer: B
Explanation: The Ganga Action Plan was a multi-crore project which came about in 1985 because the quality of the water in the Ganga was very poor. The Ganga runs its course of over 2500 km from Gangotri in the Himalayas to Ganga Sagar in the Bay of Bengal. It is being turned into a drain by more than a hundred towns and cities in Uttar Pradesh, Bihar and West Bengal that pour their garbage and excreta into it. Largely untreated sewage is dumped into the Ganges every day. In addition, pollution is caused by other human activities like bathing, washing of clothes and immersion of ashes or unburnt corpses. And then, industries contribute chemical effluents to the Ganga pollution load and the toxicity kills fish in large sections of the river. 


#2. Amrita Devi Bishnoi National Award is given in which of the following fields? 
A. In the field of wildlife conservation 
B. In the soil conservation 
C. In the field of women empowerment 
D. In science 
Answer: B
Explanation: The Government of India instituted an ‘Amrita Devi Bishnoi National Award for Wildlife Conservation’ in the memory of Amrita Devi Bishnoi, who in 1731 sacrificed her life along with 363 others for the protection of ‘khejri’ trees in Khejarli village near Jodhpur in Rajasthan. 


#3. Consider the following statements: \n 1. The Chipko movement is associated with forest conservation. \n 2. The Chipko movement, got originated from Himachal Pradesh. \n 3. The Chipko movement began during the early 1970s. Which of the statements given above is/are correct? 
A. 1 and 2 only 
B. 1 and 3 only 
C. 2 and 3 only 
D. All of the above 
Answer: B
Explanation: Sunderlal Bahuguna, a noted environmentalist initiated - The Chipko Andolan (Hug the Trees Movement). It was the result of a grassroot level effort to end the alienation of people from their forests. The movement originated from an incident in a remote village called Reni in Garhwal, Uttarakhand. The movement began high-up in the Himalayas during the early 1970s. 


#4. Match List-I with List-II: 
List-I (Methodology of Water Harvesting) List-II (Related State) 
A. Pynes 1. Tamil Nadu 
B. Kulhs 2. Bihar 
C. Khadins 3. Himachal Pradesh 
D. Eris 4. Rajasthan 
A. A-2 B-3 C-4 D-1 
B. A-2 B-3 C-1 D-4 
C. A-1 B-2 C-3 D-4 
D. A-4 B-2 C-1 D-3 
Answer: A
Explanation: Water harvesting is an age-old concept in India. Khadins, tanks and nadis in Rajasthan, Bandharas and Tals in Maharashtra, Bundhis in Madhya Pradesh and Uttar Pradesh, Ahars and Pynes in Bihar, Kulhs in Himachal Pradesh, Ponds in the Kandi Belt of Jammu region, Eris (tanks) in Tamil Nadu, Surangams in Kerala, Kattas in Karnataka are some of the ancient water harvesting, including water conveyance, structures still in use today. 


#5. Consider the following statements: \n 1. Coal and Petroleum have been derived from inorganic sources. \n 2. Carbon monoxide is the byproduct of limited supply of oxygen during combustion. \n Which of the statements given above is/are correct? 
A. 1 only 
B. 2 only 
C. Both 1 and 2 
D. Neither 1 nor 2 
Answer: B
Explanation: Coal and Petroleum have been formed from biomass, in addition to carbon, these contain hydrogen, nitrogen and sulphur. When these are burnt, the products are carbon dioxide, water, oxides of nitrogen and oxides of sulphur.When combustion takes place in insufficient air (oxygen), carbon monoxide is formed instead of carbon dioxide. 


#6. Which of the following does not lead to the depletion of groundwater? 
A. Establishing thermal power plants 
B. Cultivation of high yielding varieties of crops 
C. Process of deforestation 
D. Process of afforestation 
Answer: D 


#7. Among the following choose the correct option which includes acts related to the three R's strategy which can bseful for conserving our natural resources? 
A. Recycle, regenerate, reuse 
B. Reduce, regenerate, reuse 
C. Reduce, reuse, redistribute 
D. Reduce, recycle, reuse 
Answer: D 
Explanation D

#8. In our country, there are attempts to increase the height of several existing dams like Tehri and 
Almati dam across the Narmada. \n Choose the correct statements among the following that are a consequence of raising the height of dams \n 1. Terrestrial flora and fauna of the area is destroyed completely \n 2. Dislocation of people and domestic animals living in the area \n 3. Valuable agricultural land may be permanently lost \n 4. It will generate permanent employment for people Choose the correct option from the following: 
A. 1 and 2 
B. 1, 2 and 3 
C. 2 and 4 
D. 1, 3 and 4 
Answer: B
Explanation: A large area of land is covered in building the dams which causes the devastation of the terrestrial flora and fauna. Masses of people have to be displaced to other locations and also there is a loss of a large part of the valuable agricultural land. 


#9. Given below are a few statements related to biodiversity. Pick those that correctly describe the concept of biodiversity  \n 1. Biodiversity refers to the different species of flora and fauna present in an area  \n 2. Biodiversity refers to only the flora of a given area  \n 3. Biodiversity is greater in a forest  \n 4. Biodiversity refers to the total number of individuals of a particular species living in an area  \n Choose the correct option from the following: 
A. 1 and 2 
B. 2 and 4 
C. 1 and 3 
D. 2 and 3 
Answer: C
Explanation: The term biodiversity refers to the variety of life forms on Earth. Forests are rich in biodiversity as a number of forms of life are found in forests, including trees, plants, animals, fungi and microorganisms, and their roles in nature. 



#10. Which among the statements given below is incorrect? 
A. Sustainable development does not take into consideration the viewpoints of all stakeholders 
B. Sustainable development is a long planned and persistent development 
C. Economic development is linked to environmental development 
D. Sustainable development meets the current basic human needs along with preserving resources for future generations 
Answer: A
Explanation: The Sustainable Development Goals agenda was accepted by all members of the United Nations in 2012 at the Rio De Janeiro Council Meet with an aim to promote a healthy and developed future of the planet and its people. It was in 2015 when the Sustainable Development Goals were implemented after a successful fifteen-year plan of development called the Millennium Development Goals. It is a group of 17 goals with 169 targets and 304 indicators, as proposed by the United Nation General Assembly’s Open Working Group on Sustainable Development Goals to be achieved by 2030. Post negotiations, agenda titled “Transforming Our World: the 2030 agenda for Sustainable Development” was adopted at the United Nations Sustainable Development Summit. SDGs is the outcome of the Rio+20 conference (2012) held in Rio De Janerio and is a non-binding document. The 17 goals under the Sustainable Development Goals are as mentioned below: 1. End poverty in all its forms everywhere 2. End hunger, achieve food security and improved nutrition and promote sustainable agriculture 3. Ensure healthy lives and promote well being for all at all stages 4. Ensure inclusive and equitable quality education and promote lifelong learning opportunities for all 5. Achieve gender equality and empower all women and girls 6. Ensure availability ans sustainable management of water and sanitation for all 7. Ensure access to affordable, reliable, sustainable and modern energy for all 8. Promote sustained, inclusive and sustainable economic growth, full and productive employment and decent work for all 9. Built resilient infrastructure, promote inclusive and sustainable industrialisation and foster innovation 10. Reduce inequalities within and among countries 11.Make cities and human settlements inclusive, safe, resilient and sustainable 12. Ensure sustainable consumption and production pattern 13. Take urgent actions to combat climate change and its impact 14. Conserve and sustainably use the oceans, seas and marine resources 15. Protect, restore and promote sustainable use of terrestrial ecosystems, sustainably managed forests, combat desertification and halt and reverse land degradation and halt biodiversity loss 16. Promote peaceful and inclusive societies for sustainable development, provide access to justice for all and build effective, accountable and inclusive institutions at all levels 17. Strengthen the means of implementation and revitalise the global partnership for sustainable development Sustainable Development Goals in India India’s record in implementing Sustainable Development Goals  Mahatma Gandhi National Rural Employment Guarantee Act (MNREGA) is being implemented to provide jobs to unskilled labourers and improve their living standards.  National Food Security Act is being enforced to provide subsidized food grains.  The government of India aims to make India open defecation free by the year 2019 under its flagship programme Swachh Bharat Abhiyan.  Renewable energy generation targets have been set at 175 GW by 2022 to exploit solar energy, wind energy and other such renewable sources of energy efficiency and reduce the dependence on fossil fuels.  Atal Mission for Rejuvenation and Urban Transformation (AMRUT) and Heritage City Development and Augmentation Yojana (HRIDAY) schemes have been launched for improving the infrastructure aspects.  India has expressed its intent to combat climate change by ratifying the Paris Agreement. 