#1. Angiospermae is a division of kingdom Plantae. The main characteristic feature of angiosperms is: 
A. Seeds are naked in fruits 
B. Seeds are covered with coats and are in fruits 
C. Fruits are without seeds 
D. Seeds are naked without fruits 
Answer: B
Explanation: Angiosperms are the largest group of plants on Earth. There are approximately 270,000 known species alive today. There's probably one nearby right now. Angiosperms include all plants that have flowers and account for approximately 80% of all known living plants. Example of an angiosperm: Carpel of Broomrape plant Characteristics: Angiosperms are able to grow in a variety of habitats. They can grow as trees, shrubs, bushes, herbs, and small flowering plants. Some of the characteristics of angiosperms include: 1. All angiosperms have flowers at some stage in their life. The flowers serve as the reproductive organs for the plant, providing them a means of exchanging genetic information. 2. Angiosperms have small pollen grains that spread genetic information from flower to flower. These grains are much smaller than the gametophytes, or reproductive cells, used by non-flowering plants. This small size allows the process of fertilization to occur quicker in the flowers of angiosperms and makes them more efficient at reproducing. 3. All angiosperms have stamens. Stamens are the reproductive structures found in flowers that produce the pollen grains that carry the male genetic information. 4. Angiosperms have much smaller female reproductive parts than non-flowering plants, allowing them to produce seeds more quickly. 5. Angiosperms have carpels that enclose developing seeds that may turn into a fruit. 6. A great advantage for angiosperms is the production of endosperm. Endosperm is a material that forms after fertilization and serves as a highly nutritional food source for the developing seed and seedling. 

#2. Which of the following have an open circulatory system?  \n (i) Arthropoda  \n (ii) Mollusca  \n (iii) Annelida \n (iv) Cnidaria 
A. (i) and (ii) 
B. (i) and (iii) 
C. (ii) and (iv) 
D. (ii) and (iii) 
Answer: A
Explanation: The open circulatory system is common to molluscs and arthropods. Open circulatory systems pump blood into a cavity (hemocoel) with the blood diffusing back to the circulatory system between cells. Blood is pumped by a heart into the body cavities, where blood surrounds the organs. This system lacks a true heart or capillaries. The blood found in these organisms is a mix of blood and an interstitial fluid called as hemolymph. 


#3. Which adaptation of the frog differentiates it from the fish? 
A. Ability to live in water also 
B. Lays eggs in water 
C. Have a streamlined body 
D. Presence of mucous glands in the skin 
Answer: A
Explanation: Due to mucous glands, these enable the frog to leave the water and live an amphibious life on both land and in water. Fish do not need mucous glands as they don’t leave the water so don’t need to protect their skin from drying out. A frog cannot live solely in the water, they need to breathe air and would drown if they stayed in the water, fish need to stay in the water to survive.

#4. Among the following choose the correct option that best describes the characteristics of spirogyra. 
A. Multicellular, auto trophic, root like rhizoids. 
B. Cytoplasmic strands, autotrophic, presence of rhizome. 
C. Presence of male cones, nonvascular, filaments. 
D. Filamentous, presence of Cytoplasmic strands, presence of pyrenoids. 
Answer: D
Explanation: The important characteristics of spirogyra are: 1. It is an unbranched, filamentous green algae occurring in the stagnant water so known as pond scum. 2. A mucilaginous covering or sheath is present which makes the surface slimy hence known as pond silk. 3. A double layered cell wall is present in which the outer wall is made of pectin and inner wall of cellulose. 4. A spirally coiled or ribbon shaped chloroplast with number of pyrenoids is present. 5. Reproduction occurs both in vegetative and sexual methods. 6. Vegetative reproduction is by fragmentation. 7. Sexual reproduction takes place by conjugation. 8. Three types of conjugation is found in spirogyra. They are i. scalariformconjugation, ii. Lateral conjugation, iii. Self conjugation. 9. The lifecycle of the spirogyra is haplobiontic.

#5. Choose the option that best describes the characteristics of the kingdom to which the mushroom belongs:
(a) Unicellular prokaryotic organisms
(b) Saprophytic, eukaryotic, multicellular organisms 
(c) Unicellular eukaryotic organisms 
(d) Autotrophic eukaryotic organisms 
Answer: B 
Explanation: Answer B

#6. Which of the following can sometimes be ‘zero’ for a moving body? \n i. Average velocity \n ii.
Distance travelle \niii. Average speed \n iv. Displacement 
(a) Only (i) 
(b) (i) and (ii) 
(c) (i) and (iv) 
(d) Only (iv) 
Answer: C
Explanation: 1. The average velocity of an object is its total displacement divided by the total time taken. In other words, it is the rate at which an object changes its position from one place to another. Average velocity is a Vector quantity. The SI unit is meters per second. 2. The distance travelled is the path taken by a body to get from an initial point to an end point in a given period of time, at a certain velocity. If the velocity is constant: Distance = time * velocity. 3. The average speed of an object is the total distance traveled by the object divided by the elapsed time to cover that distance. It's a scalar quantity which means it is defined only by magnitude. A related concept, average velocity, is a vector quantity. A vector quantity is defined by magnitude and direction. 4. If an object moves relative to a reference frame—for example, if a professor moves to the right relative to a whiteboard, or a passenger moves toward the rear of an airplane—then the object’s position changes. This change in position is known as displacement. The word displacement implies that an object has moved, or has been displaced. Displacement is defined to be the change in position of an object. 

#7. Which of the following statement is correct regarding velocity and speed of a moving body? 
(a) Velocity of a moving body is always higher than its speed 
(b) Speed of a moving body is always higher than its velocity 
(c) Speed of a moving body is its velocity in a given direction 
(d) Velocity of a moving body is its speed in a given direction 
Answer: D
Explanation: Answer D

#8. In a free fall the velocity of a stone is increasing equally ion equal intervals of time under the 
effect oravitational force of the earth. Then what can you say about the motion of this stone? Whether the stone is having: 
(a) Uniform acceleration 
(b) Non-uniform acceleration 
(c) Retardation 
(d) Constant speed 
Answer: A
Explanation: Uniform or constant acceleration is a type of motion in which the velocity of an object changes by an equal amount in every equal time period. A frequently cited example of uniform acceleration is that of an object in free fall in a uniform gravitational field. The acceleration of a falling body in the absence of resistances to motion is dependent only on the gravitational field strength g (also called acceleration due to gravity). 


#9. If the displacement of an object is proportional to square of time, then the object moves with: 
(a) Uniform velocity 
(b) Uniform acceleration 
(c) Increasing acceleration 
(d) Decreasing acceleration 
Answer: B
Explanation: If the displacement of an object is proportional to the square of the time taken then the body is moving with uniformly accelerated motion as it will follow Newton's second equation of motion for a particular initial velocity, which can be given by,s=ut+\frac { 1 }{ 2 } a{ t }^{ 2 }s=ut+21at2.


#10. In which of the following cases of motions, the distance moved and the magnitude of the displacement are equal? \ni. If the car is moving on a straight road \nii. If the car is moving in circular path \niii. The pendulum is moving to and fro iv. The earth is moving around the sun 
(a) only (ii) 
(b) (i) and (iii) 
(c) (ii) and (iv) 
(d) only (i) 
Answer: D
Explanation: If the car is moving on a straight road than the displacement is always equal to the distance, when an object is moving in the a straight path (displacement is defined as the shortest distance covered by a body)


#1. Newton’s third law of motion explains the two forces namely ‘action’ and ‘reaction’ coming into action when the two bodies are in contact with each other. These two forces: 
(a) Always act on the same body 
(b) Always act on the different bodies in opposite directions 
(c) Have same magnitude and direction 
(d) Acts on either body at normal to each other 
Answer: B
Explanation: Newton’s Third Law Of Motion Force is a push or pull acting on an object resulting in its interaction with another object. Force is a result of an interaction. Force can be classified into two categories: contact force such as frictional force and non-contact force such as gravitational force. According to Newton, when two bodies interact, they exerted force on each other and these forces are known as action and reaction pair which is explained in Newton’s third law of motion. Newton’s third law of motion states that: When one body exerts a force on the other body, the first body experiences a force which is equal in magnitude in the opposite direction of the force which is exerted. The above statement means that in every interaction, there is a pair of forces acting on the interacting objects. The magnitude of the forces are equal and the direction of the force on the first object is opposite to the direction of the force on the second object.,  

#2. In a rocket, a large volume of gases produced by the combustion of fuel is allowed to escape through its tail nozzle in the downward direction with the tremendous speed and makes the rocket to move upward. Which principle is followed in this take off of the rocket? (a) Moment of inertia (b) Conservation of momentum (c) Newton’s third law of motion (d) Newton’s law of gravitation 

Answer: B
Explanation: Law of conservation of momentum states that For two or more bodies in an isolated system acting upon each other, their total momentum remains constant unless an external force is applied. Therefore, momentum can neither be created nor destroyed. Law of conservation of momentum is an important consequence of Newton’s third law of motion. Following are the examples of law of conversation of momentum: 1. Air filled balloons,  2. System of gun and bullet 3. Motion of rockets 3. The seat belts are provided in the cars so that if the car stops suddenly due to an emergency braking, the persons sitting on the front seats are not thrown forward violently and saved from getting injured. Can you guess the law due to which a person falls in forward direction on the sudden stopping of the car? (a) Newton’s first law of motion (b) Newton’s second law of motion (c) Newton’s third law of motion (d) Newton’s law of gravitation 

Answer: A
Explanation: Sir Isaac Newton published three laws in the 17th century. In this article, we are going to talk about Newton’s 1st law. This law does an introduction of motion of the object and the force acting on it. In other words, it deals with the motion of an object and its relation to force. Newton’s first law states that: a body remains in the state of rest or uniform motion in a straight line unless and until an external force acts on it.,  Putting Newton’s 1st law of motion in simple words, a body will not start moving until and unless an external force acts on it. Once it is set in motion, it will not stop or change its velocity until and unless some force acts upon it once more. The first law of motion is sometimes also known as the law of inertia. There are two conditions on which the 1st law of motion is dependent: 1. Objects at rest: When an object is at rest velocity (v= 0) and acceleration (a = 0) are zero. Therefore, the object continues to be at rest. 2. Objects in motion: When an object is in motion, velocity is not equal to zero (v ≠ 0) while acceleration (a = 0) is equal to zero. Therefore, the object will continue to be in motion with constant velocity and in the same direction. What is an External Force: An external force is defined as the change in the mechanical energy that is either the kinetic energy or the potential energy in an object. These forces are caused by external agents. Examples of external forces are friction, normal force and air resistance. 4. Which of the following situations involves the Newton’s second law of motion? (a) A force can stop a lighter vehicle as well as a heavier vehicle which are moving,  (b) A force exerted by a lighter vehicle on collision with a heavier vehicle results in both the (vehicles coming to a standstill (c) A force can accelerate a lighter vehicle more easily than a heavier vehicle which are moving (d) A force exerted by the escaping air from a balloon in the downward direction makes the balloon to go upwards 

Answer: C
Explanation: Newton’s Second Law of Motion Newton’s second law of motion can be formally stated as follows: The acceleration of an object as produced by a net force is directly proportional to the magnitude of the net force, in the same direction as the net force, and inversely proportional to the mass of the object. The second law of motion gives us a method to measure the force acting on an object as a product of the mass of the object and the acceleration of the object which is the change in velocity with respect to time. 5. Newton’s first law of motion says that a moving body should continue to move forever, unless some external forces act on it. But a moving cycle comes to rest after some time if we stop pedaling it. Can you choose the correct reason for the stoppage of cycle?,  i. Air resistance ii. Gravitational pull of the earth iii. Friction of the road iii. Heat of the environment Choose the correct option: (a) (iii) and (iv) (b) (i) and (iii) (c) (i) and (ii) (d) (ii) and (iii) 
Answer: B 6. Two objects of different masses falling freely near the surface of moon would: (a) Have 
different accelerationb) Undergo a change in their inertia (c) Have same velocities at any instant (d) Experience forces of same magnitude,  

Answer: C
Explanation: Objects of different masses falling freely near the surface of the moon would have the same velocities at any instant because they will have same acceleration due to gravity. 7. The school bags are generally provided with the broad strips because: (a) It will spread the force of the bag over the large area of the shoulder of the child producing large pressure (b) It will spread the force of the bag over the large area of the shoulder of the child producing less pressure (c) It has become a trend among the students to carry the bags with wide strips (d) It will spread the force of the bag over the small area of the shoulder of the child producing less pressure 

Answer: B
Explanation: As we know that pressure is inversely proportional to area, so an increase in area means decrease in pressure.School bags are provided with broad straps to increase the surface area in contact with the shoulders and reduce pressure on the shoulders. If thin straps would be used, then surface area in contact with the shoulders would decrease which will lead to increase in pressure on shoulders.,  8. Choose the correct unit for the relative density among the following: (a) kg/cm (b) unitless (c) kg/cm (d) kg/m3 

Answer: B
Explanation: The difference between the specific gravity and density is that at room temperature and pressure is 1gram per 1 cubic cm is the density of water this density is treated as a standard and density of any other material (usually liquids) is calculated relative to the , this is called relative density or specific gravity. Hence, specific gravity is the ratio of the mass of a substance to that of a reference substance, let’s consider the density of honey is approx. 1.42 grams/cm3, so its specific gravity would be 1.42/1 = 1.42. Notice that specific gravity is a ratio, therefore specific gravity does not have a unit, and hence specific gravity is a dimensionless physical quantity. The specific gravity of a substance will let us know if it will float or sink, it gives us the idea about relative mass or relative density. If,  the specific gravity of a substance is below 1 then it will float and if it is greater than 1 it will sink. Relative density is the ratio of density of a substance to the density of a given reference material. Thus, it is a unit less quantity. 9. The earth and the moon are attracted to each other by gravitational force. The earth attracts the moon with a force that is: (a) More than that exerted by the moon (b) Same as that exerted by the moon (c) Less than that exerted by the moon (d) Not related to that exerted by the moon 

Answer: B
Explanation: Gravitational attraction is caused by the mass of an object. Since Earth is far more massive than the Moon, the gravitational force exerted on the Moon is far greater than that of the Moon on the Earth. An example of the difference: while the Moon causes tides on the Earth, the Earth has the Moon locked so that the same face (minus some wobbling) is always visible from the Earth.,  10. An apple falls from a tree because of the gravitational attraction between the earth and the apple. If F1 is the magnitude of the force exerted by the earth on the apple and F2 is the magnitude of the force exerted by the apple on the earth, then (a) F1 is very much greater than F2 (b) F2 is very much greater than F1 (c) F1 and F2 are equal (d) F1 is only a little greater than F2 
Answer: C The force of attraction f is equal to f₂ , option (d),  1. In the SI system, the unit of P.E. is: A. Erg 
B. Newto. Dyne D. Joule 

Answer: D
Explanation: The term potential energy was introduced by the 19thcentury Scottish engineer and physicist William Rankine. There are several types of potential energy, each associated with a distinct type of force. It is the energy by virtue of an object’s position relative to other objects. We can define potential energy as: The energy held by an object because of its position relative to other objects, stresses within itself, its electric charge, or other factors. Similarly, in the case of a spring, when it is displaced from its equilibrium position, it gains some amount of energy which we observe in the form of stress we feel in our hand upon stretching it. We can define potential energy as a form of energy that results from the alteration of its position or state. Potential Energy Formula: The formula for potential energy depends on the force acting on the two objects. For the gravitational force the formula is: W = m×g×h = mgh,  Where, m is the mass in kilograms g is the acceleration due to gravity h is the height in meters Unit: Gravitational potential energy has the same units as kinetic energy: kg m2 / s2 Note: All energy has the same units - kg m2 / s2, and is measured using the unit Joule (J). 2. What happens to the kinetic energy of the body if its velocity is doubled? A. Remains same B. Becomes double C. Becomes four times D. Becomes half 

Answer: C
Explanation: A force must be applied on a body to accelerate an object. Work must be done in order to apply a force. The body will move with an unvarying speed after the work has been done due to the energy provided by it. The speed and the mass of the body are factors on which the energy transfer that makes up the kinetic energy depends.,  The kinetic energy of an object is the energy that it possesses due to its motion. Kinetic energy definition is given as: The energy of an object because of its motion or the energy gained by an object from its state of rest to motion. Formula of Kinetic Energy: Following is the formula of kinetic energy: KE=12mv2 Where, KE is the kinetic energy of the object m is the mass of an object v is the velocity of an object Kinetic energy is an example of scalar quantity which means that the quantity has only magnitude and no direction. Unit of Kinetic Energy: The SI unit of kinetic energy is Joule which is equal to 1 kg.m2.s-2. The CGS unit of kinetic energy is erg. What are the Examples of Kinetic Energy? 1. A semi-truck travelling down the road has more kinetic energy than a car travelling at the same speed because the truck’s mass is much more than the car’s.,  2. A river flowing at a certain speed comprises kinetic energy as water has certain velocity and mass. 3. The kinetic energy of an asteroid falling towards earth is very large. 4. The kinetic energy of the aeroplane is more during the flight due to large mass and speedy velocity. 3. Expression for Power of an object is equal to: A. Power = Work done X Time B. Power = Time/ Work done C. Power = Work done/Time D. Power = Force X Displacement 

Answer: C
Explanation: Power is always dependent on work done, so if a person does work at different rates his power also differs at different times. This is where the concept of average-power comes into the picture. What is Power? We can define power as the rate of doing work, it is the work done in unit time. The SI unit of power is Watt (W) which is joules per second (J/s). Sometimes the power of motor vehicles and other,  machines are given in terms of Horsepower (hp), which is approximately equal to 745.7 watts. What is Average Power? We can define average power as the total energy consumed divided by the total time taken. In simple language, we can say that average power is the average amount of work done or energy converted per unit of time. Power Formula: Power is defined as the rate at which work is done upon an object. Power is a time-based quantity. which is related to how fast a job is done. The formula for power is mentioned below. Power = Work / time P = W / t Unit of Power: The unit for standard metric work is the Joule and the standard metric unit for time is the second, so the standard metric unit for power is a Joule / second, defined as a Watt and abbreviated W. 4. In winters, rubbing of hands together for some time, causes a sensation of warmth mainly because of: A. heat caused by the force of friction,  B. heat caused by the momentum C. heat caused by the motion D. heat flows from the blood to skin 

Answer: A
Explanation: If you rub your hands together for several seconds, you'll notice that your hands feel warm. That warmth is caused by a force called friction. When objects like your hands come in contact and move against each other, they produce friction. Friction is defined as: The resistance offered by the surfaces that are in contact with each other when they move over each other. Friction works in the opposite direction in which the body is moving making the body slow down. Friction is useful in most of the cases. Friction is also dependent on the external factors. Factors Affecting Friction: Following are the two factors on which friction depends 1. On the nature of the two surfaces that are in contact: Friction is dependent on the smoothness or roughness of the two surfaces that are in contact with each other. When the surface is smooth, the friction between the two reduces as there is no much interlocking of irregularities taking place. While the surface is rough, friction increases. 2. On the force that is acting on these surfaces: When force is applied along with the irregularities, friction increases.,  What Causes Friction? When we see any object, we can see the smooth surface but when the same object is viewed under a microscope, it can seen that even the smooth appearing object has rough edges. Tiny hills and grooves can be seen through the microscope and they are known as irregularities of the surface. So, when one object is moved over the other, these irregularities on the surface get entangled giving rise to friction. More the roughness, more will the irregularities and greater will be the force applied. 5. The value of 1 Kilo Watt Hour is A. 1.8 X 105J B. 3.6 X 106J C. 5.4 X 108J D. 7.2 X 1010J 
Answer: B 6. According to the law of conservation of energy, A. Energy can only be transformed from one form to 
another,  Bt can neither be created nor destroyed C. The total energy before and after the transformation always remains constant D. All of the above 

Answer: D
Explanation: Energy is required for the evolution of life forms on earth. In Physics, it is defined as the capacity to do work. We know that energy exists in different forms in nature. You have learned about various forms of energy - heat, electrical, chemical, nuclear, etc. In this article, we will learn about the laws and principles that govern energy. This law is known as the law of conservation of energy. What is the Law of Conservation of Energy? The law of conservation of energy states that energy can neither be created nor be destroyed. Although, it may be transformed from one form to another. If you take all forms of energy into account, the total energy of an isolated system always remains constant. All the forms of energy follow the law of conservation of energy. In brief, the law of conservation of energy states that: In a closed system, i.e., a system that is isolated from its surroundings, the total energy of the system is conserved. Law of Conservation of Energy Examples: In physics, most of the inventions rely on the fact that energy is conserved when it is transferred from one form to another. A,  number of electrical and mechanical devices operate solely on the law of conservation of energy. We will discuss a few examples here.  In a torch, the chemical energy of the batteries is converted into electrical energy, which is converted into light and heat energy.  In hydroelectric power plants, waterfalls on the turbines from a height. This, in turn, rotates the turbines and generates electricity. Hence, the potential energy of water is converted into the kinetic energy of the turbine, which is further converted into electrical energy.  In a loudspeaker, electrical energy is converted into sound energy.  In a microphone, sound energy is converted into electrical energy.  In a generator, mechanical energy is converted into electrical energy.  When fuels are burnt, chemical energy is converted into heat and light energy.  Chemical energy from food is converted to thermal energy when it is broken down in the body and is used to keep it warm. 7. On an object the work done does not depend upon: A. Displacement B. Angle between force and displacement,  C. Force applied D. Initial velocity of an object 

Answer: D
Explanation: The amount of work done depends on the amount of the force applied and the distance object moves along with the angle between force and displacement. W=F\times S\times cos\theta=F×S×cosθ 8. What is the smallest unit of power? A. Watt B. Kilowatt C. Horse power D. Milliwatt 
Answer: A 9. Which of the following device converts chemical energy in to electrical energy? A. Battery B. Loud 
Speaker,  Colar Cell D. Electric Motor 

Answer: A
Explanation: What is Energy? Energy is defined as the ability to do work where work is the movement of a body to be some force. We need energy all the time and energy comes in various forms. What is Chemical Energy? Chemical energy is defined as : the energy which is stored in the bonds of chemical compounds (molecules and atoms). It is released in the chemical reaction and mostly produces heat as a by-product, known as an exothermic reaction. The examples of stored chemical energy are biomass, batteries, natural gas, petroleum, and coal. Mostly, when the chemical energy is released from a substance, it is transformed into a new substance completely. For instance, when an explosion goes off, the chemical energy in it is transferred to the surroundings as thermal energy, kinetic energy, and sound energy. Chemical Energy in Everyday Life:  We know that plants need solar energy to produce sugar from carbon dioxide to water. Sugar, water and carbon dioxide stay together by chemical bonds that hold the chemicals together.,   For instance, all sugars consist of oxygen, carbon and hydrogen atoms held together by chemical bonds. These atoms do not connect together automatically; rather some energy is required to make them stay together.  Plants use solar energy to put the hydrogen, the carbon and the oxygen atoms as a whole in the form of sugar. This is a suitable example of energy transformation where energy is transformed from one form to another. Here, solar energy is transformed into chemical energy and prevents from falling apart. 10. Our planet Earth receives or transfers most of its energy in the form of A. Radiant energy. B. Geo-thermal energy C. Chemical energy D. Wind energy 

Answer: A
Explanation: It is actually well known that heat is simply a form of energy which transfers hotness from one point to the other. It is gets transfer from one point to another by three modes of which radiation is the one. The most common illustration for it is electromagnetic radiation. Light gets transferred in the form of,  electromagnetic radiation. Other examples are solar radiation, nuclear radiation etc Electromagnetic radiation is continuously emitted from all substances because of the molecular as well as atomic vibration accompanying the internal energy. The equilibrium state gives the internal energy proportional to the material temperature as well as that’s the reason we’re not aware of these radiant energy which constantly surrounds us as our body can detect only some portions of the spectrum. Our eyes can detect light as they’re very sensitive to changes in light occurring around us as well as our sensitive skin can detect heat radiation however unless these heat radiations is quite large we can’t sense it . The Radiant heat energy is a notion which is rather more nicely explained by Stefan’s law. This law elucidates how the heat is radiated! It states that, The amount of heat which is radiated (E) by a flawless black body for a second in a provided unit area is directly related to the fourth power of its absolute temperature (T).,  1. Match the column: Column-I Column-II A. Microphone 1.Wind energy into mechanical energy B. Speaker 2.Mechanical energy into sound energy C. Reeds of a harmonium 3.Electrical energy into sound energy D. Sails of a ship 4.Sound energy into electrical energy A. a - 4, b - 3, c - 1, d - 2 B. a - 1, b - 2, c - 3, d - 4 C. a - 4, b - 2, c - 3, d - 1 D. a - 4, b - 3, c - 2, d - 1 

Answer: D
Explanation: Microphone − Sound energy into electrical energy. Speaker − Electrical energy into sound energy. Reeds of a harmonium − Mechanical energy into sound energy. Sails of a ship − Wind energy into mechanical energy. 2. Arrange the following media in ascending order of the speed of sound in them, 1. Water 2. Steel 3. Nitrogen A. 3, 2 and 1 B. 1, 3 and 2 C. 3, 1 and 2 D. 2, 1 and 3 

Answer: C
Explanation: Sound waves travel through a medium by alternately contracting and expanding the parts of the medium in which it travels. The speed of sound is the distance travelled per unit time by any sound. In the next section few sections, let us learn to determine the speed of sound in various medium. What is Speed of Sound? The speed of sound is defined as the dynamic propagation of sound waves. This depends on the characteristics of the medium through which the propagation takes place. Speed of sound is used for describing the speed of sound waves in an elastic medium. Speed of Sound Formula The formula for speed of sound is given with respect to gases. It is the square root of product of coefficient of adiabatic expansion and pressure of the gas divided by the density of the medium. The mathematical representation is given as: What is the Speed of Sound in Air? The speed of sound is an essential parameter used in a variety of field in Physics. The speed of sound refers to the distance travelled per unit time by a sound wave propagating through a medium. The speed of sound in air at 20oC is 343.2 m/s which translates to 1,236 km/h. The speed of sound in gases is proportional to the square root of the absolute temperature (measured in Kelvin) but it is independent of the frequency of the sound wave or the pressure and the density of the medium. But none of the gases we find in real life are ideal gases and this causes the properties to slightly change. Speed of Sound in Solid, Liquid and Gases Sound can travel through wood too! In fact, sound likes travelling through solid more. In the olden days, doctors used stethoscopes consisting of thin wooden rods with broadened ends and they worked exactly like our modern ones and performed just as well. How can sound travel through solids? Speed of Sound in Solid Sound is nothing more than a disturbance which is propagated by the collisions between the particles; one molecule hitting the next and so forth. Solids are significantly denser than liquids or gases. This means that the molecules are closer to each other in solids than in liquids and in liquids than in gases. This closeness due to density means that they can collide very quickly. Effectively it takes less time for a molecule of a solid to bump into its neighbouring molecule. Due to this advantage, the speed of sound in a solid is larger than in a gas. Speed of Sound in Liquid Similarly, the density of a liquid is greater than the density of a gas. Therefore the distances between molecules are more in liquids than in solids but are less than in gases. Hence the speed of sound in liquids lies in between the speed of sound in solids and gases. Speed of Sound in Gas We should remember that the speed of sound is independent of the density of the medium when it enters a liquid or solid. Since gases expand to fill the given space, density is quite uniform irrespective of the type of gas. This clearly isn’t the case with solids and liquids. 3. What is the minimum distance between two crests called? A. Wavelength B. Amplitude C. Displacement D. Wave pulse 

Answer: A
Explanation: Wavelength can be defined as the distance between two successive crests or troughs of a wave. It is measured in the direction of the wave. Description: Wavelength is the distance from one crest to another, or from one trough to another, of a wave (which may be an electromagnetic wave, a sound wave, or any other wave). Crest is the highest point of the wave whereas the trough is the lowest. Since wavelength is distance/length, it is measured in units of lengths such as metres, centimetres, milimetres, nanometres, etc. 4. What type of waves are Sound Waves? A. Latitudinal waves B. Longitudinal waves C. Latitudinal mechanical waves D. Longitudinal waves 

Answer: D
Explanation: A sound is a form of energy, just like electricity, heat or light. Let’s examine some sources of sounds like a bell. When you strike a bell, it makes a loud ringing noise. Now, instead of just listening to the bell, put your finger on the bell after you have struck it. Can you feel it vibrating? This is the key to the sound. It is even more evident in guitars and drums. You can see the wires vibrating every time you pluck it. When the bell or the guitar stops vibrating, the sound also stops. The to and fro motion of the body is termed as Vibration. You can see examples of vibrations everywhere. Vibrating objects produce sound. Some vibrations are visible, some aren’t. If you pull and then release a stretched rubber band, the band moves to and fro about the central axis and while doing so it also produces a sound. The sound moves through a medium by alternately contracting and expanding parts of the medium it is travelling through. In physics, Sound is a vibration that propagates as an acoustic wave, through a transmission medium such as a gas, liquid or solid. Sound is pictorially represented by a continuous succession of peaks and valleys. The distance between two consecutive peaks or trough is termed as the wavelength of the wave or the period. The number of cycles per unit time is termed as the frequency of the sound. Frequency is measured in cycles per second or Hertz. The faster an object vibrates, i.e. the higher the frequency, then the higher the pitch of the sound. The difference between a man’s voice and women must be clearly evident to you. The voice of a man has a lower frequency which contributes to the deepness of the bass in the voice. Women, in contrast, have a voice with higher frequency resulting in a higher shrillness or pitch. What Is Longitudinal Wave? Longitudinal waves are the waves where the displacement of the medium is in the same direction, or in opposite direction, and direction of the travel of the wave. The distance between the centers of two consecutive regions of compression or the rarefaction is defined by wavelength, λ. When the compression and rarefaction regions are of two waves coincide with each other, it is known as constructive interference and if the regions of compression and rarefaction do not coincide, it is known as destructive interference. Sound Waves: A sound wave is an example of a longitudinal wave and is produced by the vibrating motion of the particles that travel through a conductive medium. The example of sound waves in a longitudinal direction is tuning fork. In Sound waves, the amplitude of the wave is the difference between the maximum pressure caused by the wave and the pressure of the undisturbed air. The propagation speed of sound depends upon the type, composition of the medium and temperature through which it propagates. 5. Which of the following is/ are not applications of Ultrasonic Waves? A. For measuring the depth of Sea. B. In sterilizing of a liquid. C. In Ultrasonography D. In sterilizing a needle. Options are: A. Both A and B B. Only D C. Both C and D D. Only B 

Answer: B
Explanation: What is Ultrasound? Sound waves with frequencies higher than the upper audible limit of human hearing are called ultrasound. The limit varies from person to person but is approximately 20,000 Hertz. The physical properties of ultrasound are similar to the normal audible sound. This type of scientific concept is used in many different fields such as navigation, medicine, imaging, cleaning, mixing, communication, testing etc. Even in nature, bats and porpoises use this particular technique for the location of prey and obstacles. In the following section, we shall learn about its applications. Applications: 1. Cleaning: In objects with parts that are difficult to reach, for example, spiral tubes and electronic components, the process of ultrasonic cleaning is used. Here, the object is dipped in a solution of suitable cleaning material and ultrasonic waves are passed into it. As a result of this, high-frequency waves are generated that cause the dirt and grease to detach from the surface. 2. Detection of cracks: Ultrasound is used to detect cracks in the metallic components that are used in the construction of high rise structures such as buildings and bridges. They generate and display an ultrasonic waveform that is interpreted by a trained operator, often with the aid of analysis software, to locate and categorize flaws in test pieces. High-frequency sound waves reflect from flaws in predictable ways, producing distinctive echo patterns that can be displayed and recorded by portable instruments. A trained operator identifies specific echo patterns corresponding to the echo response from good parts and from representative flaws. The echo pattern from a test piece may then be compared to the patterns from these calibration standards to determine its condition. 3. Echocardiography: In the process of electrocardiography, the ultrasonic waves are used to form an image of the heart using reflection and detection of these waves from various parts. 4. Ultrasonography: ,Medical ultrasound is a diagnostic imaging technique based on it. It is used for the imaging of internal body structures such as muscles, joints and internal organs. Ultrasonic images are known as sonograms. In this process, pulses of ultrasound are sent to the tissue using a probe. The sound echoes off the tissue, where different tissues reflect sound varying in degrees. These echoes are recorded and displayed an image. 5. Lithotripsy: Ultrasonic waves are used to break stones in the kidney. High energy sound waves are passed through the body without injuring it and break the stone into small pieces. These small pieces move through the urinary tract and out of the body more easily than a large stone. 6. SONAR: SONAR, Sound Navigation, and Ranging is a technique in which sound waves are used to navigate, detect and communicate under the surface of the water. 7. Echolocation: Echolocation is the process where sound waves and echoes are used to determine objects in space. Echolocation is used by bats to navigate and find their food in the dark. Bats send out sound waves from their mouth and nose, which then hit the objects in their vicinity producing echoes, which are then received by the bats. The nature of the echo helps them determine the size, the shape and the distance of the object. 6. What will be the effect of temperature on speed of sound? A. The speed of sound decreases with the increases of temperature of the medium. B. The speed of sound decreases with the decrease of temperature of the medium. C. The speed of sound increases with the decrease of temperature of the medium. D. The speed of sound increases with the increase of temperature of the medium. 

Answer: D
Explanation: The speed of sound increases with the increase of temperature of the medium. The speed of sound in air increases by 0.61 m/s when the temperature is increased by10C. 7. What is the unit of loudness? A. Bel B. Phon C. Decibel D. All of the above 

Answer: D
Explanation: The sensation of a sound perceived in a ear is measured by another term called loudness which depends on intensity of sound and sensitiveness of the ear. Unit of loudness is bel. A practical unit of loudness is decibel (dB) which is 1/10th of bel. Another unit of loudness is phon. 8. Which of the following statement is or are correct about longitudinal mechanical waves? A. The longitudinal mechanical waves which lie in the frequency range 20 Hz to 20000 Hz are called audible or sound waves. B. The longitudinal mechanical waves having frequencies less than 20 Hz are called infrasonic. C. The longitudinal mechanical waves having frequencies greater than 20,000 Hz are called ultrasonic waves. D. All of the above are correct 

Answer: D
Explanation: Sound or Audible waves are sensitive to human ear and are generated by the vibrating bodies like tuning fork, vocal cords etc. Infrasonic waves are produced by sources of bigger size such as earth quakes, volcanic eruptions, ocean waves etc. Human ear cannot detect Ultrasonic waves. But dog, cat, bat etc can detect these waves. Bat not only detect but also produce ultrasonic waves. 9. Due to which phenomena sound is heard at longer distances in nights than in day? A. Reflection B. Refraction C. Interference of sound D. Diffraction of sound 

Answer: B
Explanation: Due to refraction, sound is heard at longer distances in nights than in day. 10. Name the characteristic of the sound which distinguishes a sharp sound from a grave or dull sound? A. Intensity B. Echo C. Pitch D. Resonance 

Answer: C
Explanation: Pitch is that characteristic of sound which distinguishes a sharp or shrill sound from a grave or dull sound. It depends upon frequency. Higher the frequency higher will be the pitch and shriller will be the sound and vice versa. 1. The anti-viral protein is known as: A. Antibodies B. Interferon C. Antibiotics D. Virus protein 

Answer: B
Explanation: Viral proteins are proteins generated by a virus. As viruses hijack much of their host's cellular machinery to support their life cycle, they encode very few of their own genes; viral proteins are therefore generally structural components, for the viral envelope and capsid. Non-structural proteins and regulatory or accessory proteins can also be viral proteins. Antiviral proteins are proteins that are induced by human or animal cells to interfere with viral replication. These proteins are isolated to inhibit the virus from replicating in a host's cells and stop it from spreading to other cells. 2. What is the other name of rabbis? A. Chemo phobia B. Herpetophobia C. Ophidiophobia D. Hydrophobia 

Answer: D
Explanation: Rabies is a viral disease that is spread through the animal bite such as the dog. It is caused by the infection of rabies virus. The infection caused from this leads to encephalomyelitis i.e the inflammation of the brain as well as the spinal cord. The transmission of the virus happens through the saliva and affects the CNS or Central nervous system. This virus belongs to the family called Rhabdoviridae. It takes the shape of a bullet. Animals such as dogs, rabbits, cats, fox and etc carry this virus and transmit the disease to human beings. Usually, this disease causes about 24,000 - 60,000 deaths in worldwide per year. Control and Prevention of Rabies Prevention is better than cure. Rabies is an infectious disease that can be preventable. Following measures can reduce the infection to nearly high extents. Some of the measures are given below-  Get rabies vaccination injected to prevent the infection.  Vaccinating your pet against the disease.  Maintain distance with the wild animals.  Wash wounds with soap and water and maintain good hygiene.  Keep your pets away from the other stray dogs.  Prevent bats wandering around your campuses and living places. 3. Children are vaccinated against polio because 1. Vaccination kills polio causing microbes 2. Prevents the entry of polio causing organisms 3. It creates immunity against the virus. 4. All the above 
Answer: D 4. Which of the following is a mismatch? 1. Leprosy- Bacterial infection 2. AIDS- Bacterial infection 
3. Malariarotozoan infection 4. Common cold- Viral infection 

Answer: B
Explanation: A. Leprosy Definition “Leprosy is a chronic infection that affects the skin, mucous membrane, and nerves, and causes discolouration, lumps, disfigurement and deformities in skin.” What is Leprosy? Leprosy is a disease that causes severe, scarring skin sores and nerve damage in the limbs. Leprosy disease has affected people on every continent. Leprosy is actually not that infectious but it spreads when a healthy person comes in regular and close contact with mouse droplets and leprosy patient. Children get more affected by this disease than adults. Almost, 180000 people all over the world get infected with leprosy. The Leprosy disease mainly affects the peripheral nerves, skin, upper respiratory tract and the eyes. The most prevalent possibility of transmission is through the respiratory route. Leprosy is also transmitted through insects. B. What is AIDS? AIDS or Acquired Immune Deficiency Syndrome is a syndrome caused by the HIV virus. In this condition, a person’s immune system becomes too weak to fight off any kind of infection or disease. AIDS is usually the last stage of HIV infection; a stage where the body can no longer defend itself and thus spawns various diseases. AIDS, when untreated, leads to death. AIDS is an advanced HIV infection or late-stage HIV. Someone with AIDS may develop a wide range of health conditions like - pneumonia, thrush, fungal infections, TB, toxoplasmosis and cytomegalovirus. There is also an increased risk of developing a medical illness like cancer and brain illnesses. CD4 count refers to the number of Thelper cells in a cubic millilitre of blood. A person may be referred to as “AIDS-affected” when the CD4 count drops below 200 cells per cubic millilitre of blood. C. What is Malaria? Malaria is a mosquito-borne infectious disease caused by various species of the parasitic protozoan microorganisms called Plasmodium. Malaria is a disease that man has battled with for a long time. The first evidence of this protozoan came from mosquitoes preserved in amber nearly 30 million years ago. It is even thought to have brought the Roman Empire to its knees. Malaria was so prevalent during the Roman times that the disease is also called ‘Roman Fever’. Today, the credit for actually discovering the parasite is given to Charles Louis Alphonse Laveran, a French physician. He even won the Nobel Prize in 1907 for his findings. Malaria parasite exists in the form of a motile sporozoite. The vector of malaria i.e. the female Anopheles mosquito transmits the malarial sporozoites into the hosts. When an infected mosquito bites a human, the sporozoites are injected into the blood through the mosquito’s saliva. The sporozoites travel into our body and accumulate in the liver. These parasites initially multiply within the liver, by damaging the liver and rupturing the blood cells in the body. Malaria kills by causing the destruction of the red blood cells in the host. The parasites reproduce asexually in the RBCs, bursting the cells and releasing more parasites to infect more cells. The rupture of red blood cells by the malaria parasite releases a toxin called hemozoin which causes the patient to experience a condition known as the chills. When the female Anopheles mosquito bites an infected human, the parasites enter the mosquito’s body along the human blood it is drinking. It is inside the mosquito’s body that the actual development and maturing of the parasite happens. The parasites produced in the human body reach the intestine of the mosquito where the male and females cells fertilize each other to lead to the formation of a sporozoite. On maturing, the sporozoite breaks out the mosquito’s intestine and migrate to the salivary glands. Once they reach salivary glands, they wait till the mosquito bites another human and the process of infection and disease begins all over again. It is prudent however to observe that the complete development of the malaria parasite takes place in two different hosts; humans and mosquitoes. D. What is Common Cold? Cold or common cold is a disease diagnosed with a headache, runny nose, scratchy throat, fever and non stop sneezing. It is a viral infectious disease of the upper respiratory tract which primarily affects the nose and sometimes sinuses, ears, and bronchial tubes. A common cold is caused by viruses. Some of them include -  Rhinovirus - This one usually intrudes your system during early fall, spring, and summer. They are behind 10%-40% of colds. Even though these are the main common viruses which affect you, they would rarely make you seriously sick.  Coronavirus - The virus affects the human system during winter and early spring. This virus is behind 20% of colds. There are more than 30 types of coronavirus, out of which only 3 or 4 ones are harmful.  RSV and parainfluenza - These tiny organisms are behind severe infections like pneumonia, in young children. 5. Which bacteria is responsible for causing typhoid? A. Salmonella typhi B. E. Coli C. Coccus D. Bacilli 

Answer: A
Explanation: What Is Typhoid? Typhoid is an infectious bacterial disease that mainly spreads through contaminated food or water. It can also spread due to the poor hygienic conditions. The major symptoms of this disease are characterized by high fever, loss of appetite and diarrhoea. Salmonella typhi is the bacterium responsible for this disease and humans are the only carriers. The first case of typhoid fever was reported in the United States in the early 1900’s. Overall, about 21 million people are infected with this disease annually, and about 200,000 cases are fatal. Furthermore, scientists have identified 2 types of typhoid causing bacterium, namely: 1. ST1 2. ST2 Causes Of Typhoid: Also called as “Salmonella enterica serotype Typhi”, this microbe is the causative agent for this disease. It is a gram-negative bacteria characterized by a thin cell wall and an outer membrane. The cells are reddish in colour, with some having black stains in the centre. It is rod-shaped and grows in the small intestine of the human body. Human beings are only the main hosts of this bacteria. This type of species can survive in environments which are rich in oxygen and also, they are found in sewage, water bodies and some eventually make their own on to food. The bacteria enter the human body through the contaminated foods and water, where it then enters into the intestinal cells of the human body. Later, it passes through the bloodstream and destroys the lymphatic system and spreads throughout the body. This bacterium is mainly carried by the white blood cells present in the liver and also the bone marrow. There, they multiply and re-enter the blood cells, which in turn, causes a number of symptoms to appear in the later stages. 6. The disease which are present since birth and are due to some genetic disorder is called: A. Congenital disease B. Infectious C. Non infectious D. Chronic disease 

Answer: A
Explanation: Congenital refers to a condition or disease which is present at birth. The condition can be inherited (genetic) or caused by environmental factors. Some maternal infections, such as HIV, can be passed onto the child and cause a congenital condition. Maternal factors such as alcohol or drug consumption, nutritional intake and placental health can all cause congenital problems Examples  Congenital heart defects are those which affect the structure of the heart and the way blood flows through it. Ranging from minor—with no symptoms—to life-threatening, these are the most common type of birth defect  Cleft lip and palate, which affect the development of the roof of the mouth and upper lip. A cleft lip can be caused by genes passed down from the parents, environmental toxins, viruses or may occur in concert with other birth defects.  Neural tube defects such as spina bifida  Down Syndrome caused by an extra chromosome 7. Diarrhea, cholera, typhoid are the diseases that have one thing in common that is A. All of them are caused by bacteria B. All of them is transmitted by contaminated food and water C. All of them are cured by antibiotics D. All of the above 
Answer: D 8. Which of the following statements is correct regarding vaccination? A. It develops resistance 
against pathogettack. B. It kills pathogen causing disease C. It blocks the food supplied to pathogens D. It does not allow pathogens to multiply in hosts. 

Answer: A
Explanation: A vaccine is a preparation that improves immunity to a particular disease. It is a biological prepared product which contains typical agents resembling a microorganism that causes disease, made from weakened or dead forms of the microbes, one of its surface proteins or its toxins. It helps in the stimulation of the immune system and to identify the invaded microbes as the foreign agent and destroy it so that the immune system can be recognized and destroy any microorganism encountered later. A vaccine is an antigenic substance that develops immunity against a disease which can be delivered through needle injections or by mouth or by aerosol.Vaccination is the injection of a dead or weakened organism that forms immunity against that organism in the body.Immunization is the process by which an animal or a person stays protected from diseases. Vaccines are Safe and Effective: Vaccines are the perfect defense against a preventable and contagious disease that can be deadly. Vaccines are one of the safest medical products available but there are some preventive measures one should adopt. Precise information of the values of vaccines along with their possible side-effects assist people to take decisions on vaccines. How Well Do Vaccines Work? No medicine can be labeled as perfect but most of the vaccines produce immunity for about 90-100% of the cases. Certainly, better sanitation and hygiene can help prevent the spread of disease but the germs that are responsible still stay around. The germs continue to make people sick as long as their existence. Every vaccine has to be licensed by the Food and Drug Administration abbreviated by FDA before being brought into use in the United States. A vaccine needs to go through extensive tests to confirm that it is safe before the approval of FDA. Among these tests are the clinical tests trials that compare groups of people who get a control such as a placebo with the group of people who get a vaccine. A vaccine is approved only when FDA confirms that it is safe for intended use. Vaccines save millions of lives every year. When a particular section of a city or town is immunized against a communicable disease, several members of the same community are shielded against the diseases as the opportunity for an outbreak is minimum. The principle of immunity refers to the control of various contagious diseases that involve rabies, mumps, influenza, measles and pneumococcal disease. 9. If you live in an overcrowded and poorly ventilated house, it is possible that you may suffer from A. Waterborne disease B. Airborne disease C. Congenital disease D. Sexually transmitted disease 

Answer: B
Explanation: Airborne Transmission: Some infectious agents remain suspended in the air for a long period of time. These pathogens might attack the immune system of a person in contact. For eg., if you enter a room that was initially occupied by a patient of measles, you too might catch the infection. 10. An organism which harbours a pathogen and may pass it on to another person to cause a disease is known as A. Host B. Vector C. Parasite D. Predator 

Answer: B
Explanation: A vector is an organism that does not cause disease itself but that transmits infection by conveying pathogens from one host to another. Vectors may be mechanical or biological. 1. Arrange the following components in decreasing order according to their percentage present in air. A. Nitrogen, Oxygen, Argon, Carbon dioxide. B. Oxygen, Argon, Carbon dioxide, Nitrogen. C. Nitrogen, Carbon dioxide, Oxygen, Argon. D. Nitrogen, Oxygen, Carbon dioxide, Argon. 

Answer: A
Explanation: The major components of the air are: Nitrogen - 78% Oxygen - 20% Argon - 0.93% Carbon dioxide - 0.031% Other gases such as Neon, Helium, Hydrogen etc, are present in trace amounts. The increasing order of the boiming points of the above gases is : Nitrogen (lowest) Argon Oxygen Carbon dioxide 2. The term "water-pollution" can be defined in several ways. Which of the following statements does not give the correct definition? A. The addition of undesirable substances to water-bodies B. The removal of desirable substances from water-bodies C. A change in pressure of the water bodies D. A change in temperature of the water bodies 

Answer: C
Explanation: Various Causes of Water Pollution 1. Industrial waste: Industries produce a huge amount of waste which contains toxic chemicals and pollutants which can cause air pollution and damage to us and our environment. They contain pollutants such as lead, mercury, sulfur, asbestos, nitrates, and many other harmful chemicals. Many industries do not have a proper waste management system and drain the waste in the fresh water which goes into rivers, canals and later into the sea. The toxic chemicals have the capability to change the color of water, increase the number of minerals, also known as eutrophication, change the temperature of water and pose a serious hazard to water organisms. 2. Sewage and wastewater: The sewage and wastewater that is produced by each household is chemically treated and released into the sea with fresh water. The sewage water carries harmful bacteria and chemicals that can cause serious health problems. Pathogens are known as a common water pollutant; The sewers of cities house several pathogens and thereby diseases. Microorganisms in water are known to be causes of some very deadly diseases and become the breeding grounds for other creatures that act as carriers. These carriers inflict these diseases via various forms of contact onto an individual. A very common example of this process would be Malaria. 3. Mining activities: Mining is the process of crushing the rock and extracting coal and other minerals from underground. These elements when extracted in the raw form contains harmful chemicals and can increase the number of toxic elements when mixed up with water which may result in health problems. Mining activities emit a large amount of metal waste and sulphides from the rocks which is harmful to the water. 4. Marine dumping: The garbage produced by each household in the form of paper, aluminum, rubber, glass, plastic, food is collected and deposited into the sea in some countries. These items take from 2 weeks to 200 years to decompose. When such items enter the sea, they not only cause water pollution but also harm animals in the sea. 5. Accidental oil leakage: Oil spill poses a huge concern as a large amount of oil enters into the sea and does not dissolve with water; thereby opens problem for local marine wildlife such as fish, birds and sea otters. For e.g.: a ship carrying a large quantity of oil may spill oil if met with an accident and can cause varying damage to species in the ocean depending on the quantity of oil spill, size of the ocean, the toxicity of pollutant. 6. The burning of fossil fuels: Fossil fuels like coal and oil when burnt produce a substantial amount of ash in the atmosphere. The particles which contain toxic chemicals when mixed with water vapor result in acid rain. Also, carbon dioxide is released from the burning of fossil fuels which result in global warming. 7. Chemical fertilizers and pesticides: Chemical fertilizers and pesticides are used by farmers to protect crops from insects and bacterias. They are useful for the plant’s growth. However, when these chemicals are mixed up with water produce harmful for plants and animals. Also, when it rains, the chemicals mix up with rainwater and flow down into rivers and canals which pose serious damages for aquatic animals. 8. Leakage from sewer lines: A small leakage from the sewer lines can contaminate the underground water and make it unfit for the people to drink. Also, when not repaired on time, the leaking water can come on to the surface and become a breeding ground for insects and mosquitoes. 9. Global warming: An increase in earth’s temperature due to the greenhouse effect results in global warming. It increases the water temperature and results in the death of aquatic animals and marine species which later results in water pollution. 10. Radioactive waste: Nuclear energy is produced using nuclear fission or fusion. The element that is used in the production of nuclear energy is Uranium which is a highly toxic chemical. The nuclear waste that is produced by radioactive material needs to be disposed of to prevent any nuclear accident. Nuclear waste can have serious environmental hazards if not disposed of properly. Few major accidents have already taken place in Russia and Japan. 11. Urban development: As the population has grown, so has the demand for housing, food, and cloth. As more cities and towns are developed, they have resulted in increasing use of fertilizers to produce more food, soil erosion due to deforestation, increase in construction activities, inadequate sewer collection, and treatment, landfills as more garbage is produced, increase in chemicals from industries to produce more materials. 12. Leakage from the landfills: Landfills are nothing but a huge pile of garbage that produces the awful smell and can be seen across the city. When it rains, the landfills may leak and the leaking landfills can pollute the underground water with a large variety of contaminants. 13. Animal waste: The waste produced by animals is washed away into the rivers when it rains. It gets mixed up with other harmful chemicals and causes various water-borne diseases like cholera, diarrhea, jaundice, dysentery and typhoid. 14. Underground storage leakage: Transportation of coal and other petroleum products through underground pipes is well known. Accidentals leakage may happen anytime and may cause damage to the environment and result in soil erosion. 3. Which one of the following is not a cause of ozone depletion? A. Oxides of nitrogen B. Oxides of sulphur dioxide C. CFCs D. None of these 

Answer: D
Explanation: To understand ozone layer, it would be helpful to know the different layers of the atmosphere. The earth’s atmosphere is composed of many layers, each playing a significant role. The first layer stretching approximately 10 kilometers upwards from the earth’s surface is known as the troposphere. A lot of human activities such as gas balloons, mountain climbing, and small aircraft flights take place within this region. The stratosphere is the next layer above the troposphere stretching approximately 15 to 60 kilometers. The ozone layer sits in the lower region of the stratosphere from about 20-30 kilometers above the surface of the earth. The thickness of the ozone layer is about 3 to 5 mm, but it pretty much fluctuates depending on the season and geography. Ozone layer is a deep layer in earth’s atmosphere that contain ozone which is a naturally occurring molecule containing three oxygen atoms. These ozone molecules form a gaseous layer in the Earth’s upper atmosphere called stratosphere. This lower region of stratosphere containing relatively higher concentration of ozone is called Ozonosphere. The Ozonosphere is found 15-35 km (9 to 22 miles) above the surface of the earth. The concentration of ozone in the ozone layer is usually under 10 parts per million while the average concentration of ozone in the atmosphere is about 0.3 parts per million. The thickness of the ozone layer differs as per season and geography. The highest concentrations of ozone occur at altitudes from 26 to 28 km (16 to 17 miles) in the tropics and from 12 to 20 km (7 to 12 miles) towards the poles. Why Ozone Layer is Necessary? An essential property of ozone molecule is its ability to block solar radiations of wavelengths less than 290 nanometers from reaching Earth’s surface. In this process, it also absorbs ultraviolet radiations that are dangerous for most living beings. UV radiation could injure or kill life on Earth. Though the absorption of UV radiations warms the stratosphere but it is important for life to flourish on planet Earth. Research scientists have anticipated disruption of susceptible terrestrial and aquatic ecosystems due to depletion of ozone layer. Ultraviolet radiation could destroy the organic matter. Plants and plankton cannot thrive, both acts as food for land and sea animals, respectively. For humans, excessive exposure to ultraviolet radiation leads to higher risks of cancer (especially skin cancer) and cataracts. It is calculated that every 1 percent decrease in ozone layer results in a 2-5 percent increase in the occurrence of skin cancer. Other ill-effects of the reduction of protective ozone layer include - increase in the incidence of cataracts, sunburns and suppression of the immune system. Causes of Ozone Layer Depletion: Credible scientific studies have substantiated that the cause of ozone layer depletion is human activity, specifically, human-made chemicals that contain chlorine or bromine. These chemicals are widely known as ODS, an acronym for Ozone-Depleting Substances. The scientists have observed reduction in stratospheric ozone since early 1970’s. It is found to be more prominent in Polar Regions. Ozone-Depleting Substances have been proven to be eco-friendly, very stable and non-toxic in the atmosphere below. This is why they have gained popularity over the years. However, their stability comes at a price; they are able to float and remain static high up in the stratosphere. When up there, ODS are comfortably broken down by the strong UV light and the resultant chemical is chlorine and bromine. Chlorine and bromine are known to deplete the ozone layer at supersonic speeds. They do this by simply stripping off an atom from the ozone molecule. One chlorine molecule has the capability to break down thousands of ozone molecules. Ozone-depleting substances have stayed and will continue to stay in the atmosphere for many years. This, essentially, implies that a lot of the ozone-depleting substances human have allowed to go into the atmosphere for the previous 90 years are still on their journey to the atmosphere, which is why they will contribute to ozone depletion. The chief ozone-depleting substances include chlorofluorocarbons (CFCs), carbon tetrachloride, hydrochlorofluorocarbons (HCFCs) and methyl chloroform. Halons, sometimes known as brominated fluorocarbons, also contribute mightily to ozone depletion. However, their application is greatly restricted since they are utilized in specific fire extinguishers. The downside to halons is they are so potent that they are able to deplete the ozone layer 10 times more than ozone-depleting substances. Scientists in this age are working around the clock to develop Hydrofluorocarbons (HFCs) to take the place of hydrochlorofluorocarbons (HCFCs) and chlorofluorocarbons (CFCs) for use in vehicle air conditioning. Hydrochlorofluorocarbons are powerful greenhouse gases, but they are not able to deplete ozone. Chlorofluorocarbons, on the other hand, significantly contribute to climate change, which means Hydrofluorocarbons continue to be the better alternative until safer alternatives are available. There are two regions in which the ozone layer has depleted.  In the mid-latitude, for example, over Australia, ozone layer is thinned. This has led to an increase in the UV radiation reaching the earth. It is estimated that about 5-9% thickness of the ozone layer has decreased, increasing the risk of humans to over-exposure to UV radiation owing to outdoor lifestyle.  In atmospheric regions over Antarctica, ozone layer is significantly thinned, especially in spring season. This has led to the formation of what is called ‘ozone hole’. Ozone holes refer to the regions of severely reduced ozone layers. Usually ozone holes form over the Poles during the onset of spring seasons. One of the largest such hole appears annually over Antarctica between September and November. 4. The nitrogen molecules present in air can be converted into nitrates and nitrites by: A. A biological process of nitrogen fixing bacteria present in soil. B. A biological process of carbon fixing factor present in soil. C. Any of the industries manufacturing nitrogenous compounds. D. The plants used as cereal crops in field. 

Answer: A
Explanation: Nitrogen fixation is the essential biological process and the initial stage of the nitrogen cycle. In this process, nitrogen in the atmosphere is converted into ammonia (another form of nitrogen) by certain bacterial species like Rhizobium, Azotobacter, etc. and by other natural phenomena. Plants are the main of the sources of food. The nutrients obtained from plants are synthesized by plants using various elements which they obtain from the atmosphere as well as from the soil. This group of elements includes nitrogen as well. Plants obtain nitrogen from the soil through the process of protein synthesis. Unlike carbon dioxide and oxygen, atmospheric nitrogen cannot be obtained through stomata of leaves. Because the concentration of nitrogen gas present in the atmosphere can not be directly used by plants and also the concentration of the usable form of nitrogen in the atmosphere is very less. There are certain bacteria and some natural phenomenon which help in Nitrogen fixation. 5. The water cycle in nature involves which of the following processes sequentially? A. Evaporation, precipitation and condensation. B. Precipitation, condensation and evaporation. C. Evaporation, condensation and precipitation. D. Condensation, evaporation and precipitation. 
Answer: C 6. Select the incorrect statement regarding the rainfall with low pH. A. It affects aquatic life 
and corrodeetal, stone, etc. B. It is caused by reaction of sulphur and nitrogen oxides with moisture in air. C. It is the consequence of increased use of non-conventional sources of energy. D. It affects plants also. 

Answer: C
Explanation: Rain with low pH is called acid rain. Acid rain is caused when sulphuric acid and nitric acid are formed in atmosphere by reaction of oxides of nitrogen and sulphur with moisture present in air. These oxides are released into atmospheric air by combustion of fossil fuels, i.e., conventional sources of energy. 7. Match column I with column II and select the correct option from the given codes. Column I Column II (a) Particulate matter (i) Chemical water pollutant (b) Heat (ii) Non-degradable soil pollutant (c) Detergent (iii) Degradable (d) Plastic (iv) Air pollutant (e) Vegetable peel (v) Physical water pollutant A) (a)-(iv), (b)-(iii), (c)-(i), (d)-(ii), (e)-(v) B) (a)-(iv), (b)-(v), (c)-(i), (d)-(ii), (e)-(iii) C) (a)-(i), (b)-(ii), (c)-(iii), (d)-(iv), (e)-(v) D) (a)-(i), (b)-(ii), (c)-(iii), (d)-(v), (e)-(iv) 
Answer: B 8. Which one among the following is not a layer of atmosphere? A. Mesosphere B. Stratosphere C. 
Lithosphere Droposphere 

Answer: C
Explanation: There are five layers in the structure of the atmosphere depending upon temperature. These layers are:  Troposphere  Stratosphere  Mesosphere  Thermosphere  Exosphere Troposphere  It is considered as the lowest layer of Earth’s atmosphere.  The troposphere starts at surface of the earth and goes up to a height of 7 to 20 km.  All weather occurs within this layer.  This layer has water vapour and mature particles.  Temperature decreases at the rate of 1 degree Celsius for every 165 m of height.  Tropopause separates Troposphere and Stratosphere. Stratosphere  It is the second layer of the atmosphere found above the troposphere.  It extends up to 50 km of height.  This layer is very dry as it contains little water vapour.  This layer provides some advantages for flight because it is above stormy weather and has steady, strong, horizontal winds.  The ozone layer is found in this layer.  The ozone layer absorbs UV rays and safeguards earth from harmful radiation.  Stratopause separates Stratosphere and Mesosphere. Mesosphere  The Mesosphere is found above the stratosphere.  It is the coldest of the atmospheric layers.  The mesosphere starts at 50 km above the surface of Earth and goes up to 85 km.  The temperature drops with altitude in this layer.  By 80 km it reaches -100 degrees Celsius.  Meteors burn up in this layer.  The upper limit is called Mesopause which separate Mesosphere and Thermosphere. Thermosphere  This layer is found above Mesopause from 80 to 400 km.  Radio waves which are transmitted from the earth are reflected back by this layer.  The temperature increases with height.  Aurora and satellites occur in this layer. Ionosphere  The lower Thermosphere is called the Ionosphere.  Ionosphere consists of electrically charged particles known as ions.  This layer is defined as the layer of the atmosphere of Earth that is ionized by cosmic and solar radiation.  It is positioned between 80 and 400 km above the Mesopause. Exosphere  It is the outermost layer of the atmosphere.  The zone where molecules and atoms escape into space is mentioned as the exosphere.  It extends from the top of the thermosphere up to 10,000 km. 9. Few characteristics of a type of soil are given below. Find the type of the soil based on the given characteristics. 1. Presence of iron oxides in the soil. 2. Formed by the erosion of rocks in areas of high temperature. 3. Soil appears red in colour. 4. Found in western and eastern ghats. 5. Soil is not very fertile. A. Alluvial soil B. Laterite soil C. Red soil D. Black soil 
Answer: B 10. What would happen, if all the oxygen present in the environment is converted to ozone? A. We will 
be protecteore. B. It will become poisonous and kill living forms. C. Ozone is not stable, hence it will be toxic. D. It will help harmful sun radiations to reach earth and damage many life forms. 

Answer: B
Explanation: Ozone is a poisonous gas and is thus only present in a thin layer in the stratosphere. lf all the oxygen is converted to ozone, the environment will become poisonous and kill all living forms. Ozone is O3, in its elemental form or even in its combined form it is highly toxic. Ozone can damage lungs and even low amounts of ozone can cause chest pain, coughing, shortness of breath and lung irritation.However at the stratospheric level, no life form exists and there ozone prevent us from the harmful U.V radiations of the sun. 1. Which of the following are the three primary nutrients needed for plant growth? A. Calcium, sulphur and magnesium. B. Nitrogen, phosphorus and potassium. C. Zinc, boron and copper. D. Calcium/zinc/copper. 

Answer: B
Explanation: Soil is a major source of nutrients needed by plants for growth. The three main nutrients are nitrogen (N), phosphorus (P) and potassium (K). Together they make up the trio known as NPK. Other important nutrients are calcium, magnesium and sulfur. Plants also need small quantities of iron, manganese, zinc, copper, boron and molybdenum, known as trace elements because only traces are needed by the plant. The role these nutrients play in plant growth is complex, and this document provides only a brief outline. Major elements: 1. Nitrogen (N) Nitrogen is a key element in plant growth. It is found in all plant cells, in plant proteins and hormones, and in chlorophyll. Atmospheric nitrogen is a source of soil nitrogen. Some plants such as legumes fix atmospheric nitrogen in their roots; otherwise fertiliser factories use nitrogen from the air to make ammonium sulfate, ammonium nitrate and urea. When applied to soil, nitrogen is converted to mineral form, nitrate, so that plants can take it up. Soils high in organic matter such as chocolate soils are generally higher in nitrogen than podzolic soils. Nitrate is easily leached out of soil by heavy rain, resulting in soil acidification. You need to apply nitrogen in small amounts often so that plants use all of it, or in organic form such as composted manure, so that leaching is reduced. 2. Phosphorus (P) Phosphorus helps transfer energy from sunlight to plants, stimulates early root and plant growth, and hastens maturity. Very few Australian soils have enough phosphorus for sustained crop and pasture production and the North Coast is no exception. The most common phosphorus source on the North Coast is superphosphate, made from rock phosphate and sulfuric acid. All manures contain phosphorus; manure from grain-fed animals is a particularly rich source. 3. Potassium (K) Potassium increases vigour and disease resistance of plants, helps form and move starches, sugars and oils in plants, and can improve fruit quality. Potassium is low or deficient on many of the sandier soils of the North Coast. Also, heavy potassium removal can occur on soils used for intensive grazing and intensive horticultural crops (such as bananas and custard apples). 4. Calcium (Ca) Calcium is essential for root health, growth of new roots and root hairs, and the development of leaves. It is generally in short supply in the North Coast's acid soils. Lime, gypsum, dolomite and superphosphate (a mixture of calcium phosphate and calcium sulfate) all supply calcium. Lime is the cheapest and most suitable option for the North Coast; dolomite is useful for magnesium and calcium deficiencies, but if used over a long period will unbalance the calcium/magnesium ratio. Superphosphate is useful where calcium and phosphorus are needed. 5. Magnesium (Mg) Magnesium is a key component of chlorophyll, the green colouring material of plants, and is vital for photosynthesis (the conversion of the sun's energy to food for the plant). Deficiencies occur mainly on sandy acid soils in high rainfall areas, especially if used for intensive horticulture or dairying. Heavy applications of potassium in fertilisers can also produce magnesium deficiency, so banana growers need to watch magnesium levels because bananas are big potassium users. Magnesium deficiency can be overcome with dolomite (a mixed magnesium-calcium carbonate), magnesite (magnesium oxide) or epsom salts (magnesium sulfate). 6. Sulfur (S) Sulfur is a constituent of amino acids in plant proteins and is involved in energy-producing processes in plants. It is responsible for many flavour and odour compounds in plants such as the aroma of onions and cabbage. Sulfur deficiency is not a problem in soils high in organic matter, but it leaches easily. On the North Coast seaspray is a major source of atmospheric sulfur. Superphosphate, gypsum, elemental sulfur and sulfate of ammonia are the main fertiliser sources. 2. Which of the following is a food-fodder mixed farming system? A. Growing food crops alone B. Growing fodder crops alone C. Growing crops which can be used as food and fodder by using more fertilizers. D. Growing food crops and fodder crops in the same field. 

Answer: D
Explanation: Mixed farming system can be defined as - (i).The use of a single farm for multiple purposes, as the growing of cash crops or the raising of livestock is called as mixed farming. (ii).Farming involving both the growing of crops and the keeping of livestock is called as mixed farming. (iii). A type of commercial agriculture concerned with the production of both crops and animals on one farm. Stock on a mixed farm used to be grazed on fallow land, but many modern mixed farms produce some, or all, of their fodder crops. Mixed farming is a system of farming in which a farmer conducts different types of agricultural practices together, on a single farm in view of increasing his income through different sources, is called as mixed farming. But, what is a farm? & what is Farming? Receiving the radiant energy of sun through crops and cattle is called as farming whereas, the place where these practices are done, is called as a farm. Here is one example: Rearing of cattle and growing fodder for them on a piece of land is called as farming. In other words, Mixed farming is the combining of two independent agricultural enterprises on the same farm. In mixed farming a farmer can take up different types of practices for income generation while doing his main business of agriculture. Some of these practices that can be done together with the main agricultural practices are - poultry faming, dairy farming, bee keeping, sericulture, Pisciculture, shrimp farming, goat and sheep rearing, piggery and agro forestry. Thus a farmer can raise his income manifold through carrying out different farming practices together. The greatest benefit from this type of farming is that if any one business does not pay desired benefit, the same can be recovered from the benefit of the other business. Merits of Mixed Farming System: The mixed farming system is the largest category of livestock system in the world which covers about 2.5 billion hectares of land, of which 1.1 billion hectares are arable rainfed crop land, 0.2 billion hectares are irrigated cropland and 1.2 billion hectares are grassland. This system of farming produces 92% of the world’s milk supply. Some of the important merits of mixed farming are mentioned below. 1. This farming system maintains soil fertility by recycling soil nutrients and allowing the introduction and use of rotations between various crops and forage legumes and trees, or for land to remain fallow and grasses and shrubs to become reestablished; 2. Mixed farming maintains soil biodiversity, minimize soil erosion, help to conserve water and provide suitable habitats for birds; 3. It makes the best use of crop residues. When they are not used as feed, stalks may be incorporated directly into the soil, where, for some time, they act as a nitrogen trap, exacerbating deficiencies. In the tropical semi-arid areas, termite action results in loss of nutrients before the next cropping season. Burning, the other alternative, increases carbon dioxide emissions; and 4. Mixed farming allows intensified farming, with less dependence on natural resources and preserving more biodiversity than would be the case if food demands were to be met by crop and livestock activities undertaken in isolation. There are a number of methods that are adopted in the mixed farming system. Some of these methods are being described below 1. Food -Fodder Farming In this method of mixed farming, the fodder crops are also grown along with other crops. Farmers can grow Sorghum, Pusa Giant Napier; berseem etc. as fodder crops for their cattle alongwith food crops. It is through this system that the availability of high variety of fodder is ensured for milch cattle while growing crops for production of grains, pulses, vegetables, oil and fruits etc. 2. Agroforestry System According to the World Forestry Centre (ICRAF) 1993, - “Agroforestry system is a collective name forthe land use systems and practices in which woody perennial plants are deliberately integrated with crops (and some times animals) on the same land management unit. The integration can be either in a spatial mixture or in a temporal sequence. There are normally both ecological and economic interactions between woody and non-woody components in Agroforestry". Again in the year 2003, the ICRAF further defined the agroforestry system as - a dynamic, ecology based, natural resources management system that, through the integration of trees on farms and in the agricultural landscape, diversifies and sustains production for increased social, economic and environmental benefits for land users at all levels (World Agroforestry Centre, 2003). 3. Why is organic manure considered better than fertilisers? A. It cannot be prepared in fields. B. It is less rich in plant nutrients. C. It improves the fertility and texture of the soil. D. It decreases the water holding capacity of the soil. 

Answer: C
Explanation: Organic manure or organic fertilizer is made up of organic biodegradable waste like animal excreta and agriculture wastes. As it biodegradable it cannot harm the environment and increase the soil fertility it is better than chemical fertilizer. But chemical fertilizers are nutrient specific and prepared in chemical factories. 4. Which of the following refers to inland fisheries? A. Culturing fish in fresh water B. Deep sea fisheries C. Trapping and capturing fish from sea coast D. Extraction of oil from fishes 

Answer: A
Explanation: Inland fisheries are any activity conducted to extract fish and other aquatic organisms from "inland waters". The term "inland waters" is used to refer to lakes, rivers, brooks, streams, ponds, inland canals, dams, and other land-locked (usually freshwater) waters (such as the Caspian Sea, Aral Sea, etc.). Whilst most inland waters are freshwater( i.e. zero salinity), there are many areas that are classified nationally as inland waters which have daily or seasonal fluctuations in salinity (e.g. estuaries, deltas, some coastal lagoons). Some areas are permanently brackishwater (coastal lagoons, the Caspian Sea, Lake Van) or even hypersaline (e.g. The Utah Great Salt Lake). Fisheries in inland waters have long provided an important source of food for mankind. The global population now stands at 7.6 billion and is projected to rise to 9.7 billion people by 2050. Inland capture fisheries have an important role to play in the global challenge to sustainably feed this growing population, as they deliver quality nutrition to some of the world’s most vulnerable populations in a manner that is both accessible and affordable. Inland fisheries are critical for a group of developing countries in the world, providing an important source of nutrition, food security as well as micronutrients. These nutritional and food security benefits are an integral part of the agricultural landscape of these countries, thus inland fisheries are closely linked to food production, water and land management, biodiversity and ecosystems. Inland fisheries are also increasingly impacted and changed as countries develop their water and land resources for agriculture. They are under increasing pressure and threats arising from far reaching changes to the aquatic environment arising from human activities such as damming, navigation, wetland reclamation for agriculture, urbanization, water extraction and transfer, and waste disposal. 5. Which of the following is an important objective of biotechnology in the field of agriculture? A. To produce disease resistant varieties of plants B. To increase the nitrogen content C. To decrease the seed number D. To increase the plant weight 

Answer: A
Explanation: Biotechnology is being used to address problems in all areas of agricultural production and processing. This includes plant breeding to raise and stabilize yields; to improve resistance to pests, diseases and abiotic stresses such as drought and cold; and to enhance the nutritional content of foods. Biotechnology is being used to develop low-cost, disease-free planting materials for crops. 6. Who is called the "Father of Green Revolution" in India? A. M.S. Ramaiah B. Aryabhatta C. M.S. Swaminathan D. Jawaharlal Nehru 
Answer: C 7. What is the advantage of crop rotation? A. Getting different kinds of crops B. Increasing the 
quality oinerals C. Increasing the fertility of soil D. Increasing the quality of proteins 

Answer: C
Explanation: Principles of crop rotation The basic principles of crop rotations are as follows:  Deep rooted crops should be succeeded by shallow rooted crop s such as cotton, castor, pigeon peapotato, lentil, green gram etc.  Dicot crops should be rotated by monocot crops such as mustar d, potato- rice, wheat- sugarcane.  Leguminous crops should be succeeded by nonleguminous crops and vice versa (green gram- wheat).  Exhaustive crops should be succeeded with restorative crops s uch as potato, sorghum, sugarcane, castor- sunhemp, black gra m, cowpea.  Grain crops should be followed by foliage crops such as, wheatdhaincha, black gram.  Long duration crops should be succeeded by short duration cro ps such as sugarcane, napier, Lucerne- cowpea, black gram, grou nd nut.  Crops susceptible to soil borne pathogens and parasitic weeds s hould be followed by tolerant trap crops such as sugarcane- ma rigold, mustard (for nematodes); tobacco- rice, pulses (for oro banche); pearl millet- castor (for striga); lucern, berseem- oats (for cuscuta).  Crops with problematic weeds should be followed by clean crop s/ multi cut crops and other dissimilar crops such as wheat- pu ddle rice for Phalaris minor; berseem- potato for Chicorium int ybus; rice- vegetables for Echinochloa crusgalli.  Heavy irrigation and intensive labour requiring crops should be followed by less water and labour requiring crops such as sugar cane, paddy- mungbean and sesame. Advantages of crop rotation Following are the advantages of an appropriate crop rotation:  Higher yield without incurring extra investment.  Enhance soil fertility and microbial activity.  Avoid accumulation of toxic substances.  The legumes in a cropping system, assimilating nitrogen from th e atmosphere and enriching the soil with their root system.  Better utilization of nutrients and soil moisture.  Insurance against natural devastation.  Maintain soil health by avoiding insect pest, diseases and weed problems.  Provide proper labour, power and capital distribution throughou t the year.  Higher chances to provide diversified commodities.  Slow but steady income, which is beneficial to marginal and sma ll farmers.  Deep rooted crops work the soil below plough layer. 8. Which of these is a macronutrient of plant? A. Copper B. Iron C. Chlorine D. Magnesium 

Answer: D
Explanation: Macronutrients include carbon, hydrogen, nitrogen, oxygen, phosphorous, potassium, calcium, sulfur, and magnesium. Micronutrients are boron, chlorine, manganese, iron, zinc, copper, and molybdenum. A plant uses these nutrients to support its growth, life cycle, and biological functions. 9. Which is the Indian breed of high milk-yielding variety of cow? A. Jersey B. Ongole C. Sahiwal D. Red sindhi 

Answer: C
Explanation: Sahiwal is a breed of 'Zebu cattle' and is considered to be one of the best milch cattle breed in India. The breed has derived its name from the Sahiwal area in Montgomery district of Punjab in Pakistan. 10. Which of the following organisms is/are causes diseases in poultry? A. Bacteria B. Fungi C. Virus D. All of the above 

Answer: D
Explanation: Poultry fowl suffers from a number of diseases caused by virus, bacteria, fungi, parasites as well as from nutritional deficiencies. ACHIEVE IAS MCQ SERIES, DAY 31, SCIENCE, SOLUTIONS 1. Regarding the Chemical reactions, consider the following statements: 1. New atoms are produced in a chemical reaction. 2. Exothermic reactions are accompanied by absorption of Heat. Which of the statements given above is/are correct? A. 1 only B. 2 only C. Both 1 and 2 D. Neither 1 nor 2 

Answer: D
Explanation: During a chemical reaction atoms of one element do not change into those of another element. Nor do atoms disappear from the mixture or appear from elsewhere, in fact chemical reactions involve the breaking and making of bonds between atoms to produce new substances. Exothermic reactions are accompanied by release of Heat. 2. Which of the following best defines the oxidation of any substance? A. Gain in oxygen B. Loss in oxygen ACHIEVE IAS MCQ SERIES, DAY 31, SCIENCE, SOLUTIONS C. Gain in hydrogen D. Loss in hydrogen 

Answer: A
Explanation: If a substance gains oxygen during a reaction, it is said to be oxidised. If a substance loses oxygen during a reaction, it is said to be reduced. If in a reaction, one reactant gets oxidised while the other gets reduced, it is called oxidation-reduction reactions or redox reactions. 3. Regarding the oxidation, consider the following statements: 1. The Black coating on silver and the green coating on copper are examples of Corrosion. 2. Bags of chips are flushed with gas such as nitrogen to prevent the chips from getting oxidised. Which of the statements given above is/are correct? A. 1 only B. 2 only C. Both 1 and 2 D. Neither 1 nor 2 ACHIEVE IAS MCQ SERIES, DAY 31, SCIENCE, SOLUTIONS 

Answer: C
Explanation: When a metal is attacked by substances around it such as moisture, acids, etc., it is said to be corroded and this process is called corrosion. The black coating on silver and the green coating on copper are examples of corrosion. Corrosion causes damage to car bodies, bridges, iron railings, ships and to all objects made up of metals, especially those of iron. When fats and oils are oxidised, they become rancid and their smell and taste change. Usually substances which prevent oxidation (antioxidants) are added to foods containing fats and oil. Keeping food in air tight containers helps to slow down oxidation. Due to this reason, bags of chips are flushed with gas such as nitrogen to prevent the chips from getting oxidised. 4. Regarding the lime, consider the following statements: 1. Calcium hydroxide is also called as quicklime. 2. Calcium hydroxide reacts slowly with the carbon dioxide in air to form a thin layer of calcium carbonate on the walls after two to three days of whitewashing. Which of the statements given above is/are correct? A. 1 only B. 2 only ACHIEVE IAS MCQ SERIES, DAY 31, SCIENCE, SOLUTIONS C. Both 1 and 2 D. Neither 1 nor 2 

Answer: B
Explanation: Calcium oxide is called as quicklime. Calcium oxide reacts vigorously with water to produce slaked lime (calcium hydroxide) releasing a large amount of heat. Calcium hydroxide reacts slowly with the carbon dioxide in air to form a thin layer of calcium carbonate on the walls. Calcium carbonate is formed after two to three days of white washing and gives a shiny finish to the walls. 5. Respiration is the type of which of the following reactions? A. Combination reaction B. Decomposition reaction C. Exothermic reaction D. Endothermic reaction 

Answer: C
Explanation: Respiration is an exothermic process. Reactions in which heat is released along with the formation of products are called exothermic chemical reactions. As we all know, we need energy to stay alive. We get this energy from the food we eat. During digestion, food is broken down into simpler substances. For ACHIEVE IAS MCQ SERIES, DAY 31, SCIENCE, SOLUTIONS example, rice, potatoes and bread contain carbohydrates. These carbohydrates are broken down to form glucose. This glucose combines with oxygen in the cells of our body and provides energy. 6. The neutralization reaction between an acid and a base is a type of: A. Double displacement reaction B. Displacement reaction C. Addition reaction D. Decomposition reaction 

Answer: A
Explanation: A double displacement reaction is a type of reaction in which two reactants exchange ions to form two new compounds. Double displacement reactions typically result in the formation of a product that is a precipitate. Double displacement reactions take the form: AB + CD → AD + CB Key Takeaways: Double Displacement Reaction 1. A double displacement reaction is a type of chemical reaction in which the reactant ions exchange places to form new products. ACHIEVE IAS MCQ SERIES, DAY 31, SCIENCE, SOLUTIONS 2. Usually, a double displacement reaction results in precipitate formation. 3. The chemical bonds between the reactants may be either covalent or ionic. 4. A double displacement reaction is also called a double replacement reaction, salt metathesis reaction, or double decomposition. The reaction occurs most often between ionic compounds, although technically the bonds formed between the chemical species may be either ionic or covalent in nature. Acids or bases also participate in double displacement reactions. The bonds formed in the product compounds are the same type of bonds as seen in the reactant molecules. Usually, the solvent for this type of reaction is water. 8. Which of the following gases is used in the storage of fat and oil containing foods for a long time? A. Carbondioxide gas B. Nitrogen gas C. Oxygen gas D. Neon gas 

Answer: B
Explanation: Helium or nitrogen both can be used for storage of fresh sample of an oil for a long time. These gases are called inert gases as they do not react with most elements including oxygen. ACHIEVE IAS MCQ SERIES, DAY 31, SCIENCE, SOLUTIONS Thus these gases create an inert environment for the oil and prevent its reaction with any element in the environment that prevent from getting it rancid. 9. The respiration process during which glucose undergoes slow combustion by combining with oxygen in the cells of our body to produce energy, is a kind of: A. Exothermic process B. Endothermic process C. Reversible process D. Physical process 
Answer: A 10. A chemical reaction does not involve: A. Formation of new substances having entirely different 
properties thahat of the reactants B. Breaking of old chemical bonds and formation of new chemical bonds C. Rearrangement of the atoms of reactants to form new products D. Changing of the atoms of on element into those of another element to form new products ACHIEVE IAS MCQ SERIES, DAY 31, SCIENCE, SOLUTIONS 
Answer: D Changing of the atoms of on element into those of another element to form new products 1. Regarding a 
'humatomach', consider the following statements: 1. Human stomach produces sulphuric acid that helps in the digestion of food without harming the stomach. 2. During indigestion, the stomach produces too much acid and this causes pain and irritation. Which of the statements given above is/are correct? A. 1 only B. 2 only C. Both 1 and 2 D. Neither 1 nor 2 

Answer: B
Explanation: Our stomach produces hydrochloric acid. It helps in digestion of food without harming the stomach. During indigestion, the stomach produces too much acid and this causes pain and irritation. To get rid of this pain, people use bases called antacids. These antacids neutralise the excess acid. Magnesium hydroxide (Milk of magnesia), a mild base, is often used for this purpose. 2. Consider the following statements: 1. The human body works within the pH range of 7.0 to 7.8. 2. When pH of rain water is less than 5.6, it is called acid rain. 3. Tooth decay starts when the pH value of the mouth increases. Which of the statements given above is/are correct? A. 1 and 2 only B. 2 and 3 only C. 3 only D. 1, 2 and 3 only 

Answer: A
Explanation: Our body works within the pH range of 7.0 to 7.8. Living organisms can survive only in a narrow range of pH change. When pH of rain water is less than 5.6, it is called acid rain. When acid rain flows into the rivers, it lowers the pH of the river water. The survival of aquatic life in such rivers becomes difficult. Tooth decay starts when pH of the mouth is lower than 5.5. Tooth enamel, made up of calcium phosphate is the hardest substance in the body. It does not dissolve in water, but gets corroded when the pH level in the mouth goes below 5.5. Using toothpastes, which are generally basic, for cleaning the teeth can neutralise the excess acid and thus can prevent tooth decay . 3. Match List-I with List-II and using the code given below, select the correct 
answer: List-I (matter) List-II (pH value) A. Human blood 1. 5.5 to 7.5 B. Milk 2. 7.3 to 7.5 C. Human saliva 3. 6.4 
D. Humarine 4. 6.5 to 7.5 A B C D A. 2 3 1 4 B. 2 3 4 1 C. 3 2 1 4 D. 1 3 2 4 

Answer: A
Explanation: The correct combination of the above list is as follows: (Substance) - (pH value) Human blood - 7.3 to 7.5 Milk - 6.4 Human saliva - 5.5 to 7.5 Human urine - 6.5 to 7.5 In addition, Lemon juice - 2.2 (approx) Gastric juice - 1.2 (approx.) Blood- 7.4 Pure water- 7 Sodium Hydroxide- 14 (Approx.) Milk of magnesia- 10 4. Match List-I with List-II and using the code given below, select the correct 
answer: List I (Natural source) List II (Acid) A. Bee sting 1. Tartaric acid B. Tamarind 2. Methanoic acid C. 
Tomato 3cetic acid D. Vinegar 4. Oxalic acid A B C D A. 2 1 4 3 B. 2 4 1 3 C. 1 2 3 4 D. 2 1 3 4 

Answer: A
Explanation: Correct combination of the above list with some other important natural sources containing acids is as follow: Natural source Acid Vinegar Acetic acid Orange Citric acid Tamarind Tartaric acid Tomato Oxalic acid Sour milk (Curd)Lactic acid Lemon Citric acid Ant sting Methanoic acid Nettle sting Methanoic acid 5. Which of the following is used in soda-acid fire extinguisher? A. Washing soda B. Bleaching powder C. Baking soda D. Sodium chloride 

Answer: C
Explanation: Baking soda is also known as sodium hydrogen carbonate (NaHCO3), sodium bicarbonate, sweet soda and food soda. It is also an ingredient of antacids. Due to its alkaline nature, it relieves the depression of excessive acid in the stomach.It is also used in soda-acid fire extinguishers. 6. Consider the following pairs: Match-I (Homogeneous mixture) Match-II (Alloy) 1. Mercury and any other metal Amalgam 2. Copper and Zinc Brass 3. Copper and tin Bronze 4. Lead and Tin Solder Which of the following pairs given above is/are correctly matched? A. 1 and 4 only B. 1 and 2 only C. 2 and 3 only D. 1, 2, 3 and 4 

Answer: D
Explanation: An alloy is a homogeneous mixture of two or more metals, or a metal and a non-metal. It is prepared by first melting the primary metal, and then, dissolving the other elements in it in definite proportions. It is then cooled to room temperature. If one of the metals is mercury, then the alloy is known as an amalgam. The electrical conductivity and melting point of an alloy is less than that of pure metals. For example, brass, an alloy of copper and zinc (Cu and Zn), and bronze, an alloy of copper and tin (Cu and Sn), are not good conductors of electricity whereas copper is used for making electrical circuits. Solder, an alloy of lead and tin (Pb and Sn), has a low melting point and is used for welding electrical wires together. 7. With reference to Non-metals, consider the following statements: 1. Unlike metals, they are neither malleable nor ductile. 2. They are bad conductors of heat and electricity, except for graphite, which conducts electricity. 3. They react with hydrogen of dilute acids to form hydrides. Which of the statements given above is/are correct? A. 1 and 2 only B. 2 only C. 3 only D. 1, 2 and 3 

Answer: D
Explanation: 1. Non-metals have properties opposite to that of metals. They are neither malleable nor ductile. 2. They are bad conductors of heat and electricity, except for graphite, which conducts electricity. 3. Non-metals form negatively charged ions by gaining electrons when reacting with metals. 4. Non-metals form oxides which are either acidic or neutral. 5. Non-metals do not displace hydrogen from dilute acids. They react with hydrogen to form hydrides. 8. Regarding the physical properties of metals, consider the following pairs: 1. Malleability: metals can be beaten into thin sheets. 2. Ductility: the ability of metals to be drawn into thin wires. 3. Sonorous: metals that produce a sound on striking a hard surface. Which of the pairs given above is/are correctly matched? A. 1 only B. 2 and 3 only C. 3 only D. 1, 2 and 3 

Answer: D
Explanation: Some metals can be beaten into thin sheets. This property is called malleability. Gold and silver are the most malleable metals. The ability of metals to be drawn into thin wires is called ductility. Gold is the most ductile metal. It is because of their malleability and ductility that metals can be given different shapes according to our needs.The metals that produce a sound on striking a hard surface are said to be sonorous. For instance, school bells are made of metals. 9. Regarding the non-metals, consider the following statements: 1. Diamond is the hardest natural substance. 2. Graphite is the poor conductor of electricity. Which of the statements given above is/are correct? A. 1 only B. 2 only C. Both 1 and 2 D. Neither 1 nor 2 

Answer: A
Explanation: Carbon is a non-metal that can exist in different forms. Each form is called an allotrope. Diamond, an allotrope of carbon, is the hardest natural substance known and has a very high melting and boiling point. Graphite, another allotrope of carbon, is a good conductor of electricity. 10. Regarding the anodising, which of the following statements is correct? A. It is a process of forming a oxide layer on the non-metals. B. It is a process of removing the rust layer of metals. C. It is a process of removing the oxide layer from aluminium. D. It is a process of forming a thick oxide layer of aluminium. 

Answer: D
Explanation: Anodising is a process of forming a thick oxide layer of aluminium. Aluminium develops a thin oxide layer when exposed to air. This aluminium oxide coat makes it resistant to further corrosion. The resistance can be improved further by making the oxide layer thicker. During anodising, a clean aluminium article is made the anode and is electrolysed with dilute sulphuric acid. The oxygen gas evolved at the anode reacts with aluminium to make a thicker protective oxide layer. This oxide layer can be dyed easily to give aluminium articles an attractive finish. 1. Regarding the covalent bonds, which of the following statements is/are incorrect? 1. Bonds which are formed by the sharing of an electron pair between two atoms are known as Covalent Bonds. 2. Covalently bonded molecules are seen to have weak bonds within the molecule, but intermolecular forces are large. 3. Covalent compounds are poor conductors of electricity. Select the correct answer using the code given below: A. 1 and 2 only B. 2 only C. 1 and 3 D. 1, 2 and 3 

Answer: B
Explanation: The bonds which are formed by the sharing of an electron pair between two atoms are known as covalent bonds. Since the electrons are shared between atoms and no charged particles are formed, such covalent compounds are generally poor conductors of electricity as they do not have a free electron to move. Covalently bonded molecules are seen to have strong bonds within the molecule, but intermolecular forces are small. This gives rise to the low melting and boiling points of these compounds. 2. Which of the following structures are allotropes of carbon? 1. Diamond 2. Graphite 3. Fullerene Select the correct answer using the code given below: A. 1 and 2 only B. 2 and 3 only C. 1 and 3 only D. 1, 2 and 3 

Answer: D
Explanation: Diamond and Graphite are formed by carbon atoms, the difference lies in the manner in which the carbon atoms are bonded to one another. In Diamond, each carbon atom is bonded to four other carbon atoms forming a rigid three-dimensional structure. In graphite, each carbon atom is bonded to three other carbon atoms in the same plane giving a hexagonal array. Fullerenes forms another class of carbon allotropes. The first one to be identified was C-60 which has carbon atoms arranged in the shape of a football. 3. Regarding ethanol, which of the following statements is incorrect? A. It is used in making tincture iodine and cough syrups. B. It tends to positively affect the central nervous system and makes it even stronger. C. Consumption of ethanol tends to slow metabolic processes. D. Ethanol is a liquid at room temperature. 

Answer: B
Explanation: Ethanol is a good solvent, it is also used in medicines such as tincture iodine, cough syrups, and many tonics. When large quantities of ethanol are consumed, it tends to depress the central nervous system. This results in lack of coordination, mental confusion, drowsiness, lowering of the normal inhibitions, and finally stupor. Consumption of ethanol tends to slow metabolic process. Alcohol contains empty calories and has no nutritional value. It often contributes to malnutrition because the high levels of calories in most alcoholic drinks can account for a large percentage of your daily energy requirementsEthanol is liquid at room temperature. Ethanol is commonly called alcohol and is the active ingredient of all alcoholic drinks. 4. With reference to alcohol, consider the following statements: 1. Sugarcane juice can be used to prepare molasses which is fermented to give alcohol (ethanol). 2. Alcohol can be used as an additive in petrol since it is a cleaner fuel. Which of the statements given above is/are correct? A. 1 only B. 2 only C. Both 1 and 2 D. Neither 1 nor 2 

Answer: C
Explanation: Sugarcane plants are one of the most efficient convertors of sunlight into chemical energy. Sugarcane juice is used to prepare molasses which is fermented to give alcohol (ethanol). Some countries now use alcohol as an additive in petrol since it is a cleaner fuel which gives rise to only carbon dioxide and water on burning in sufficient air (oxygen). As the Alcohol molecule contains oxygen, it allows the engine to more completely combust the fuel, resulting in fewer emissions and thereby reducing the occurrence of environmental pollution. Since alcohol is produced from plants that harness the power of the sun, it is also considered as renewable fuel. 5. Which of the following elements are responsible for the hardness of water? A. Hydrogen and Oxygen B. Calcium and Magnesium C. Carboxylic acid D. Sodium hydroxide 

Answer: B
Explanation: Bathing foam forms an insoluble substance (scum), this is caused by the reaction of soap with the calcium and magnesium salts, which cause the hardness of water. This problem is overcome by using another class of compounds called detergents as cleansing agents. Detergents are generally ammonium or sulphonate salts of long chain carboxylic acids. The charged ends of these compounds do not form insoluble precipitates with the calcium and magnesium ions in hard water. Thus, they remain effective in hard water. Detergents are usually used to make shampoos and products for cleaning clothes. 6. Match List-I with List-II and select the correct 
answer: List-I (Scientist) List-II (Work/theory) A. Johann Wolfgang Döbereiner 1. Law of Octaves B. John Newlands 
2. Coinehe term ‘Triads’ C. Dmitri Ivanovich Mendeléev 3. Modern Periodic Table D. Henry Moseley 4. Early development of the periodic table of elements. A B C D A. 4 1 3 2 B. 2 1 4 3 C. 3 4 1 2 D. 2 3 4 1 

Answer: B
Explanation: In the year 1817, Johann Wolfgang Döbereiner, a German chemist, tried to arrange the elements with similar properties into groups. He identified some groups having three elements each. So he called these groups 'triads'. In 1866, John Newlands, an English scientist arranged the then known elements in the order of increasing atomic masses. He compared the table to the octaves found in music. Therefore, he called it the ‘Law of Octaves’. The main credit for classifying elements goes to Dmitri Ivanovich Mendeléev, a Russian chemist. He was the most important contributor to the early development of a Periodic Table of elements wherein the elements were arranged on the basis of their fundamental property, the atomic mass, and also on the similarity of chemical properties. In 1913, Henry Moseley showed that the atomic number of an element is a more fundamental property than its atomic mass as described below. Accordingly, Mendeléev’s Periodic Law was modified and atomic number was adopted as the basis of Modern Periodic Table. 7. With reference to Mendeléev’s Periodic Table, consider the following statements: 1. Mendeléev arranged the elements in the order of their increasing atomic masses. 2. Mendeléev left some gaps in Periodic Table Which of the statements given above is/are correct? A. 1 only B. 2 only C. Both 1 and 2 D. Neither 1 nor 2 

Answer: C
Explanation: Mendeleev examined the relationship between the atomic masses of the elements and their physical and chemical properties. He observed that most of the elements got a place in a Periodic Table and were arranged in the order of their increasing atomic masses. It was also observed that there occurs a periodic recurrence of elements with similar physical and chemical properties. Mendeléev left some gaps in his Periodic Table. Instead of looking upon these gaps as defects, Mendeléev boldly predicted the existence of some elements that had not been discovered at that time. 8. Regarding the limitations of Mendeléev’s classification, consider the following statements: 1. The Position of Hydrogen was ambiguous. 2. Isotopes of all elements posed a challenge to Mendeleev’s Periodic Law. 3. The atomic masses do not increase in a regular manner in going from one element to the next. Which of the statements given above is/are correct? A. 1 and 2 only B. 2 only C. 1 and 3 only D. 1, 2 and 3 

Answer: D
Explanation: Certainly, no fixed position can be given to hydrogen in the Periodic Table. This was the first limitation of Mendeléev’s Periodic Table. He could not assign a correct position to hydrogen in his Table. Isotopes were discovered long after Mendeléev had proposed his periodic classification of elements. Isotopes of an element have similar chemical properties, but different atomic masses. Thus, isotopes of all elements posed a challenge to Mendeleev’s Periodic Law. Another problem was that the atomic masses do not increase in a regular manner in going from one element to the next. So it was not possible to predict how many elements could be discovered between two elements — especially when we consider the heavier elements. 9. Regarding the modern periodic table, consider the following statements: 1. Properties of elements are a periodic function of their atomic number. 2. There are 7 groups and 18 periods. Which of the statements given above is/are correct? A. 1 only B. 2 only C. Both 1 and 2 D. Neither 1 nor 2 

Answer: A
Explanation:  The Modern Periodic Law states: 'Properties of elements are a periodic function of their atomic number'.  The atomic number gives us the number of protons in the nucleus of an atom and this number increases by one in going from one element to the next.  Elements, when arranged in order of increasing atomic number Z, lead us to the classification known as the Modern Periodic Table.  Prediction of properties of elements could be made with more precision when elements were arranged on the basis of increasing atomic number.  The Modern Periodic Table has 18 vertical columns known as ‘groups’ and 7 horizontal rows known as ‘periods’. 10. Regarding the modern periodic table, which of the following statements is incorrect? A. The elements present in any one group have the same number of valence electrons. B. Each valence shell indicates that the outer shell is filled with electrons. C. The atomic radius decreases in moving from left to right along a period, whereas the atomic size increases down the group. D. Metallic character increases across a period and decreases down a group. 

Answer: D
Explanation: As the effective nuclear charge acting on the valence shell electrons increases across a period, the tendency to lose electrons will decrease. Down the group, the effective nuclear charge experienced by valence electrons is decreasing because the outermost electrons are farther away from the nucleus. Therefore, these can be lost easily. Hence metallic character decreases across a period and increases down a group. 1. Match List-I with List-II and select the correct 
answer: List-I (Plant hormones) LIST-II (Effects) A. Auxin 1. Promote cell division B. Gibberellins 2. Inhibits 
growth Cytokinins 3. It helps the cells to grow longer D. Abscisic acid 4. Help in the growth of the stem A. 1 2 3 4 B. 3 4 1 2 C. 2 3 1 4 D. 1 3 2 4 

Answer: B
Explanation: Different plant hormones help to coordinate growth, development and responses to the environment. They are synthesised at places away from where they act and simply diffuse to the area of action. 1. When growing plants detect light, a hormone called Auxin, synthesised at the shoot tip, helps the cells to grow longer. 2. Another example of plant hormones are Gibberellins which, help in the growth of the stem. 3. Cytokinins promote cell division, and it is natural that they are present in greater concentration in areas of rapid cell division, such as in fruits and seeds. These are examples of plant hormones that help in promoting growth. 4. But plants also need signals to stop growing. Abscisic acid is one example of a hormone which inhibits growth. Its effects include wilting of leaves. 2. Consider the following hormones: 1. Adrenaline 2. Testosterone 3. Estrogen Which of the above hormones, prepare the human body for fighting or running instantaneously? A. 1 only B. 2 only C. 2 and 3 only D. 1, 2 and 3 

Answer: A
Explanation:  Adrenaline is secreted directly into the blood and carried to different parts of the body and is known as "fight or flight" Hormone.  The target organs or the specific tissues on which it acts include the heart. As a result, the heart beats faster, resulting in supply of more oxygen to our muscles.  The blood to the digestive system and skin is reduced due to contraction of muscles around small arteries in these organs. This diverts the blood to our skeletal muscles.  The breathing rate also increases because of the contractions of the diaphragm and the rib muscles.  All these responses together enable the animal body to be ready to deal with the situations like fighting and running.  Testosterone and estrogen are the reproductive hormones such animal hormones are part of the endocrine system which constitutes a second way of control and coordination in our body. 3. Consider the following pairs: 1. Thyroid Gland: Thyroxine hormone 2. Pituitary Gland: Growth hormone 3. Pancreas: Insulin Which of the pairs given above is/are correctly matched? A. 1 and 2 only B. 2 only C. 1 and 3 only D. 1, 2 and 3 

Answer: D
Explanation: 1. Thyroxine regulates carbohydrate, protein and fat metabolism in the body so as to provide the best balance for growth. Iodine is necessary for the thyroid gland to make thyroxine hormone. In case iodine is deficient in our diet, there is a possibility that we might suffer from goitre. 2. Growth hormone is one of the hormones secreted by the pituitary. As its name indicates, growth hormone regulates growth and development of the body. If there is a deficiency of this hormone in childhood, it leads to dwarfism. 3. Insulin is the hormone which is produced by the pancreas and helps in regulating blood sugar levels. If it is not secreted in proper amounts, the sugar level in the blood rises causing many harmful effects. 4. Consider the following statements: 1. Voluntary actions are controlled by Hindbrain. 2. Hearing, Smell is controlled by Fore-Brain. Which of the above statements given above is/are Correct? A. 1 only B. 2 only C. Both 1 and 2 D. Neither 1 nor 2 

Answer: B
Explanation: Medulla in Hind Brain controls involuntary function such as blood pressure, salivation, vomiting. Forebrain is the main thinking part of brain .It has regions which receive sensory impulses from various receptors. Separate areas of the forebrain are specialised for hearing, smell and sight and so on. 5. Consider the following pairs: 1. Benzoic acid formed in our muscles during a physical activity leads to cramps. 2. The rate of breathing in aquatic organisms is much faster than that in terrestrial organisms. Which of the statements given above is/are correct? A. 1 only B. 2 only C. Both 1 and 2 D. Neither 1 nor 2 

Answer: B
Explanation: In anaerobic respiration, there is a lack of oxygen in our muscles. Pyruvate instead of breaking down to form carbondioxide and water, it disintegrates into lactic acid and its accumulation in our muscles leads to cramps. Aquatic organisms breathe much faster than terrestrial organisms as the amount of oxygen dissolved in water is less as compared to the amount of oxygen in air. To compensate this, breathing rate of aquatic organisms is faster. 6. Consider the following statements: 1. Plants store carbohydrates in the form of glycogen. 2. Human beings store carbohydrates in the form of starch. Which of the statements given above is/are incorrect? A. 1 only B. 2 only C. Both 1 and 2 D. Neither 1 nor 2 

Answer: D
Explanation: Carbon and energy requirements of an autotrophic organism are fulfilled by photosynthesis. It is the process by which autotrophs take in substances from the outside and convert them into stored forms of energy. This material is taken in the form of carbon dioxide and water which is then converted into carbohydrates, in the presence of sunlight and chlorophyll. The carbohydrates which are not used immediately are stored in the form of starch, which serves as the internal energy reserve to be used as and when required by the plant. Hence, Statement 1 is not Correct. The energy derived from the food human beings eat, is stored in their body in the form of glycogen. 7. Consider the following events about Photosynthesis: 1. Absorption of light energy by chlorophyll. 2. Reduction of carbon dioxide to carbohydrates. Which of the statements given above is/are correct? A. 1 only B. 2 only C. Both 1 and 2 D. Neither 1 nor 2 

Answer: C
Explanation: Photosynthesis is a process by which plants converts inorganic materials such as carbon dioxide and oxygen into carbohydrates using sunlight. This process occurs in the chloroplast of plants. Chlorophyll traps the sunlight to reduce carbon dioxide and water into oxygen and Glucose which allows the plant to grow. Hence, Statement 1 and 2 are Correct. Water splits into oxygen, hydrogen ions, and electrons to replace the lost electrons in light-dependent reaction. As hydrogen ions pass through ATP synthase, ATP is formed. 6CO2 + 6H2O (chlorophyll + sunlight) → C6H12O6 + 6O2 (Glucose) Carbon gets into the leaf through minute pores in leaves called stomata and water goes into the plant through its roots. Oxygen gets out of the cell through stomata. 8. Consider the following statements about Digestion: 1. The secretion is done by the small intestine. 2. Hydrochloric acid inhibits the functioning of Pepsin. 3. The Mucus protects the inner lining of the stomach from the action of the acid. Select the correct answer using the code given below: A. 1 only B. 2 and 3 only C. 3 only D. 1, 2 and 3 

Answer: C
Explanation: The digestion functions are taken care of by the gastric glands present in the wall of the stomach. These release Hydrochloric acid, a protein digesting enzyme called pepsin, and mucus. The Hydrochloric acid creates an acidic medium which facilitates the action of the enzyme ‘pepsin’. The mucus protects the inner lining of the stomach from the action of the acid, that causes "acidity" in adults. 9. With reference to the respiration in human beings, consider the following statements: 1. Cartilage blocks the air passage. 2. Haemoglobin has a very high affinity for oxygen. 3. Fine hairs in the nostril traps harmful microbes in respiration. Which of the statements given above is/are correct? A. 1 and 2 only B. 2 only C. 2 and 3 only D. 1, 2 and 3 

Answer: C
Explanation: In human beings, air is taken into the body through the nostrils. The air passing through the nostrils is filtered by fine hairs and harmful microbes are trapped. The passage is also lined with mucus which helps in this process. Rings of cartilage are present in throat which ensure that the air passage does not collapse. Once oxygen enters the blood from the lungs, it is taken up by haemoglobin (Hb) in the red blood cells and form oxyhaemoglobin thereby oxygen get circulated in the entire body. 10. Regarding blood pressure, consider the following statements: 1. The Blood pressure is higher in Veins than in Arteries. 2. The normal systolic pressure is about 120 mm of Hg and diastolic pressure is 80 mm of Hg. 3. Blood pressure is measured with an instrument called sphygmomanometer. Which of the statements given above is/are correct? A. 1 and 2 only B. 2 and 3 only C. 3 only D. 1, 2 and 3 

Answer: B
Explanation: The force that blood exerts against the wall of a vessel is called blood pressure. This pressure is much greater in arteries than in veins as arteries carry oxygenated blood from heart to other parts of body that emerge at high pressure and remains in the same. The pressure of blood inside the artery during ventricular systole (contraction) is called systolic pressure and pressure in artery during ventricular diastole (relaxation) is called diastolic pressure. The normal systolic pressure is about 120 mm of Hg and diastolic pressure is 80 mm of Hg. Blood pressure is measured with an instrument called sphygmomanometer. 1. Regarding 'reproduction', consider the following statements: 1. The basic event in reproduction is the creation of a DNA copy. 2. Two copies of DNA in a reproducing cell are completely identical to each other. 3. Reproduction is linked to the stability of population of species. Which of the statements given above is/are correct? A. 1 and 2 only B. 1 and 3 only C. 3 only D. 1, 2 and 3 

Answer: B
Explanation: The DNA in a cell nucleus is an information source for making proteins. If the information is changed, different proteins will be made. Different proteins will eventually lead to altered body designs. Therefore, a basic event in reproduction is the creation of a DNA copy. Cells use chemical reactions to build copies of their DNA. This creates two copies of the DNA in a reproducing cell, that will need to be separated from each other. No bio-chemical reaction is absolutely reliable. Therefore, it is only to be expected that the process of copying the DNA will have some variations each time. As a result, the DNA copies generated will be similar, but may not be identical to the original. A population of organisms fill well-defined places, or niches, in the ecosystem, by using their ability to reproduce. The consistency of DNA copying during reproduction is important for the maintenance of body design features that allows the organism to use that particular niche. Reproduction is therefore linked to the stability of population of species. 2. With reference to the male and female gametes, consider the following statements: 1. A female gamete is smaller than a male gamete and likely to be motile. 2. Male gametes are large and contains the food-stores. Which of the statements given above is/are correct? A. 1 only B. 2 only C. Both 1 and 2 D. Neither 1 nor 2 

Answer: D
Explanation: Conventionally, the motile germ cell is called the male gamete and the germ-cell containing the stored food is called the female gamete. One germ-cell is large and contains the food-stores while the other is smaller and likely to be motile. 3. With reference to the reproductive parts of flower, consider the following statements: 1. Stamen produces pollen grains. 2. The Swollen part of Carpel is known as ovary. 3. Cross Pollination involves transfer of pollen grains to the same flower. Which of the statements given above is/are correct? A. 1 only B. 2 and 3 only C. 1 and 3 only D. 1 and 2 only 

Answer: D
Explanation: Stamen is the male reproductive part and it produces pollen grains that are yellowish in colour. Carpel is present in the centre of a flower and is the female reproductive part. It is made up of three parts. The swollen bottom part is the ovary, middle elongated part is the style and the terminal part which may be sticky is the stigma. The ovary contains ovules and each ovule has an egg cell. The pollen needs to be transferred from the stamen to the stigma. If this transfer of pollen occurs in the same flower, it is referred to as self-pollination. On the other hand, if the pollen is transferred from one flower to another, it is known as cross pollination. This transfer of pollen from one flower to another is achieved by agents like wind, water or animals. 4. Regarding the male reproductive system, consider the following statements: 1. Sperm formation requires a higher temperature than the normal body temperature. 2. Testosterone triggers Changes at Puberty. Which of the statements given above is/are correct? A. 1 only B. 2 only C. Both 1 and 2 D. Neither 1 nor 2 

Answer: B
Explanation: The formation of germ-cells or sperms takes place in the testes. These are located outside the abdominal cavity in scrotum because sperm formation requires a lower temperature than the normal body temperature. The role of the testes in the secretion of the hormone, testosterone, is to regulate the formation of sperms, and brings about changes in appearance, seen in boys at the time of puberty such as broadening of chest. In adolescent boys, sometimes, the muscles of the growing Voice Box go out of control and the voice becomes hoarse. 5. Regarding the female reproductive system, consider the following statements: 1. Fertilization takes place in the female's Uterus. 2. The embryo gets nutrition from the mother’s blood with the help of a special tissue called placenta. Which of the statements given above is/are incorrect? A. 1 only B. 2 only C. Both 1 and 2 D. Neither 1 nor 2 

Answer: C
Explanation: The ovary gets produced in the ovaries in the female. The fertilization of gametes takes place in the fallopian tube. The fertilised egg, the zygote, gets implanted in the lining of the uterus, and starts dividing. The embryo gets nutrition from the mother’s blood with the help of a special tissue called placenta. This is a disc which is embedded in the uterine wall. It contains villi on the embryo’s side of the tissue. On the mother’s side are blood spaces, which surround the villi. 6. Consider the following: 1. Menstruation takes place when egg is not fertilized. 2. HIV gets transmit by eating with an infected person. Which of the statements given above is/are incorrect? A. 1 only B. 2 only C. Both 1 and 2 D. Neither 1 nor 2 

Answer: A
Explanation: Menstruation is a periodic cycle that recurs every month and lasts for about two to eight days. When egg does not gets fertilized, the thin lining of uterus which nourishes the embryo in event of fertilization, breakdowns and comes out of vagina as blood and mucous HIV is a Sexually Transmitted Disease which by mere contact (activities such as playing or eating) with infected person does not get transferred. 7. Consider the following statements: 1. The number of successful variations in second generation is maximised by sexual reproduction than by asexual mode of reproduction. 2. Dominant and Recessive traits get expressed in sexual reproduction. Which of the statements given above is/are correct? A. 1 only B. 2 only C. Both 1 and 2 D. Neither 1 nor 2 

Answer: C
Explanation: In asexual reproduction, if one bacterium divides, and then the resultant two bacteria divide again, the four individual bacteria generated would be very similar. There would be only very minor differences between them, generated due to small inaccuracies in DNA copying. However, if sexual reproduction is involved, greater diversity will be generated, and the second generation will have differences that they inherit from the first generation, as well as newly created differences. Variations generated during birth can be hereditary. Due to these differences, survival of the organism can increase. Two copies of the trait are inherited in each sexually reproducing organism. These two may be identical, or may be different, depending on the parentage. For example, both TT and Tt are tall plants, while only tt is a short plant. In other words, a single copy of ‘T’ is enough to make the plant tall, while both copies have to be ‘t’ for the plant to be short. Traits like ‘T’ are called dominant traits, while those that behave like ‘t’ are called recessive traits. The trait that gets expressed in phenotype is called the dominant trait and the other which fails to express is called the recessive trait. 8. Regarding sex determination, consider the following statements: 1. Sex is the determined by temperature in some species. 2. Sex is not genetically determined in snails. 3. The sex of human children is determined by which chromosome they inherit from their father. Which of the statements given above is/are correct? A. 1 and 2 only B. 2 only C. 1 and 3 only D. 1, 2 and 3 

Answer: D
Explanation: Some species rely entirely on environmental cues. Thus, in some animals, the temperature at which fertilised eggs are kept determines whether the animals developing in the eggs will be male or female. In other animals, such as snails, individuals can change sex, indicating that sex is not genetically determined. . In human beings, the sex of the individual is largely genetically determined. Most human chromosomes have a maternal and a paternal copy, and we have 22 such pairs. But one pair, called the sex chromosomes, is different in men and women. So women are XX, while men are XY. Thus, the sex of the children will be determined by what they inherit from their father. A child who inherits an X chromosome from her father will be a girl, and one who inherits a Y chromosome from him will be a boy. 9. Which of the following places, witnesses the emergence of a modern human species? A. Europe B. Africa C. India D. North America 

Answer: B
Explanation: The earliest members of human species, Homo sapiens, arose in Africa, moved across continents and developed into distinct races. 10. Speciation is a process that leads to: A. formation of new species from older ones B. formation of clones C. formation of taller organisms D. formation of dwarf organisms 

Answer: A
Explanation: Speciation is defined as the formation of new species from an existing species, either by evolution or by genetic modification. A species is a group of organisms with similar characteristics and can interbreed to give a fertile offspring. Speciation is an evolutionary process of the formation of new and distinct species. The new species get reproductively isolated from the previous species i.e., the new species can’t reproduce with the old species. 11. With reference to the fossils, consider the following statements: 1. Preserved traces of living organisms, in form of their impressions on mud are called fossils. 2. Fossil dating is done by detecting the ratios of different isotopes of the same element in the fossil material. Which of the statements given above is/are correct? A. 1 only B. 2 only C. Both 1 and 2 D. Neither 1 nor 2 

Answer: C
Explanation: If a dead insect gets caught in hot mud, for example, it will not decompose quickly, and the mud will eventually harden and retain the impression of the body parts of the insect. All such preserved traces of living organisms are called fossils. There are two methods of fossil dating. One is relative. If we dig into the earth and start finding fossils, it is reasonable to suppose that the fossils we find closer to the surface are more recent than the fossils we find in deeper layers. The second way of dating fossils is by detecting the ratios of different isotopes of the same element in the fossil material. 1. If an opaque object on the path of light becomes very small, light has a tendency to bend around it and not walk in a straight line. Which of the following effects given below, defines the statement? A. Diffraction of light B. Refraction of light C. Light interference D. Reflection of light 

Answer: A
Explanation: If an opaque object on the path of light becomes very small, light has a tendency to bend around it and not walk in a straight line - this effect is known as the diffraction of light. 2. Regarding the properties of the image formed by a plane mirror, consider the following statements: 1. Virtual and Erect images are formed by Plane Mirror. 2. The size of the image is smaller to that of the object. Which of the statements given above is/are correct? A. 1 only B. 2 only C. Both 1 and 2 D. Neither 1 nor 2 

Answer: A
Explanation: The properties of the image formed by a plane mirror are: 1. Image formed by a plane mirror is always virtual and erect. 2. The size of the image is equal to that of the object. 3. The image formed is as far behind the mirror as the object is in front of it. 4. The image formed is laterally inverted. 3. Regarding the lens, consider the following statements: 1. A transparent material bound by two surfaces, of which one or both surfaces are spherical, forms a lens. 2. A lens, either a convex lens or a concave lens, has two spherical surfaces. Which of the statements given above is/are correct? A. 1 only B. 2 only C. Both 1 and 2 D. Neither 1 nor 2 

Answer: C
Explanation: A transparent material bound by two surfaces, of which one or both surfaces are spherical, forms a lens. This means that a lens is bound by at least one spherical surface. In such lenses, the other surface would be plane. A lens may have two spherical surfaces, bulging outwards. Such a lens is called a double convex lens. It is simply called a convex lens. It is thicker at the middle as compared to the edges. Similarly, a double concave lens is bounded by two spherical surfaces, curved inwards. It is thicker at the edges than at the middle. A double concave lens is simply called a concave lens.A lens, either a convex lens or a concave lens, has two spherical surfaces. Each of these surfaces forms a part of a sphere. 4. Which of the following use concave mirror? 1. Torches 2. Vehicles headlights 3. Dentists to see larger images of the patient's teeth. 4. Shaving mirror Select the correct answer using the code given below: A. 1 and 2 only B. 2, 3 and 4 only C. 1, 3 and 4 only D. All of the above 

Answer: D
Explanation: Concave mirrors are commonly used in torches, searchlights and vehicles headlights to get powerful parallel beams of light as these mirrors possess the ability to focus parallel rays of light to a point, produce a large size image.They are often used as shaving mirrors to see a larger image of the face The dentists use concave mirrors to see large images of the teeth of patients.Large concave mirrors are used to concentrate sunlight to produce heat in solar furnaces. 5. Light travels the fastest in which of the following mediums? A. In air B. In vacuum C. In water D. In diamond 

Answer: B
Explanation: Light propagates with different speeds in different media. Light travels the fastest in vacuum with the highest speed of 3×108 m s-1. Vacuum is free space and there are no obstacle to slow down light as the refractive index of vacuum is very low. In air, the speed of light is only marginally less, compared to that in vacuum. It reduces considerably in glass or water. 6. Regarding the human eye, consider the following statements: 1. Light enters the eye through the cornea. 2. The pupil controls the amount of light entering the eye. 3. An inverted real image of the object is formed on the retina. Which of the statements given above is/are correct? A. 1 only B. 2 and 3 only C. 1 and 3 only D. 1, 2 and 3 

Answer: D
Explanation: Light enters the eye, through a thin membrane called cornea. Cornea forms a transparent bulge on the front surface of the eyeball. .Pupil is a pigmented layer of tissues that makes up the colored portion of the eye. Its primary function is to control the amount of light entering in the eye. The eye lens forms an inverted real image of the object on the retina. The retina is a delicate membrane having enormous number of light-sensitive cells. 7. Regarding the myopia, consider the following statements: 1. In a myopic eye, the image of a distant object gets formed behind the retina. 2. Myopia is caused by shortening of eyeball. Which of the statements given above is/are correct? A. 1 only B. 2 only C. Both 1 and 2 D. Neither 1 nor 2 

Answer: D
Explanation: Myopia is also known as near sightedness. A person with myopia can see nearby objects clearly but cannot see distant objects distinctly. Myopia is caused by two reasons: 1. Excessive curvature of eye lens. 2. Elongation of eyeball. A person with this defect has the far point nearer than infinity. Such a person may see clearly upto a distance of a few metres.In a myopic eye, the image of a distant object is formed in front of the retina and not at the retina itself. 8. Regarding the hypermetropia, consider the following statements: 1. A person with hypermetropia can see distant objects clearly. 2. Image is formed behind the retina. 3. Convex lens is used to correct this defect. Which of the statements given above is/are correct? A. 1 only B. 2 and 3 only C. 3 only D. All of the above 

Answer: D
Explanation: Hypermetropia is also known as farsightedness. A person with hypermetropia can see distant objects clearly but cannot see nearby objects distinctly. The near point, for the person, is farther away from the normal near point (25 cm). Such a person has to keep a reading material much beyond 25 cm from the eye for comfortable reading. This is because the light rays from a closeby object are focussed at a point behind the retina. This defect can be corrected by using a convex lens of appropriate power. Eye-glasses with converging lenses provide the additional focussing power required for forming the image on the retina. 9. Which of the following parts of eyes is transplanted during eye donation? A. Retina B. Ciliary muscles C. Pupil D. Cornea 

Answer: D
Explanation: Corneal blindness can be cured through corneal transplantation of donated eyes. Eye donors can belong to any age group or sex. People who use spectacles, or those operated for cataract, can still donate the eyes. People who are diabetic, have hypertension, asthma patients and those without communicable diseases can also donate eyes. 10. Which of the following is the correct cause of the twinkling of stars? A. Scattering of light B. Refraction of light C. Total internal reflection of light D. Diffraction of light 

Answer: B
Explanation: The twinkling of a star is due to atmospheric refraction. The starlight, on entering the earth’s atmosphere, undergoes refraction continuously, before it reaches the earth. The atmospheric refraction occurs in a medium of gradually changing refractive index. Since the atmosphere bends starlight towards the normal, the apparent position of the star is slightly different from its actual position. The star appears slightly higher (above) than its actual position when viewed near the horizon. Further, this apparent position of the star is not stationary, but keeps on changing slightly. Since the stars are very distant, they approximate point-sized sources of light. As the path of rays of light coming from the star goes on varying slightly, the apparent position of the star fluctuates and the amount of starlight entering the eye flickers - the star sometimes appears brighter, and at some other time, fainter, which gives the twinkling effect. 11. Which of the following phenomena are caused by the scattering of light? 1. The colour of clear sky is blue. 2. Twinkling of the stars. 3. The colour of the Sun at Sunrise and Sunset is red. A. 1 and 2 only B. 1, 2 and 3 C. 2 and 3 only D. 1 and 3 only 

Answer: D
Explanation: The sky appears blue during a clear cloudless day because the molecules in the air scatter blue light from the sun more than they scatter red light. Twinkling of stars is caused by the atmospheric refraction of light. Light from the sun near the horizon passes through thicker layers of air and travel larger distance in the earth’s atmosphere before reaching our eyes. Near the horizon, most of the blue light and other lights with shorter wavelengths are scattered away by the particles. Therefore, the light that reaches our eyes is of longer wavelength. This gives rise to the reddish appearance of the Sun. ACHIEVE IAS MCQ SERIES, DAY 37, SOLUTIONS 1. Consider the following pairs: 1. SI unit of electric charge - Ampere (A) 2. SI unit of electric current - Coulomb (C) 3. Instrument for measuring electric current in a circuit - Ammeter Which of the pairs given above is/are correctly matched? A. 1 only B. 2 and 3 only C. 1 and 3 only D. 3 only 

Answer: D
Explanation: The SI unit of electric charge is coulomb (C), which is equivalent to the charge contained in nearly 6 × 1018 electrons. The electric current is expressed by a unit called ampere (A). One ampere is constituted by the flow of one coulomb of charge in a second, that is, 1 A = 1 C/1 s. An instrument called ammeter measures electric current in a circuit. To measure the current in a circuit, it is connected in series in a circuit. 2. Which of the following laws states that- the electric current flowing through a metallic wire is directly proportional to the ACHIEVE IAS MCQ SERIES, DAY 37, SOLUTIONS potential difference V, across its ends, provided its temperature remains the same? A. Faraday's law B. Charles’s law C. Ohm’s law D. Fleming's law 

Answer: C
Explanation: In 1827, a German physicist Georg Simon Ohm (1787- 1854) found out the relationship between the current I, flowing in a metallic wire and the potential difference across its terminals. He stated that the electric current flowing through a metallic wire is directly proportional to the potential difference V, across its ends provided its temperature remains the same. This is called Ohm’s law. 3. Resistance of a conductor depends upon which of the following factors: 1. On its length 2. On its area of cross-section 3. On the nature of its material Which of the statements given above is/are correct? A. 1 only ACHIEVE IAS MCQ SERIES, DAY 37, SOLUTIONS B. 1 and 3 only C. 2 and 3 only D. 1, 2 and 3 

Answer: D
Explanation: On applying Ohm’s law, we observe that the resistance of a conductor depends (i) on its length, (ii) on its area of crosssection, and (iii) on the nature of its material. Precise measurements have shown that resistance of a uniform metallic conductor is directly proportional to its length (l) and inversely proportional to the area of cross-section (A). 4. Which of the following inorganic gases are filled in bulbs? 1. Nitrogen 2. Helium 3. Argon 4. Hydrogen A. 1 and 3 only B. 1 and 4 only C. 3 and 4 only D. 2 and 3 only ACHIEVE IAS MCQ SERIES, DAY 37, SOLUTIONS 

Answer: A
Explanation: The bulbs are usually filled with chemically inactive gas - nitrogen/ argon, to prolong the life of filament. The closed glass chamber of a bulb contains an inactive gas Argon or Nitrogen. The glass chamber cannot be filled with air as the presence of oxygen will cause the filament to burn and a vacuum will evaporate the filament. Inert gases like argon do not react and will maintain a particular pressure preventing filament from burning. 5. Which one of the following describes the direction of an electric current? A. Opposite to the direction of flow of electrons. B. In the direction of flow of electrons. C. In the direction of flow of protons. D. Opposite to the direction of flow of protons. 

Answer: A
Explanation: A stream of electrons moving through a conductor constitutes an electric current. Conventionally, the direction of current is taken opposite to the direction of flow of electrons. 6. Which of the following scientists had first observed the magnetic effect of electric current? ACHIEVE IAS MCQ SERIES, DAY 37, SOLUTIONS A. Henry B. Oersted C. Faraday D. Volt 

Answer: B
Explanation: Hans Christian Oersted, one of the leading scientists of the 19th century, played a crucial role in understanding electromagnetism. In 1820 he accidentally discovered that a compass needle got deflected when an electric current passed through a metallic wire placed nearby. Through this observation Oersted showed that electricity and magnetism are related phenomena. His research later created technologies such as radio, television and fiber optics. The unit of magnetic field strength has been named as Oersted in his honor. 7. If the current through the wire increases, the magnitude of the magnetic field produced at a given point: A. Increases B. Decreases C. Firstly increases and then decreases D. Can’t comment anything 
Answer: A ACHIEVE IAS MCQ SERIES, DAY 
37, SOLUTIONS
Explanation: The current flowing through a conductor is directly proportional to magnetic field. As the electric current through the wire increases, the magnitude of the magnetic field produced at a given point increases. 8. Consider the following statements: 1. The Direct Current does not change its direction with time. 2. A current, which changes direction after equal intervals of time, is called an alternating current. 3. Alternating current is used in transmissions over long distances. Which of the statements given above is/are correct? A. 1 only B. 1 and 2 only C. 2 and 3 only D. All of the above 

Answer: D
Explanation: A current, which changes direction after equal intervals of time, is called an alternating current while direct current (DC) does not change its direction with time. The difference between the direct and alternating currents is that the direct current always flows in one direction, whereas the ACHIEVE IAS MCQ SERIES, DAY 37, SOLUTIONS alternating current reverses its direction periodically. Most power stations constructed these days produce alternating current (AC). In India, the AC changes direction after every 1/100 second, that is, the frequency of AC is 50 Hz. An important advantage of AC over DC is that electric power can be transmitted over long distances without much loss of energy. 9. Which of the following statements is the phenomenon of electromagnetic induction? A. The process of charging a body. B. The process of generating magnetic field due to a current passing through a coil. C. Producing induced current in a coil due to relative motion between a magnet and the coil. D. The process of rotating a coil of an electric motor. 

Answer: B
Explanation: The phenomenon of electromagnetic induction is the production of induced current in a coil placed in a region where the magnetic field changes with time. The magnetic field may change due to a relative motion between the coil and a magnet placed near to the coil. If the coil is placed near to a current-carrying conductor, the magnetic field may change either due to a change in the current through the conductor or due to the relative motion between the coil and conductor. ACHIEVE IAS MCQ SERIES, DAY 37, SOLUTIONS 10. Which of the following organs in the human body produce the magnetic field? 1. Brain 2. Kidney 3. Heart 4. Liver A. 1 and 2 only B. 2, 3 and 4 only C. 1 and 3 only D. All of the above 

Answer: C
Explanation: Two main organs in the human body where the magnetic field gets produced are the heart and the brain. The magnetic field inside the body forms the basis of obtaining the images of different body parts. This is done using a technique called Magnetic Resonance Imaging (MRI). Analysis of these images helps in medical diagnosis. Magnetism has, thus also got an important application in medicine. ACHIEVE IAS MCQ SERIES, DAY 37, SOLUTIONS ACHIEVE IAS MCQ SERIES, DAY 38, SOLUTIONS 1. Different sources are used in India to meet the energy requirement. Depending on the consumption of energy from different sources, select the correct answer from the code given below in descending order: A. Coal > Water > Petroleum and Natural Gas > Nuclear energy B. Coal > Petroleum and Natural Gas > Water > Nuclear energy C. Coal > Petroleum and Natural Gas > Nuclear energy > Water D. Coal > Nuclear energy > Water > Petroleum and Natural Gas 

Answer: A
Explanation: The following sources of energy are used to meet the energy requirements in India, in descending order: Coal > Water > Petroleum and Natural Gas > Nuclear energy. 2. Consider the following statements: 1. Hydro power plants convert the potential energy of falling water into electricity. 2. Construction of big dams generates greenhouse gas. Which of the statements given above is/are correct? A. 1 only B. 2 only ACHIEVE IAS MCQ SERIES, DAY 38, SOLUTIONS C. Both 1 and 2 D. Neither 1 nor 2 

Answer: C
Explanation: Hydro power plants convert the potential energy of falling water into electricity. In order to produce hydel electricity, high-rise dams are constructed on the river to obstruct the flow of water and thereby collect water in larger reservoirs. The water level rises and in this process the kinetic energy of flowing water gets transformed into potential energy. But, constructions of big dams have certain problems associated with it. Large ecosystems get destroyed when submerged under the water in dams. The vegetation which is submerged, rots under anaerobic conditions and gives rise to large amounts of methane which is a greenhouse gas. 3. Consider the following statements: 1. The fuels which are produced by plants and animals are called biomass. 2. The main Constituent of Biogas is Methane. Which of the statements given above is/are correct? A. 1 only ACHIEVE IAS MCQ SERIES, DAY 38, SOLUTIONS B. 2 only C. Both 1 and 2 D. Neither 1 nor 2 
Answer: C Explanation Cow dung cakes serve as a steady source of fuel. Since, these fuels are derived from plants 
and animalence they constitute - Biomass. These fuels, however, do not produce much heat on burning but a lot of smoke is given out when they are burnt. Cow-dung, various plant materials like crops residue, vegetable waste and sewage are decomposed in the absence of oxygen to give Biogas. Since the starting material is mainly cow-dung, it is popularly known as ‘gobar-gas’. Bio-gas is produced in a dome-shaped plant. Bio-gas is an excellent fuel as it contains up to 75% methane. It burns without smoke, leaves no residue like ash in wood, charcoal and coal burning. Its heating capacity is high. 4. The slurry left behind in biogas plant, contains which of the following elements? A. Potash and phosphorus B. Nitrogen and phosphorus C. Potash only D. Nitrogen only ACHIEVE IAS MCQ SERIES, DAY 38, SOLUTIONS 
Answer: B Explanation The slurry left behind in bio-gas plant is removed periodically and used as excellent manure, 
rich iitrogen and phosphorous. 5. Which of the following countries is called the country of 'winds'? A. Norway B. Canada C. Denmark D. Japan 

Answer: C
Explanation: Denmark is called the country of ‘winds’. More than 25% of their electricity needs are generated through a vast network of windmills. In terms of total output, Germany is the leader, while India is ranked fifth in harnessing wind energy for the production of electricity. 6. Which of the following elements is/are used in making solar cells? 1. Silicon 2. Astatine ACHIEVE IAS MCQ SERIES, DAY 38, SOLUTIONS 3. Sirium 4. Vanadium A. 1 only B. 3 and 4 only C. 1, 2 and 3 only D. All of the above 

Answer: A
Explanation: Solar cells convert solar energy into electricity. Silicon is used for making solar cells, which is abundant in nature, but availability of the special grade silicon for making solar cells is limited. The entire process of manufacture is still very expensive, silver used for interconnection of the cells in the panel further adds to the cost. The principal advantages associated with solar cells are that they have no moving parts, require little maintenance and work quite satisfactorily without the use of any focusing device. Another advantage is that they can be set up in remote and inaccessible hamlets or very sparsely inhabited areas in which laying of a power transmission line may be expensive and not commercially viable. ACHIEVE IAS MCQ SERIES, DAY 38, SOLUTIONS 7. Sea energy can be converted into electricity in oceanthermal-energy-conversion plants when- A. The temperature difference between the water at the surface and water at depths up to 2 km is 293 K (20°C) or more. B. There are narrow valleys on the coast where the dams can be built. C. There are shallow and wide coastal shores, where dams are to be built by the river. D. All of the above. 

Answer: A
Explanation: The water at the surface of the sea or ocean is heated by the Sun while the water in deeper sections is relatively cold. This difference in temperature is exploited to obtain energy in ocean-thermal-energy conversion plants. These plants can operate if the temperature difference between the water at the surface and water at depths up to 2 km is 293 K (20°C) or more. The warm surface-water is used to boil a volatile liquid like ammonia. The vapours of the liquid are then used to run the turbine of generator. The cold water from the depth of the ocean is pumped up and condense vapour again to liquid. ACHIEVE IAS MCQ SERIES, DAY 38, SOLUTIONS 8. Consider the following statements: 1. In the nuclear fission, the nucleus of a heavy atom can be split apart into lighter nuclei. 2. In a nuclear reactor designed for electric power generation, nuclear ‘fuel’ releases energy at a controlled rate. Which of the statements given above is/are correct? A. 1 only B. 2 only C. Both 1 and 2 D. Neither 1 nor 2 

Answer: C
Explanation: In Nuclear fission, the nucleus of a heavy atom (such as uranium, plutonium or thorium), when bombarded with low-energy neutrons, can be split apart into lighter nuclei. When this is done, a tremendous amount of energy is released if the mass of the original nucleus is just a little more than the sum of the masses of the individual products. The fission of an atom of uranium, for example, produces 10 million times the energy produced by the combustion of an atom of carbon from coal. In a nuclear reactor designed for electric power generation, such nuclear 'fuel' can be part of a self-sustaining fission chain reaction ACHIEVE IAS MCQ SERIES, DAY 38, SOLUTIONS that releases energy at a controlled rate. The released energy can be used to produce steam and further generate electricity. 9. Match List-I with List-II and select the correct answer using the code given below: List-I (Nuclear power reactors) List-II (States) A. Kalpakkam 1. Uttar Pradesh B. Narora 2. Gujarat C. Kakrapar 3. Tamil Nadu D. Tarapur4. Maharashtra A B C D A. 3 1 2 4 B. 3 1 4 2 C. 3 4 2 1 D. 3 4 1 2 

Answer: A
Explanation: Nuclear power reactors located at Tarapur (Maharashtra), Rana Pratap Sagar (Rajasthan), Kalpakkam (Tamil Nadu), Narora (UP), Kakrapar (Gujarat) and Kaiga (Karnataka) have the installed capacity of less than 3% of the total electricity generation capacity of our country. ACHIEVE IAS MCQ SERIES, DAY 38, SOLUTIONS 10. With reference to nuclear fusion, consider the following statements: 1. During the nuclear fusion reaction two lighter nuclei are joined to make a heavier nucleus. 2. Nuclear fusion reaction takes place in Sun and Star. It takes considerable energy to force the nuclei to fuse. Which of the statements given above is/are correct? A. 1 only B. 1 and 2 only C. 1, 2 and 3 D. 2 and 3 only 

Answer: C
Explanation: Fusion means joining lighter nuclei to make a heavier nucleus, most commonly hydrogen isotopesare added together to create Helium. It releases a tremendous amount of energy, according to the Einstein equation, as the mass of the product is little less than the sum of the masses of the original individual nuclei. Such nuclear fusion reactions are the source of energy in the Sun and other stars. It takes considerable energy to force the nuclei to ACHIEVE IAS MCQ SERIES, DAY 38, SOLUTIONS fuse. The conditions needed for this process are extreme - millions of degrees of temperature and millions of pascals of pressure. ACHIEVE IAS MCQ SERIES, DAY 39, SOLUTIONS 1. Which of the following statements best describes the term ecosystem? A. A community of people interacting with each other (organism). B. The part of the Earth that is inhabited by living organisms. C. The interacting organisms in an area together with the non-living constituents of the environment. D. Flora and fauna of a geographical area. 

Answer: C
Explanation: All the Biotic Components in an area together with the abiotic constituents of the environment form an ecosystem. Biotic components comprising living organisms and abiotic components comprising physical factors like temperature, rainfall, wind, soil and minerals2 2. Which of the following ecosystems is/are examples of natural ecosystem? 1. Ponds 2. Lakes 3. Crop-fields 4. Gardens Select the correct answer using the code given below: ACHIEVE IAS MCQ SERIES, DAY 39, SOLUTIONS A. 1, 2 and 4 only B. 2, 3 and 4 only C. 1 and 2 only D. 2 only 

Answer: C
Explanation: Forests, ponds and lakes are natural ecosystems while gardens and crop-fields are human made (artificial) ecosystems. 3. Regarding the producers, consider the following statements: 1. They make organic compounds from inorganic substances in the presence of sunlight and chlorophyll. 2. Fungi are Heterotrophs. Which of the statements given below is/are correct? A. 1 only B. 2 only C. Both 1 and 2 D. Neither 1 nor 2 
Answer: C ACHIEVE IAS MCQ SERIES, DAY 
39, SOLUTIONS
Explanation: Organisms which can make organic compounds like sugar and starch from inorganic substances using the radiant energy of the Sun in the presence of chlorophyll are called the producers or autotrophs. All green plants and certain blue green algae which can produce food by photosynthesis come under this category. All fungi are heterotrophs, they use enzymes to break down the materials on which they are growing, fungi are largely responsible for organic decomposition, they are also capable of fermentation. 4. Regarding the categories of consumers in ecosystem, which of the following types of organisms are called decomposers? 1. Virus 2. Bacteria 3. Fungus A. 1 only B. 2 and 3 only C. 3 only D. All of the above 
Answer: B ACHIEVE IAS MCQ SERIES, DAY 
39, SOLUTIONS
Explanation: The microorganisms, comprising Bacteria and Fungi, break-down the dead remains and waste products of organisms. These microorganisms are the decomposers as they break-down the complex organic substances into simple inorganic substances that go into the soil and are used up once more by the plants. Bacteria unlike viruses have their own enzymes and all the molecules to survive on their own as long as food is available. The Bacteria feed on the animal or dead animal and grow and therefore, decompose the bodies. Contrary to this, viruses are nonliving when outside the host. For them to replicate and produce more virus particles, they take the help of machinery of other living beings. When other animals are dead, they cannot support virus particles to be able to sustain their life. If the animal is dead and if it contains virus they will also die or remain there till they find a living and suitable host. 5. Regarding a food chain in an ecosystem, consider the following statements: 1. Food chain involves flow of energy from one component of the system to another. 2. The Food chain demonstrates the numbers of every organism that are eaten by others in line. Which of the statements given above is/are correct? A. 1 only B. 2 only ACHIEVE IAS MCQ SERIES, DAY 39, SOLUTIONS C. Both 1 and 2 D. Neither 1 nor 2 

Answer: A
Explanation: Food chain is the series of organisms feeding on one another. The food we eat acts as a fuel to provide us energy to do work. Thus the interactions among various components of the environment involves flow of energy from one component of the system to another. The length and complexity of food chains vary greatly. Each organism is generally eaten by two or more other kinds of organisms which in turn are eaten by several other organisms. So instead of a straight line food chain, the relationship can be shown as a series of branching lines called a food web. 6. Consider the following statements: 1. Producers convert solar energy into chemical energy. 2. The flow of energy is unidirectional in food chain. 3. The green plants in a terrestrial ecosystem capture about 100% of the energy of sunlight that falls on their leaves and convert it into food energy. Which of the statements given below is/are correct? A. 1 only B. 2 and 3 only ACHIEVE IAS MCQ SERIES, DAY 39, SOLUTIONS C. 1 and 2 only D. All of the above 

Answer: C
Explanation: The autotrophs capture the energy present in sunlight and convert it into chemical energy which remains stored in the form of Carbohydrates. The flow of energy is unidirectional. The energy that is captured by the autotrophs does not revert back to the solar input and the energy which passes to the herbivores does not come back to autotrophs. As it moves progressively through the various trophic levels it is no longer available to the previous level. The green plants in a terrestrial ecosystem capture about 1% of the energy of sunlight that falls on their leaves, rest gets dissipated. 7. What is the average value for the amount of organic matter that is present at each step and reaches the next level of consumers? A. 10% B. 20% C. 30% D. 50% 
Answer: A ACHIEVE IAS MCQ SERIES, DAY 
39, SOLUTIONS
Explanation: When green plants are eaten by primary consumers, a great deal of energy is lost as heat to the environment, some amount goes into digestion and in doing work and the rest goes towards growth and reproduction. An average of 10% of the food eaten is turned into its own body and made available for the next level of consumers.Therefore, 10% can be taken as the average value for the amount of organic matter that is present at each step and reaches the next level of consumers. 8. Which of the following statements defines Biological Magnification? A. Attack of an exotic species of plants in any ecosystem. B. The energy flow from one level to another level in the food chain. C. Concentration of harmful biochemicals from one level to another level in the food chain. D. The lack in elements for increasing the productivity of any ecosystem. 

Answer: C
Explanation: The chemicals which are used to protect the crops from pests are either washed down into the soil or into the water bodies. From the soil, these are absorbed by the plants along with water and minerals, and from the water bodies these are taken up by aquatic plants and animals. Thus they enter the food chain. ACHIEVE IAS MCQ SERIES, DAY 39, SOLUTIONS As these chemicals are not degradable, these get accumulated progressively at each trophic level. As Human beings occupy the top level in any food chain, the maximum concentration of these chemicals get accumulated in our bodies. This phenomenon is known as Biological Magnification. 9. Regarding the ozone layer, consider the following statements: 1. Ozone (O3) is a molecule formed by three atoms of oxygen. 2. Ozone shields the surface of the earth from ultraviolet radiation from the Sun. 3. Human synthesized chemicals such as chlorofluorocarbons (CFCs) have started declining the amount of ozone. Which of the statements given above is/are correct? A. 1 only B. 2 and 3 only C. 1, 2 and 3 D. 1 and 2 only 

Answer: C
Explanation: Ozone (O3) is a molecule formed by three atoms of oxygen. Ozone, is a deadly poison. However, at the higher levels of the atmosphere, ozone performs an essential function. It shields the surface of the earth from ultraviolet (UV) radiation from the Sun. ACHIEVE IAS MCQ SERIES, DAY 39, SOLUTIONS This radiation is highly damaging to organisms, for example, it is known to cause skin cancer in human beings. The amount of ozone in the atmosphere began to drop sharply in the 1980s. This decrease has been linked to synthetic chemicals like chlorofluorocarbons (CFCs) which are used as refrigerants and in fire extinguishers. In 1987, the United Nations Environment Programme (UNEP) succeeded in forging an agreement to freeze CFC production at 1986 levels. 10. Hydrogen bomb is based on which of the following reactions? A. Controlled fusion reaction B. Thermonuclear fusion reaction C. Controlled fission reaction D. Thermonuclear fission reaction 

Answer: B
Explanation: Hydrogen bomb is based on the 'nuclear fusion' technology. In order to release more energy, atoms have been fused together in the making of H-Bomb.Hydrogen Bombs use fusion, the same way that powers the Sun or any other star. Isotopes of hydrogen are forced together to release a much bigger blast — hundred times powerful than the nuclear weapon that have been used in warfare. ACHIEVE IAS MCQ SERIES, DAY 39, SOLUTIONS 1. The Ganga Action Plan was first introduced in which of the following years? A. 1982 B. 1985 C. 1988 D. 2000 

Answer: B
Explanation: The Ganga Action Plan was a multi-crore project which came about in 1985 because the quality of the water in the Ganga was very poor. The Ganga runs its course of over 2500 km from Gangotri in the Himalayas to Ganga Sagar in the Bay of Bengal. It is being turned into a drain by more than a hundred towns and cities in Uttar Pradesh, Bihar and West Bengal that pour their garbage and excreta into it. Largely untreated sewage is dumped into the Ganges every day. In addition, pollution is caused by other human activities like bathing, washing of clothes and immersion of ashes or unburnt corpses. And then, industries contribute chemical effluents to the Ganga pollution load and the toxicity kills fish in large sections of the river. 2. Amrita Devi Bishnoi National Award is given in which of the following fields? A. In the field of wildlife conservation B. In the soil conservation C. In the field of women empowerment D. In science 

Answer: B
Explanation: The Government of India instituted an ‘Amrita Devi Bishnoi National Award for Wildlife Conservation’ in the memory of Amrita Devi Bishnoi, who in 1731 sacrificed her life along with 363 others for the protection of ‘khejri’ trees in Khejarli village near Jodhpur in Rajasthan. 3. Consider the following statements: 1. The Chipko movement is associated with forest conservation. 2. The Chipko movement, got originated from Himachal Pradesh. 3. The Chipko movement began during the early 1970s. Which of the statements given above is/are correct? A. 1 and 2 only B. 1 and 3 only C. 2 and 3 only D. All of the above 

Answer: B
Explanation: Sunderlal Bahuguna, a noted environmentalist initiated - The Chipko Andolan (Hug the Trees Movement). It was the result of a grassroot level effort to end the alienation of people from their forests. The movement originated from an incident in a remote village called Reni in Garhwal, Uttarakhand. The movement began high-up in the Himalayas during the early 1970s. 4. Match List-I with List-II: List-I (Methodology of Water Harvesting) List-II (Related State) A. Pynes 1. Tamil Nadu B. Kulhs 2. Bihar C. Khadins 3. Himachal Pradesh D. Eris 4. Rajasthan A B C D A. 2 3 4 1 B. 2 3 1 4 C. 1 2 3 4 D. 4 2 1 3 

Answer: A
Explanation: Water harvesting is an age-old concept in India. Khadins, tanks and nadis in Rajasthan, Bandharas and Tals in Maharashtra, Bundhis in Madhya Pradesh and Uttar Pradesh, Ahars and Pynes in Bihar, Kulhs in Himachal Pradesh, Ponds in the Kandi Belt of Jammu region, Eris (tanks) in Tamil Nadu, Surangams in Kerala, Kattas in Karnataka are some of the ancient water harvesting, including water conveyance, structures still in use today. 5. Consider the following statements: 1. Coal and Petroleum have been derived from inorganic sources. 2. Carbon monoxide is the byproduct of limited supply of oxygen during combustion. Which of the statements given above is/are correct? A. 1 only B. 2 only C. Both 1 and 2 D. Neither 1 nor 2 

Answer: B
Explanation: Coal and Petroleum have been formed from biomass, in addition to carbon, these contain hydrogen, nitrogen and sulphur. When these are burnt, the products are carbon dioxide, water, oxides of nitrogen and oxides of sulphur.When combustion takes place in insufficient air (oxygen), carbon monoxide is formed instead of carbon dioxide. 6. Which of the following does not lead to the depletion of groundwater? A. Establishing thermal power plants B. Cultivation of high yielding varieties of crops C. Process of deforestation D. Process of afforestation 
Answer: D 7. Among the following choose the correct option which includes acts related to the three R's strategy which 
can bseful for conserving our natural resources? A. Recycle, regenerate, reuse B. Reduce, regenerate, reuse C. Reduce, reuse, redistribute D. Reduce, recycle, reuse 
Answer: D 8. In our country, there are attempts to increase the height of several existing dams like Tehri and 
Almati damcross the Narmada. Choose the correct statements among the following that are a consequence of raising the height of dams 1. Terrestrial flora and fauna of the area is destroyed completely 2. Dislocation of people and domestic animals living in the area 3. Valuable agricultural land may be permanently lost 4. It will generate permanent employment for people Choose the correct option from the following: A. 1 and 2 B. 1, 2 and 3 C. 2 and 4 D. 1, 3 and 4 

Answer: B
Explanation: A large area of land is covered in building the dams which causes the devastation of the terrestrial flora and fauna. Masses of people have to be displaced to other locations and also there is a loss of a large part of the valuable agricultural land. 9. Given below are a few statements related to biodiversity. Pick those that correctly describe the concept of biodiversity 1. Biodiversity refers to the different species of flora and fauna present in an area 2. Biodiversity refers to only the flora of a given area 3. Biodiversity is greater in a forest 4. Biodiversity refers to the total number of individuals of a particular species living in an area Choose the correct option from the following: A. 1 and 2 B. 2 and 4 C. 1 and 3 D. 2 and 3 

Answer: C
Explanation: The term biodiversity refers to the variety of life forms on Earth. Forests are rich in biodiversity as a number of forms of life are found in forests, including trees, plants, animals, fungi and microorganisms, and their roles in nature. 10. Which among the statements given below is incorrect? A. Sustainable development does not take into consideration the viewpoints of all stakeholders B. Sustainable development is a long planned and persistent development C. Economic development is linked to environmental development D. Sustainable development meets the current basic human needs along with preserving resources for future generations 

Answer: A
Explanation: The Sustainable Development Goals agenda was accepted by all members of the United Nations in 2012 at the Rio De Janeiro Council Meet with an aim to promote a healthy and developed future of the planet and its people. It was in 2015 when the Sustainable Development Goals were implemented after a successful fifteen-year plan of development called the Millennium Development Goals. It is a group of 17 goals with 169 targets and 304 indicators, as proposed by the United Nation General Assembly’s Open Working Group on Sustainable Development Goals to be achieved by 2030. Post negotiations, agenda titled “Transforming Our World: the 2030 agenda for Sustainable Development” was adopted at the United Nations Sustainable Development Summit. SDGs is the outcome of the Rio+20 conference (2012) held in Rio De Janerio and is a non-binding document. The 17 goals under the Sustainable Development Goals are as mentioned below: 1. End poverty in all its forms everywhere 2. End hunger, achieve food security and improved nutrition and promote sustainable agriculture 3. Ensure healthy lives and promote well being for all at all stages 4. Ensure inclusive and equitable quality education and promote lifelong learning opportunities for all 5. Achieve gender equality and empower all women and girls 6. Ensure availability ans sustainable management of water and sanitation for all 7. Ensure access to affordable, reliable, sustainable and modern energy for all 8. Promote sustained, inclusive and sustainable economic growth, full and productive employment and decent work for all 9. Built resilient infrastructure, promote inclusive and sustainable industrialisation and foster innovation 10. Reduce inequalities within and among countries 11.Make cities and human settlements inclusive, safe, resilient and sustainable 12. Ensure sustainable consumption and production pattern 13. Take urgent actions to combat climate change and its impact 14. Conserve and sustainably use the oceans, seas and marine resources 15. Protect, restore and promote sustainable use of terrestrial ecosystems, sustainably managed forests, combat desertification and halt and reverse land degradation and halt biodiversity loss 16. Promote peaceful and inclusive societies for sustainable development, provide access to justice for all and build effective, accountable and inclusive institutions at all levels 17. Strengthen the means of implementation and revitalise the global partnership for sustainable development Sustainable Development Goals in India India’s record in implementing Sustainable Development Goals  Mahatma Gandhi National Rural Employment Guarantee Act (MNREGA) is being implemented to provide jobs to unskilled labourers and improve their living standards.  National Food Security Act is being enforced to provide subsidized food grains.  The government of India aims to make India open defecation free by the year 2019 under its flagship programme Swachh Bharat Abhiyan.  Renewable energy generation targets have been set at 175 GW by 2022 to exploit solar energy, wind energy and other such renewable sources of energy efficiency and reduce the dependence on fossil fuels.  Atal Mission for Rejuvenation and Urban Transformation (AMRUT) and Heritage City Development and Augmentation Yojana (HRIDAY) schemes have been launched for improving the infrastructure aspects.  India has expressed its intent to combat climate change by ratifying the Paris Agreement. 